\begin{chapter}{Conclusions and Future Directions}
\label{ch:conclusions}

In this dissertation, we have presented a systematic methodology for improved
ab initio force field development on the basis of \sapt and \isa calculations.
Critically, the strategies developed herein allow for accurate and
physically-based modeling of short-range effects in intermolecular force
fields, and enable accurate, transferable, and cost-effective treatment of the
important atomic-level anisotropies commonly found in organic
compounds. These improved methodologies for ab initio force field development
have culminated in our \mastiff model for intermolecular interactions, and we
are already approaching the stage where \mastiff can be used in the generation of
broadly-applicable force fields for large-scale molecular simulation. 

Before such large-scale simulations can become a possibility for strongly
polarizable systems, fundamental limitations in the induction model for
\mastiff  will be need to be
adddressed. 
Future work on polar systems should focus on improved and more accurate treatment of the long-range polarization, 
the development of new models to more physically treat the polarization
damping, and the decomposition of the total induction energy into
charge-transfer and polarization contributions.
Though much work is needed to outline
specific strategies to tackle each of these issues, it is hoped that
regularized \sapt methods (\citen{Misquitta2013}), which can perform the
charge-transfer/polarization decomposition, and the \isa-pol method
(\cref{ch:workflow}), which can naturally partition the long-range polarization
energies and quantify higher-order and/or anisotropic polarizabilities, might
be of good service in this endeavor. 

Assuming challenges in modeling the induction energy can be met, a second goal
for future study should be the application of \mastiff to the wide range of
chemical problems where atomic-level anisotropy is particularly important.
These applications can initially consist of ab initio force field development
for specific molecules, and we are already in the process of developing and
testing models for industrially important compounds where
isotropic force fields have known accuracy issues, such as with benzene and
ethene.
%TODO: citations here
Emphasis should also be placed on generating transferable anisotropic ab initio force
fields for general \emph{classes} of molecules, such that the \mastiff model
can ultimately be employed as a general, all-purpose model for accurate molecular
simulation. The development of these general force fields will require us to address various
challenges not yet considered with the \mastiff methodology, such as the
treatment of flexible monomer geometries and the generalization of
atom-specific anisotropic parameters into transferable and general atom type parameters.
Nevertheless, and assuming these challenges can be overcome, it is hoped that
\mastiff and other `next-generation' ab initio force fields will lead to increasingly accurate
and robust models for molecular \glspl{pes},
such that the complex, inherently anisotropic details of
intermolecular interactions may be routinely studied in
large-scale molecular simulation.

\end{chapter}
