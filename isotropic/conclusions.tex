%auto-ignore
We have presented a new methodology for describing short-range intermolecular
interactions based upon a simple model of atom-in-molecule electron density
overlap. The resulting \isaffold is a simple extension of the conventional
Born-Mayer functional form, supplemented with atomic exponents determined from
an ISA analysis of the molecular electron density. 
In contrast to simple Born-Mayer or Lennard-Jones models, the \isaffold is capable of
reproducing ab initio interaction energies over a wide range of inter-atomic
distances, and displays
extremely low sensitivity to the details of parameterization. Furthermore, the
\isaffold exhibits excellent parameter transferability. We thus recommend
\isaffold for use in the development of future ab initio (and possibly
empirically-parameterized) potentials, particularly where accuracy across wide
regions of the potential surface is paramount. 

More generally, we find that analysis of the ISA densities provides an
excellent first-principles procedure for the determination of atomic-density
decay exponents.  This analysis improves upon existing approaches (which rely
upon exponents derived from atomic radii or ionization
potentials)\cite{Rappe1992, Mayo1990, Lim2009, VanDuin2001} and explicitly
incorporates the influence of the molecular environment.  These exponents can
be used within \isaffold without further parameterization.  Alternatively, in
conjunction with an appropriate scale factor, the exponents can be used to
enhance the accuracy of standard Born-Mayer potentials and/or Tang-Toennies
damping functions. The resulting \bmsisaff retains many of the advantages of
\isaffold, but also maintains compatibility with existing force fields and
simulations packages that do not support the Slater functional form.  
Given that the \bsisa exponents appear to be essentially
optimal with respect to additional empirical optimization, we strongly
recommend use of these first-principles exponents in order to simplify 
(both ab initio and empirical) future force field development involving Born-Mayer
or related functional forms.\cite{Gordon2006}

Overall, \isaffold enables a significantly increase in force field accuracy,
particularly in describing short intermolecular contacts. Nevertheless, the
neglect of atomic anisotropy remains, in some cases, a severe approximation.
\cite{Eramian2013, Badenhoop1997, Kim2014b}
Indeed, it has been shown by many
authors\cite{stone2013theory,Day2003,Totton2010a,Wheatley1990} that
quantitatively accurate \A parameters (and to a lesser extent, \B parameters)
require incorporation of angular dependence for the generation of
highly-accurate force fields. This anisotropy becomes crucial when describing
systems containing lone pairs, hydrogen bonds, and/or $\pi$-interactions.
Promisingly, \bsisa densities naturally describe such anisotropy, 
\cite{Wheatley2012,Misquitta2015a,Misquitta2015b}
and a straightforward method for its inclusion (where essential) in
ab initio force fields is the subject of \cref{ch:mastiff}.
