%auto-ignore
Molecular simulation is an essential tool for interpreting and predicting
the structure, thermodynamics, and dynamics of chemical and
biochemical systems.  The fundamental inputs into these simulations are the
intra- and intermolecular force fields, which provide simple and
computationally efficient descriptions of molecular interactions.
Consequently, the predictive and explanatory power of molecular simulations
depends on the fidelity of the force field to the underlying (exact) potential
energy surface.

In the case of intermolecular interactions, the dominant contributions for
non-reactive systems can be decomposed into the following
physically-meaningful
energy components: electrostatic, exchange-repulsion, induction and dispersion. 
\cite{stone2013theory,margenau1969theory,Riley2010,Stone2007,Dykstra2000}
At large intermolecular distances, where monomer electron overlap can be
neglected, the physics of intermolecular interactions can be described
entirely on the basis of monomer properties (e.g. multipole moments,
polarizabilities), all of which can be calculated with high accuracy from
first principles. 
\cite{Stone1984}
In conjunction with associated distribution schemes that decompose 
molecular monomer properties into atomic contributions,
\cite{stone2013theory,Stone2007,Williams2003,Misquitta2006,Dehez2001,Stone2005,Misquitta2014}
these monomer properties lead to an accurate and computationally efficient
model of `long-range' intermolecular interactions as a sum of atom-atom terms, which
can be straightforwardly included in common molecular simulation packages.

At shorter separations, where the molecular electron density
overlap cannot be neglected, the asymptotic description of
intermolecular interactions breaks down due to the influence of Pauli
repulsion, charge penetration and charge transfer.  These effects
can be quantitatively described using modern electronic
structure methods,\cite{Riley2010,Jeziorski1994,Szalewicz2012,Raghavachari1989,Grimme2011} 
but are far more challenging to model accurately using computationally
inexpensive force fields.
For efficiency and ease of parameterization, most simple force fields use a single
`repulsive' term to model the cumulative influence of (chemically distinct)
short-range interactions. 
These simple models have seen comparatively little progress over the past
eighty years, and the Lennard-Jones\cite{Lennard-Jones1931} (${A}/{r^{12}}$)
and Born-Mayer \cite{Born1932,Buckingham1938}
($A\exp(-Br)$) forms continue as popular descriptions of
short-range effects in standard force fields despite some well-known
limitations (\emph{vide infra}).  

Because the prediction of physical and chemical properties depends on
the choice of short-range interaction model,
\cite{Nezbeda2005,
Galliero2008,Gordon2006,Ruckenstein1997,Galliero2007,Wu2000,Errington1998,McGrath2010,
Parker2015,Sherrill2009,Zgarbova2010,
Bastea2003,Errington1999,Ross1980}
it is essential to develop sufficiently accurate short-range force fields.
This is particularly true in the case of ab initio force field development.
A principle goal
of such a first-principles approach is the reproduction of a calculated
potential energy surface (PES), thus (ideally) yielding accurate predictions
of bulk properties.
\cite{Schmidt2015}
 Substantial deviations between a fitted and calculated PES
lead to non-trivial challenges in the parameterization process, which
in turn can often degrade the quality of property predictions.
The challenge of reproducing an ab initio PES becomes particularly
pronounced at short inter-molecular separations, where many common force field
functional forms are insufficiently accurate.  For example, the popular
Lennard-Jones ($A/{r^{12}}$) functional form is well-known to be substantially
too repulsive at short contacts as compared to the exact
potential.
\cite{Abrahamson1963,Mackerell2004,Parker2015,Sherrill2009,Zgarbova2010}
While the Born-Mayer ($A\exp(-Br)$) functional form is more
physically-justified\cite{Buckingham1938} and fares
better in this regard,\cite{Abrahamson1963} substantial deviations often
persist.\cite{Halgren1992} In addition, parameterization of the Born-Mayer form
is complicated by the strong coupling of the pre-exponential ($A$) and exponent
($B$) parameters, hindering the transferability of the resulting force field.
These considerations, along with the observed sensitivity of
structural and dynamic properties to the treatment of short-range
repulsion,\cite{Nezbeda2005} highlight the need for new approaches
to model short-range repulsive interactions.

Our primary goal in this Chapter is to derive a simple and accurate description
of short-range interactions in molecular systems that improves upon both the
standard Lennard-Jones and Born-Mayer potentials in terms of accuracy,
transferability, and ease of parameterization. Our focus is on ab
initio force field development, and thus we will use the
fidelity of a given force field with respect to an accurate ab initio PES as a
principle metric of force field quality. We note
that other metrics may be more appropriate for the development of empirical
potentials, where Lennard-Jones or Born-Mayer forms may yield highly accurate
`effective' potentials when parameterized against select bulk properties.
Nonetheless, we anticipate that the models proposed in this Chapter may prove
useful for empirical force field development in cases where a more
physically-motivated functional form is necessary.
\cite{Parker2015,Sherrill2009,Zgarbova2010}

The outline of this Chapter is thus as follows: first, we derive a new functional form
capable of describing short-range repulsion from first principles, and show how the
standard Born-Mayer form follows as an approximation to this more exact model.
Our generalization of the Born-Mayer functional form allows for an
improved description of a variety of short-range effects,
namely electrostatic charge penetration, exchange-repulsion, and density
overlap effects on induction and dispersion. Crucially, we also demonstrate how the associated
atomic exponents can be extracted from
first-principles monomer charge densities via an iterated stockholder atoms
(ISA) density partitioning scheme, thereby reducing the number of required
fitting parameters compared to the Born-Mayer model.
Benchmarking this `Slater-ISA' methodology (functional form and atomic
exponents) against high-level ab initio
calculations and experiment, we find that the approach exhibits increased
accuracy, transferability, and robustness as compared to a typical
Lennard-Jones or Born-Mayer potential.
In addition, we show how the ISA-derived exponents
can be adapted for use within the standard Born-Mayer form (Born-Mayer-sISA),
while still retaining retaining many of the advantages of the Slater-ISA
approach. As such, our methodology also offers an opportunity to dramatically
simplify the development of both empirically-parameterized and ab initio
simulation potentials based upon the standard Born-Mayer form.
