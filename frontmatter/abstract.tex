%% \textbf{FIXME:  basically a placeholder; do not believe}
%% 
%% \svnidlong{$LastChangedBy$}{$LastChangedRevision$}{$LastChangedDate$}{$HeadURL: http://freevariable.com/dissertation/branches/diss-template/frontmatter/abstract.tex $}
%% \vcinfo{}

%% I did some research, read a bunch of papers, published a couple myself, (pick one):
%% \begin{enumerate}
%% 	\item ran some experiments and made some graphs,
%% 	\item proved some theorems
%% \end{enumerate}
%% and now I have a job.  I've assembled this document in the last couple of months so you will let me leave.  Thanks!

Molecular simulation is an essential tool for interpreting and predicting the
structure, thermodynamics, and dynamics of chemical and biochemical systems.
The fundamental inputs into these simulations are the intra- and
intermolecular force fields, which provide simple and computationally
efficient descriptions of molecular interactions.  Consequently, the utility
of molecular simulation ultimately depends on the fidelity of the force field
to the underlying (exact) potential energy surface.  This dissertation
describes a number of novel advances designed to improve the accuracy and
predictive power of (specifically ab initio) intermolecular force fields.  By
fitting ab initio force fields to first-principles-based functional forms and
chemically-meaningful parameters, and by taking frequent advantage of the
physically-motivated partitioning afforded by \acrfull{sapt} and \acrfull{isa} approaches, we
demonstrate how the resulting force fields can be applied to describe a broad
range of molecular systems in different chemical and physical
environments. 
Our newly-developed \mastiff approach
achieves quantitative accuracy with
respect to both high-level electronic structure theory and experiment, and is
thus well suited for use in `next-generation' ab initio force field development
and large-scale molecular simulation.


%% \cref{ch:isaff} describes an approach whereby the \isa partitioning scheme can
%% be used to develop new models
%% for the \sapt exchange-repulsion energy and related
%% short-range energy contributions. The methods in \cref{ch:isaff} neglect
%% important effects due to the orientation dependence, or `atomic-level
%% anisotropy', of the \isa charge densities, and \cref{ch:mastiff} extends the
%% original \isa-based method to account for this atomic-level anisotropy in an
%% accurate cost-efficient


%% Herein, we derive a novel short-range functional form based on a simple
%% Slater-like model of overlapping atomic densities and an iterated stockholder
%% atom (ISA)
%% partitioning of the molecular electron density.
%% We demonstrate that this Slater-ISA methodology yields a more accurate,
%% transferable, and robust description of the short-range interactions
%% at minimal additional computational cost compared to standard
%% Lennard-Jones or Born-Mayer approaches.
%% 
%% Nearly all standard force fields employ the `sum-of-spheres' approximation,
%% which models intermolecular interactions
%% purely in terms of interatomic distances. In stark contrast to this
%% assumption, atoms in molecules can have significantly non-spherical shapes,
%% leading to interatomic interaction energies with strong orientation
%% dependencies.
%% Neglecting this `atomic-level
%% anisotropy' can lead to significant errors in predicting interaction energies,
%% and herein we propose a general and computationally-efficient model
%% (\mastiff) whereby atomic-level orientation dependence can be incorporated
%% into standard intermolecular force fields. Importantly, our \mastiff model
%% includes anisotropic effects, not only for long-range (multipolar)
%% electrostatics, but also for dispersion, exchange-repulsion, and charge
%% penetration. We benchmark \mastiff against various sum-of-spheres models
%% over a large library of intermolecular interactions, and find that \mastiff
%% achieves quantitative accuracy with
%% respect to both high-level electronic structure theory and experiment.
%% \mastiff is highly transferable and requires minimal additional
%% parameterization, and thus is
%% well suited for use in next-generation ab initio force field development.
%% 
%% 
%% 
%% Short-range repulsion within inter-molecular force fields
%% is conventionally described by either Lennard-Jones (${A}/{r^{12}}$)
%% or Born-Mayer ($A\exp(-Br)$) forms. Despite their widespread use,
%% these simple functional forms are often unable to describe the
%% interaction energy accurately over a broad range of inter-molecular distances,
%% thus
%% creating challenges in the development of ab initio force fields and
%% potentially leading to decreased accuracy and transferability.
%% Herein, we derive a novel short-range functional form based on a simple
%% Slater-like model of overlapping atomic densities and an iterated stockholder
%% atom (ISA)
%% partitioning of the molecular electron density.
%% We demonstrate that this Slater-ISA methodology yields a more accurate,
%% transferable, and robust description of the short-range interactions
%% at minimal additional computational cost compared to standard
%% Lennard-Jones or Born-Mayer approaches.
%% Finally, we show how this methodology can be adapted to yield the standard
%% Born-Mayer
%% functional form while still retaining many of the advantages
%% of the Slater-ISA approach.
