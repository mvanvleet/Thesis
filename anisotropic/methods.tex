\begin{subsection}{The 91 Dimer Test Set}

Our benchmarking procedures
are the same as in \cref{ch:isaff},\cite{VanVleet2016}
and we briefly summarize the relevant technical details. A full discussion of
results and example calculations are presented in \cref{sec:results}.

We have previously developed a large library of benchmark energies
for interactions between the following 13 atomic and organic species: acetone, argon, ammonia, carbon dioxide,
chloromethane, dimethyl ether, ethane, ethanol, ethene, methane, methanol,
methyl amine, and water. Using these 13 monomers, we have generated a library of dimer interaction
energies for each of the 91 possible unique dimer combinations (13 homomonomeric, 78
heteromonomeric). For each of these dimer combinations, interaction energies were
computed at a DFT-SAPT
\cite{Misquitta2002,Misquitta2003,Misquitta2005,Heßelmann2005a,Podeszwa2006a,Heßelmann2002,Heßelmann2003,Heßelmann2002a,Jansen2001}
level of theory for 1000 quasi-randomly chosen dimer configurations,
representing 91,000 benchmark interaction energies in total. 
As described below, parameters for a given force field methodology are then
fit on a component-by-component basis to reproduce the benchmark DFT-SAPT energies. 

\end{subsection}
\begin{subsection}{Parameter Determination}

We will present three types of force field fitting methodologies in this
Chapter,
termed \isoff, \isaff, and \anisoff (alternately referred to as \mastiff, as
discussed below). The nomenclature of each
name refers to, first, the isotropic/anisotropic treatment of multipolar
electrostatics and, second, the isotropic/anisotropic treatment of dispersion
and short-range effects. 
All studied force fields
use the following general functional form:
%
\begin{align}
%
\label{eq:ff_form}
\vtot &= \sum\limits_{ij} \vrep_{ij} + \velst_{ij} + \vind_{ij} + \vdhf_{ij} +
\vdisp_{ij} 
\intertext{where}
\begin{split}
\label{eq:ff_details}
\vrep_{ij} &= \Aex{ij} P(B_{ij}, r_{ij}) \exp(-B_{ij}r_{ij}) \\
\velst_{ij} &= -\Ael{ij} P(B_{ij}, r_{ij}) \exp(-B_{ij}r_{ij}) + \vmultipole
\\
\vind_{ij} &= -\Aind{ij} P(B_{ij}, r_{ij}) \exp(-B_{ij}r_{ij}) + \vdrudeind \\
\vdhf_{ij} &= -\Adhf{ij} P(B_{ij}, r_{ij}) \exp(-B_{ij}r_{ij}) +
\vdrudescf \\
\vdisp_{ij} &= - \Adisp{ij} \sum\limits_{n=3}^{6} f_{2n}(x) \frac{C_{ij,2n}}{r_{ij}^{2n}}
\\
P(B_{ij},r_{ij}) &= \frac13 (B_{ij} r_{ij})^2 + B_{ij} r_{ij} + 1 \\
\A &= A_iA_j \\
\B &= \sqrt{B_iB_j} \\
\C &= \sqrt{C_{i,2n}C_{j,2n}} \\
f_{2n}(x) &= 1 - e^{-x} \sum \limits_{k=0}^{2n} \frac{(x)^k}{k!} \\
x &= B_{ij}r_{ij} - \frac{2 B_{ij}^2 r_{ij} + 3 B_{ij} }
{ B_{ij}^2 r_{ij}^2 + 3 B_{ij} r_{ij} + 3} r_{ij}
\end{split}
\end{align}
%
For both \isoff and \isaff, $A_i$ is a fit parameter, and
$\Adisp{ij} = 1$. For \isoff (our completely isotropic model), the multipole expansion
\vmultipole is truncated to point charges, whereas \isaff and \mastiff both use a
multipole
expansion up to quadrupoles.
Finally, for
our anisotropic model, \mastiff, 
each $A_i$ is treated as an orientation-dependent function, and is represented
by the spherical harmonic
expansion 
\begin{align}
\label{eq:v_aniso}
\begin{split}
A_{i}(\theta_i,\phi_i) &= 
A_{i,\text{iso}}\big(1 + \aniso{} \big), \\
\aniso{} &\equiv \sum\limits_{l>0,k} a_{i,lk}  C_{lk}(\theta_i,\phi_i)
\end{split}
\end{align}
where $A_{i,\text{iso}}$ and $a_{i,lk}$ are fitted parameters with the
exception that $A_{i,\text{iso}}^{\text{disp}} = 1$.

%% %
%% \begin{align}
%% \label{eq:a_params}
%% A_{i}(\theta_i,\phi_i) = A_{i,\text{iso}}\left(1 + \sum\limits_{l>0k} a_{i,lk}
%% C_{lk}(\theta_i,\phi_i)\right)
%% \end{align}
%% %

Because DFT-SAPT provides a physically-meaningful energy decomposition into electrostatic,
exchange-repulsion, induction, and dispersion terms, parameters for each term in
\cref{eq:ff_form} are directly fit to model the corresponding DFT-SAPT energy
(see \citen{VanVleet2016} and references therein for details on the DFT-SAPT
terminology): 
%
\begin{align}
\begin{split}
\vrep \approx \erep &\equiv E^{(1)}_{\text{exch}} \\
\velst \approx \eelst &\equiv E^{(1)}_{\text{pol}} \\
\vind \approx \eind &\equiv E^{(2)}_{\text{ind}} + E^{(2)}_{\text{ind-exch}} \\
\vdhf \approx \edhf &\equiv \delta(\text{HF}) \\
\vdisp \approx \edisp &\equiv E^{(2)}_{\text{disp}} + E^{(2)}_{\text{disp-exch}}.
\end{split}
\end{align}
%
Fitting parameters on a component-by-component basis helps ensure parameter
transferability and minimizes reliance on error cancellation. Note that no
parameters are fit to reproduce the total energy and that,
because the DFT-SAPT energy decomposition is only calculated to second-order, 
third- and higher-order terms (mostly consisting of higher-order induction)
are estimated by \edhf.

\begin{subsubsection}{Parameters Calculated from Monomer Properties}

Of the parameters listed in \cref{eq:ff_details}, most do not need to be
fit to the DFT-SAPT energies, but can instead be calculated directly on the
basis of monomer electron densities. In particular, all multipolar
coefficients, $Q$, polarizabilities (involved in the calculation of \vdrude),
dispersion coefficients $C$, and atom-in-molecule exponents, $B^{\text{ISA}}$, are calculated in a manner nearly
identical to \citen{VanVleet2016}. Note that, for our atom-in-molecule exponents, we tested
the effects of treating $B^{\text{ISA}}$ both as a hard- and as a
soft-constraint in the final force field fit. While the conclusions from this
study are rather insensitive to this choice of constraint methodology, we have
found that the overall force field quality is somewhat improved by
relaxing the $B^{\text{ISA}}$ coefficients in the presence of a harmonic
penalty function (technical details of which can be
found in the Supporting Information of \citen{VanVleet2016}). The optimized $B$
coefficients in this Chapter are always within 5--10\% of the calculated $B^{\text{ISA}}$
coefficients from \cref{ch:isaff}, demonstrating the good accuracy of the $B^{\text{ISA}}$
calculations themselves.

As a second distinction from our prior work, and for reasons of compatibility
with the OpenMM\cite{Eastman2013} software we use for all molecular dynamics
simulations, here our molecular simulations use an induced
dipole model to describe polarization effects.
Numerical differences between this
model and the drude model used previously are very minor.
Additionally, the Thole-damping functions used in this Chapter follow
the same functional form used in the AMOEBA model,\cite{Ren2003} with a damping
parameter of 0.39. 

\end{subsubsection}
\begin{subsubsection}{Parameters Fit to Dimer Properties}

In addition to the soft-constrained $B$ parameters, all other free parameters
($A$ and $a$ parameters from \cref{eq:ff_form}
and \cref{eq:v_aniso}) are fit to reproduce
DFT-SAPT energies from the 91 dimer test set described above. For each dimer
pair, 4-5 separate optimizations (for exchange, electrostatics, induction,
\dhf, and, for \mastiff, dispersion) were carried out to minimize a weighted
least-squares error, with the weighting function given by a Fermi-Dirac functional
form,
%
\begin{align}
\label{eq:weighting-function}
w_i = \frac{1}{\exp(-E_i/kT) + 1},
\end{align}
%
where $E_i$ is the reference energy and 
the parameter $kT$, which sets the energy scale for the
weighting function, is calculated from an estimate of the global minimum well
depth, 
$E_{\text{min}}$, such that
$kT = 5.0 |E_{\text{min}}|$. 

\end{subsubsection}
\begin{subsubsection}{Local Axis Determination}

Identically to AMOEBA and other force fields that incorporate some degree of
atomic-level anisotropy,\cite{Ren2003,Day2003,Totton2010} we use a z-then-x
convention to describe the relative orientation of atomic species. By design,
the z-axis is chosen to lie parallel to the principal symmetry axis (or
approximate local symmetry axis) of an atom
in its molecular environment, and the xz-plane is similarly chosen to
correspond to a secondary symmetry axis or plane. Based on the assigned symmetry
of the local reference frame, many terms in the
spherical expansion of \cref{eq:gij} can then be set to zero, minimizing the
number of free parameters that need to be fit to a given atom type. 
%TODO: Add to SI
Representative local reference frames are shown for a few atom types in
\cref{fig:local_axis}, and a complete listing of anisotropic atom types
(along with their respective local reference frames and non-zero spherical
harmonic expansion terms)
are given in the \cref{sec:mastiff-local_axis_defs}.

\end{subsubsection}
\begin{subsubsection}{CCSD(T) Force Fields}

DFT-SAPT is known to systematically underestimate the interaction energies of
hydrogen-bonding compounds, and can also exhibit small but important errors
for dispersion-dominated compounds.\cite{Parker2014} Consequently, for
simulations involving \co, \cl, \nh, and \ho, we refit our SAPT-based force
fields to reproduce benchmark supermolecular, counterpoise-corrected CCSD(T)-F12a/aVTZ
calculations on the respective dimers. All calculations were performed using
the Molpro 2012 software.\cite{MOLPRO} Fits were still performed on a
component-by-component basis, with the energy of most components matching the
DFT-SAPT calculations used in \cref{ch:isaff}.\cite{VanVleet2016} However, so that
the total benchmark energy corresponded to the total interaction energy
calculated by CCSD(T)-F12a/aVTZ, the difference between coupled-cluster and
SAPT energies was added to the SAPT dispersion energy,
%
\begin{align}
\begin{split}
\vrep \approx \erep &\equiv E^{(1)}_{\text{exch}} \\
\velst \approx \eelst &\equiv E^{(1)}_{\text{pol}} \\
\vind \approx \eind &\equiv E^{(2)}_{\text{ind}} + E^{(2)}_{\text{ind-exch}} \\
\vdhf \approx \edhf &\equiv \delta(\text{HF}) \\
\vdisp \approx \edisp &\equiv E^{(2)}_{\text{disp}} +
E^{(2)}_{\text{disp-exch}} + \delta(\text{CC}),
\end{split}
\end{align}
%
where $\delta(\text{CC}) \equiv E_{\text{int}}^{\text{CCSD(T)-F12a}} -
E_{\text{int}}^{\text{DFT-SAPT\phantom{()}}}$.
%

In fitting these CCSD(T)-f12a-based force fields, and to account for small
errors in the original SAPT dispersion energy, we somewhat relaxed the
constraint that $\Adisp{} = 1$ for all atom types, and instead let $ 0.7 \le
\Adisp{} \le 1.3$. This constraint relaxation led, in some cases, to modest
improvements in the fitted potential.


\end{subsubsection}

\begin{subsubsection}{\co 3-body potential}
For the \co dimer, we developed a three-body model to
account for three-body dispersion effects. This three-body model is based on
the three-body dispersion Axilrod-Teller-Muto (ATM) type model developed by \citeboth{Oakley2009a}. These
authors fit the ATM term with the constraint that the total molecular $C_9$
coefficient be 1970 a.u. Based on our own calculations using a CCSD/AVTZ
level of theory,\cite{Korona2011} we have obtained an
isotropic molecular $C_9$ coefficient of 2246 a.u.; consequently, a 1.13 universal
scale factor was introduced to the Oakley potential so as to obtain dispersion
energies in line with this new dispersion coefficient.
\end{subsubsection}

\end{subsection}

%% \begin{subsection}{Comparison to Ab Initio Benchmarks}
%% 
%% As in previous work, root-mean-square (\rmse) and mean-signed errors (MSE),
%% both with respect to the DFT-SAPT reference energies,
%% were calculated for each methodology and for each dimer pair. Similarly,
%% `attractive \rmse/MSE' (a\rmse/aMSE) were computed by only considering the
%% subset of dimer configurations with net attractive total energies (as measured
%% by DFT-SAPT). After taking the absolute value of the MSE values, the
%% various error metrics were then averaged in the geometric mean sense to
%% obtain one `characteristic' \rmse or \mse for the entire 91 dimer
%% test set.
%% 
%% \end{subsection}
\begin{subsection}{Simulation Protocols}



\begin{subsubsection}{\deltahsub for \co}

For \co, the molar enthalphy of sublimation was determined according to 
%
\begin{align}
\begin{split}
\deltahsub &= H_{\text{g}} - H_{\text{crys}}  \\
           &= (U_{\text g} + PV_{\text g}) - (U_{\text{el,crystal,0K}} 
                +\Delta U_{\text{el,crystal,0K}\to T_{\text{sub}}} + PV_{\text{crys}} + E_{\text{vib,crystal}}) \\
           &\approx (RT) - \left(U_{\text{el,crystal,0K}} 
                + \int_{0K}^{T_\text{sub}} C_p dT \quad + E_{\text{vib,crystal}}\right) \\
\end{split}
\end{align}
%
which assumes ideal gas behavior and $PV_{\text{g}} >> PV_{\text{crys}}$.
For the crystal, an experimental measure of $C_p$ was obtained from \citen{Giauque1937} and numerically
integrated to obtain a value $\Delta U_{\text{el,crystal,0K}\to
T_{\text{sub}}} = 6.70 \kjmolold$. Theoretical measures of
$E_{\text{vib,crystal}} \approx 2.24 - 2.6 \kjmolold$ were obtained from
(respectively) \citen{Cervinka2017} and \citen{Heit2016a}, 
and
$U_{\text{el,crystal,0K}}$ was determined from the intermolecular force field
using a unit cell geometry taken from experiment.\cite{Simon1980}

\end{subsubsection}
\begin{subsubsection}{Other \co Simulations}

To determine the densities and enthalpies of vaporization used in this Chapter,
simulations were run in OpenMM using NPT and NVT ensembles, respectively. After an
equilibration period of at least 100ps, data was collected for a minimum of 500ps, and
uncertainties were calculated using the block averaging method. Average densities
were obtained directly from simulation, and 
the molar enthalpy of vaporization for \co was determined from the following
formula:
%
\begin{align}
\begin{split}
\deltahvap &= H_{\text{g}} - H_{\text{liq}} \\
           &= U_{\text{g}} - U_{\text{liq}} + P(V_{\text{g}} - V_{\text{liq}})
\end{split}
\end{align}
%
Note that, at the state points studied, the ideal gas approximation is
insufficiently accurate, and thus simulations were run for both the gas and
liquid phases at experimentally-determined
densities and pressures.\cite{Span1996}

\end{subsubsection}
\begin{subsubsection}{2\textsuperscript{nd} Virial Calculations}

Classical second virial coefficients were calculated for \nh, \ho, \co,
and \cl using rigid monomer geometries and following the procedure described in
\citen{McDaniel2013}.


\end{subsubsection}
\end{subsection}


