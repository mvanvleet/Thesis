Before presenting our development methodology for atomically-anisotropic
potentials, we provide the reader with a summary of prior approaches to ab
initio force field development and to models going beyond the sum-of-spheres
approximation.
In discussing the effects of anisotropic charge distributions on 
intermolecular potentials, here and throughout we employ the fairly standard\cite{Phipps2015}
decomposition of interaction energies into physically-meaningful components of electrostatics,
exchange-repulsion, induction (which includes both polarization and
charge-transfer), and dispersion. Many studies on atomically-anisotropic force
field development have focused on 
incorporating anisotropy on a component-by-component basis, and so for clarity we discuss
anisotropic modeling schemes for each energy component individually. 
As in \cref{ch:isaff},\cite{VanVleet2016} we find it useful to separate and
discuss in turn the so-called `long-range' effects (multipolar electrostatics,
polarization,
and dispersion)
from those `short-range' effects that arise only at
shorter intermolecular separations due to the non-negligible overlap of
monomer electron densities (e.g. charge penetration and exchange-repulsion).
Finally, we take advantage of the many-body
expansion\cite{Elrodt1997,Stone2007} to separately consider
into two- and many-body contributions, and primarily focus our discussion on
improvements to the two-body interaction energies themselves.


%% Lastly, and is also standard,\cite{Stone2007,McDaniel2014} we use a
%% `many-body expansion' to further decompose the total interaction
%% potential for a generic $N$-particle system into a sum of $n$-body
%% interactions:\cite{Elrodt1997,Stone2007}
%% %
%% \begin{align}
%% V_N(\vec r_1 ,\vec r_2 ,\dots,\vec r_N ) = 
%%     \sum\limits_{i < j}^{N} V_2(\vec r_i, \vec r_j) + 
%%     \sum\limits_{i < j < k}^{N} \Delta V_3(\vec r_i, \vec r_j, \vec r_k) + \dots
%% \end{align}
%% %
%% Here $V_2$ is referred to as the ``pair'' (alternately ``dimer'' or ``binary'') potential, and $\Delta V_3$ represents
%% the non-additive contributions 
%% to the interaction energies of 3-body clusters,
%% $\Delta V_3 = V_3 -  \sum_{i < j}^{3} V_2(\vec r_i, \vec r_j)$.
%% Aside from polarization, for which the complete $N$-body effects can be
%% straightforwardly evaluated,\cite{Stone2007,Rick2002} the many-body expansion is usually
%% rapidly convergent.\cite{Stone2007,stone2013theory}
%% %% , and typically only $V_2$ and
%% %%  $\Delta V_3$ are required to completely describe
%% %% $V_N$.\cite{Stone2007,stone2013theory} 
%% In fact, the sum of $V_2$ and $N$-body
%% induction accounts for upwards of 95\%
%% of the total interaction energy,\cite{McDaniel2014}
%% such that the accuracy of a given ab initio force field depends primarily on
%% the accuracy of the terms that describe the pair potential itself. 
%% For this reason, our focus in this work will be on the effects of atomic-level anisotropy
%% in descriptions of $V_2$ and the many-body induction. Additional
%% many-body effects can then be added, when important, to generate a complete
%% model of the $N$-body potential.\cite{McDaniel2014}

%
%DISCUSS: Should we include a discussion of prior work on anisotropic pair potentials?
%
% \cite{Pack1978}: Anisotropic pair-potential for CO2, one of the first of its
% kind.
%

\begin{subsection}{Prior Models for Long-Range Interactions}

The importance of atomic-level anisotropy in modeling long-range interactions,
particularly as it pertains to electrostatics, is quite well studied. A number
of groups have found that using atomic multipoles (rather than simple point
charges) greatly improves both the electrostatic potential\cite{Williams1988,Kramer2014} and 
the resulting electrostatic interaction energies.
\cite{Cardamone2014,Ren2003,Shi2013,Demerdash2014,Chaudret2014a,Giese2013,Cisneros2006,Elking2010}
Though not without additional computational cost, atomic multipoles are now
routinely employed in a number of popular force fields.
\cite{Ren2003,Shi2013,Cisneros2016a} 
% Cite other multipolar force fields?
As an alternate, and often more computationally-affordable approach, other groups have
used off-atom point charges to effectively account for anisotropic charge
densities.
\cite{Dixon1997,Harder2006,Rendine2011,Chaudret2013}
In line with chemical intuition, 
improvements from use of atomic multipoles/off-site charges are often particularly important in describing the
electric fields generated by heteroatoms and carbons in multiple bonding environments.
\cite{Mu2014,Wikfeldt2013}

The induction and dispersion energies have also been shown to exhibit
anisotropies that go beyond the sum-of-spheres model. For instance, it has
been suggested that anisotropic polarizabilities (which effect both polarization
and dispersion) are required to avoid an artificial over-stabilization of base
stacking energies in
biomolecules.\cite{Sponer2013} 
In order to more accurately treat polarization, several molecular mechanics potentials have
made use of either off-site\cite{Piquemal2007} or explicitly anisotropic
polarizabilities.\cite{Harder2006,Loboda2016}. 
Similarly, the importance of anisotropic dispersion interactions has also been
established,
\cite{Misquitta2008,Langhoff1971,Williams2003,Stone2007,Krishtal2011}
particularly for $\pi$-stacking interactions,\cite{Sponer2013,Zgarbova2010}
and select potentials have incorporated directional dependence into the functional
form for dispersion by expanding dispersion coefficients in terms of spherical
harmonics or, more generally,
\sfunc (discussed in \cref{sec:appendix}).\cite{Misquitta2008,Misquitta2016} 
% Cite s-functions

\end{subsection}
\begin{subsection}{Prior Models for Short-Range Interactions}

At closer intermolecular separations, where overlapping electron densities between
monomers leads to exchange-repulsion and charge-penetration effects, anisotropy is also
important. Exchange-repulsion has known orientation
dependencies which can play a quantitative role in 
halogen bonding\cite{Bartocci2015,Stone2013} and other chemical interactions, and many
%TODO: look up importance of exchange-repulsion for other types of
%interactions
authors have worked on developing different models for describing the
anisotropy of exchange-repulsion.
Some potentials (albeit not those that are
amenable to large-scale molecular simulation) have numerically computed overlap
integrals that can be used in conjunction with 
he density-overlap model popularised by
\citeauthor{Wheatley1990}\cite{Wheatley1990,Kita1976a,Kim1981,Nyeland1986,Ihm1990}
to quantify anisotropic exchange-repulsion, charge transfer, and/or charge
penetration interactions.
\cite{Duke2014a,Cisneros2006,Elking2010,Chaudret2014a,Gavezzotti2003,Torheyden2006}
Taking a more analytical approach, many other potentials have extended the Born--Mayer functional
form\cite{Born1932} to allow for orientation-dependent pre-factors,
\cite{Stone2007,Mitchell2001,Price2000,Stone1988,Day2003,Torheyden2006,Totton2010,Misquitta2016,Price2010a} 
and model short-range effects using an anisotropic functional form originally
proposed by \citeauthor{Stone1988}:
%
\begin{align}
\vrep_{ij} = G\exp[-\alpha_{ij}(R_{ij} - \rho_{ij}(\Omega_{ij}))].
\end{align}
%
Here $G$ is not a parameter, but rather an energy unit,\cite{stone2013theory}
%TODO: steal Stone's language to describe this
and $\alpha$ and $\rho$ represent, respectively, the hardness and shape of the
potential. In principle, one might also allow $\alpha$ to have orientation
dependence, however this seems unnecessary in practice, as the
hardness of the potential has been empirically found to behave more
isotropically than its shape.\cite{stone2013theory} Similar to treatments of
anisotropic electrostatics, this functional form typically expresses
orientation dependence, $\Omega_{ij}$, in terms of spherical harmonics
and/or $\bar{S}$-functions.\cite{stone2013theory}

Finally, we note that, aside from exchange-repulsion, 
we are aware of relatively little research on the development of simple
analytical expressions for the anisotropy of other overlap effects, such as electrostatic/inductive
charge penetration, charge-transfer, or short-range dispersion. 

\end{subsection}
