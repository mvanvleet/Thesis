We have developed a comprehensive methodology for modeling 
atomic-level anisotropy in standard intermolecular force fields. 
By treating this anisotropy through a simple extension of
standard isotropic force
fields,\cite{VanVleet2016}
%% and by accounting for this anisotropy in our models for each
%% electrostatics, exchange-repulsion, induction, and dispersion,
we have 
successfully demonstrated how this computationally-efficient treatment of atomic-level anisotropy leads to
significant improvements in models for intermolecular interactions.
Critically, 
and in contrast to popular assumption, we have shown how the accurate treatment of multipolar electrostatics 
does not \emph{by itself} account for all energetically-important effects of
atomic-level anisotropy.
Rather, our results indicate that anisotropy may need to be included in the
each electrostatic, exchange and dispersion terms in order to obtain intermolecular
force fields of the highest quality.
In the present study,
and in agreement with the more quantitative metrics proposed by others,\cite{Wheatley2012,Kramer2014} 
we have found
a comprehensive model of atomic-level anisotropy to be particularly important for obtaining sub-\kjmolold
accuracy for describing molecules with heteroatoms
(particularly ones with exposed lone pairs), carbons in multiple
bonding environments, and hydrogens bound to anisotropic heavy atoms. 
Our new intermolecular `\mastiff' force fields show great promise, not only with
respect to high-quality electronic structure benchmark energies, but also with
respect to experimental property predictions.
Importantly, \mastiff maintains high efficiency and transferability.
and can easily be implemented in common software packages such as 
OpenMM for use in condensed phase simulations.\cite{Eastman2013} 

Despite the advances presented in this Chapter, several aspects of our force
field methodology require further improvement, and will be the subject of
ongoing research. In particular, an improved description of induction effects
will become essential for accurate bulk simulations of highly polarizable
molecules such as water.  We are now actively working to develop improved
models that can describe both long-range anisotropic polarization and
short-range polarization damping, as these aspects of the force field
critically affect both the two- and many-body induction energies and
can account for a sizable fraction of the total interaction energy in
condensed phases.
We anticipate that these improved models for induction will, in
combination with an accurate description of three-body dispersion and
exchange, yield a general approach to force field development that captures
both the two- and many-body features of intermolecular interactions, in turn
enabling highly accurate, `next-generation' force field development capable of
simulating a wide array of phases and chemical environments.

