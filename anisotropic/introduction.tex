%auto-ignore
Classical molecular simulation is a standard tool for
interpreting and predicting the chemistry of an incredible host of systems ranging
from simple liquids to complex materials and biomolecules. Such simulations
always require, as input, a mathematical description of the
system's potential energy surface (PES). In principle, the
PES for most chemical systems can accurately be determined 
from one of several high-level electronic structure methods;
\cite{Rezac2016,Chalasinski2000,Jan2016}
nevertheless, these calculations are currently too expensive to use
in simulations of large systems and/or long timescales. 
\cite{Hassanali2014}
Consequently, most routine
molecular simulation today is performed with the aid of force fields:
computationally-inexpensive, parameterized mathematical expressions that
approximate the exact PES. Because the accuracy and predictive capabilities of
molecular simulation are directly tied to the underlying force
field, one of the central challenges of molecular
simulation is the development of highly accurate force fields. For ab initio
force field development, this accuracy is 
principally defined by a force field's fidelity to the underlying exact PES. 

As of now, several common
shortcomings\cite{Zgarbova2010} inhibit the accuracy and predictive capabilities of
standard ab initio force fields, and these limitations must be systematically addressed
in order to generate improved, `next-generation' force fields. One important
shortcoming, which will be the focus of this Chapter, is the so-called
`sum-of-spheres' approximation,\cite{Stone1988} in which it is assumed that the
non-bonding interactions between molecules can be treated as a superposition
of interactions between pairs of spherically-symmetric atoms.
Put differently, the sum-of-spheres, or `isotropic atom-atom', approximation assumes that the 
exact PES, \etot (which depends
both on the center of mass distance $R$ and relative orientation $\Omega$ between molecules),
can be modeled as
\begin{align}
\label{eq:sos_approx}
\etot(R,\Omega) \approx \sum\limits_{ij} f(r_{ij}) \equiv \vtot,
\end{align}
where the above sum runs over all
non-bonded pairs of atoms $i$ and $j$ with interatomic separation $r_{ij}$,
and $f(r_{ij})$ is an arbitrary, distance-dependent function that defines the pairwise
interaction.
Here and throughout, we use $E$ to denote the true PES, and $V$ to denote the
corresponding model/force field prediction. 
With some exceptions (vida infra), nearly all standard intermolecular force
fields ---
ranging from the popular ``Lennard-Jones plus point charges'' model to more complex
functional forms\cite{Schmidt2015}
---
explicitly make use of the isotropic atom-atom model.

Notwithstanding the popularity of the model, there is good experimental and
theoretical evidence to suggest that the sum-of-spheres approximation does not
hold in practice.\cite{stone2013theory,Stone1988,Price2000} 
Importantly, and as we argue in
\cref{sec:results}, models which include anisotropic (multipolar) electrostatics, but
otherwise employ the sum-of-spheres approximation, are an improved but
\emph{still incomplete} model for
describing the atomic-level anisotropy of intermolecular interactions.
Experimentally, it has long been known that 
atom-in-molecule charge densities, as determined from x-ray diffraction, can exhibit significant
non-spherical features, such as with lone pair or $\pi$ electron
densities.\cite{Coppens1979} Furthermore, statistical analyses of the
Cambridge Structural Database 
% cite this? is it the cambridge structural database or crystal structure
% database?
have shown that the the van der Waals radii of atoms-in-molecules (as measured
from interatomic closest contact distances)
are not isotropically
distributed, but rather show strong orientation dependencies, particularly for
halogens and other heteroatoms.
\cite{Bondi1964,Nyburg1985,Batsanov2001,Auffinger2004,Lommerse1996,Eramian2013} 
These experimental studies are corroborated by a significant body of
theoretical research on both the anisotropy of the atomic van der Waals radii
as well as the non-spherical features of the atomic charge densities themselves.
\cite{Wheatley2012,Kramer2014,Lommerse1996, Badenhoop1997a,Kim2014b,Bankiewicz2012}
These studies suggest
that the sum-of-spheres approximation is an insufficiently
flexible model for the subset of intermolecular interactions that arise from
atomically non-spherical charge
densities, and may help explain known difficulties in generating accurate
isotropic atom-atom force fields for such important chemical interactions as 
$\pi$-interactions,\cite{Chessari2002,Sponer2013,Sherrill2009}
$\sigma$-bonding,\cite{Bartocci2015,Rendine2011,Politzer2008}
and hydrogen bonding,\cite{Cisneros2016a}
(see \citen{Cardamone2014} and references therein).

Motivated by these observations,
a small but important  body of work has been
devoted to directly addressing the limitations of the isotropic atom-atom
model in the context of `next-generation' force field development. As will be discussed in
detail below (see \cref{sec:prior_work}), the general conclusion from these
studies is that many components of intermolecular interactions
(specifically electrostatics, exchange-repulsion, induction, and
dispersion)
can be more accurately modeled by functional forms that go beyond the
sum-of-spheres approximation.
\cite{Price2000,Hagler2015,Ren2003}
While few intermolecular potentials (and virtually no standard force fields
amenable to routine molecular simulation) explicitly account for atomic-level anisotropy
for each component of intermolecular interactions, several recent standard force fields
have incorporated atomic-level anisotropy into their description of 
long-range electrostatics.\cite{Cardamone2014} Some of these potentials
(notably AMOEBA\cite{Ponder2010,Ren2003,Shi2013} and some water
potentials\cite{Cisneros2016a,Cardamone2014}) 
are already employed in large-scale molecular simulation, often with very
encouraging success.\cite{Cardamone2014}
Furthermore, others have shown that anisotropic potentials (some of which additionally
model the anisotropy of exchange-repulsion
and/or dispersion) lead to significant improvements in predicting 
molecular crystal structures.
\cite{Cardamone2014,Price2010a,Day1999,Day2003,Price2008,Misquitta2016,Misquitta2008a}
%TODO: Cite SIBFA and NEMO here?
These and other results strongly 
%% indicate that formulating a complete
%% model for atomic-level anisotropy is 
%% a promising strategy for the development of improved `next-generation' intermolecular force
%% fields, and 
suggest that a complete incorporation of atomic anistropy into next-generation
force fields
will lead to increasingly accurate and
predictive molecular simulations in a wider variety of chemical interactions.
\cite{Hagler2015}

Given the
importance of atomic-level anisotropy in defining intermolecular
interactions,
and the critical role that computationally-affordable standard force
fields play in enabling molecular simulation, our present goal is to
develop a general methodology for standard
force field development that can both universally account for atomic-level anisotropy in all
components of intermolecular interactions \emph{and} that can be routinely
employed in large-scale molecular simulation. 
Furthermore, and in line with our usual goals for force field
development,\cite{Schmidt2015}
our aim is to develop a first-principles-based model that is as accurate and transferable as possible, all while
maintaining a simple, computationally-tractable functional form that allows
for robust parameterization and avoids over-/under-fitting.
Thus, building on prior work (both our own
\cite{VanVleet2016,Schmidt2015,Misquitta2014,Stone2007} 
and from other groups
\cite{Price2000}),
we present here a general ansatz for anisotropic force field development
that, at minimal computational overhead, incorporates atomic-level anisotropy
into all aspects of intermolecular interactions (electrostatics,
exchange, induction, and dispersion) and that accounts for this anisotropy,
not only in the asymptotic limit of large intermolecular separations, 
but also in the region of non-negligible electron density
overlap.
After motivating and establishing the
functional forms used in our anisotropic force fields, we next demonstrate,
using a large library of dimer interactions between organic
molecules, the excellent accuracy and transferability of these new force
fields with respect to the reproduction of high-quality ab initio potential
energy surfaces. Lastly, we showcase how these new force fields can be used
in molecular simulation, and benchmark the
accuracy of our models with regards to a variety of experimental properties. 
The theory and results presented in this Chapter should be
of general utility in improving the accuracy of (particularly ab initio
generated) force fields, such that the complex, inherently anisotropic details of
intermolecular interactions may eventually be routinely incorporated into
increasingly rigorous and predictive molecular simulation.
