As shown elsewhere,\cite{Stone1978,Stone1984}
an exact (under the ansatz of
radial and angular separability) model for \gij is given by Stone's $\bar{S}$-functions, 
which form a complete basis set for describing any scalar function which
depends on the relative orientation
between molecules, and are given (following Stone's
notation\cite{stone2013theory}) by the formula
% General s-function
\begin{align}
%
\bar{S}^{k_1 k_2}_{l_1 l_2 j} = 
i^{l_1 - l_2 -j}
\begin{pmatrix} l_1 & l_2 & j \\ 0 & 0 & 0 \end{pmatrix}^{-1}
\sum \limits_{m_1 m_2 m} 
[D^{l_1}_{m_1 k_1}(\Omega_1)]^*
[D^{l_2}_{m_2 k_2}(\Omega_2)]^*
C_{lm}(\theta,\phi)
\begin{pmatrix} l_1 & l_2 & j \\ m_1 & m_2 & m \end{pmatrix}.
%
\end{align}
The general form of these $\bar{S}$-functions can be quite complicated, and
involve both the Wigner $D$ rotation matrices and Wigner $3j$-symbols (quantities
in parentheses) as well as the degree ($l_1$, $l_2$, and $j$) and order
($m_1$, $m_2$, and $m$ for the global coordinate system, $k_1$ and $k_2$ for the
various local coordinate systems) of the spherical harmonic tensors. Here
subscripts reference either molecule 1 or molecule 2, and subscriptless
quantities refer to the dimer as a whole.

In order to obtain a functional form for the exchange-repulsion that is
amenable to simple combination rules (a necessary prerequisite for
transferable potentials), we must somehow be able to separate \gij into monomer
contributions. Unfortunately, many of the \sfunc depend on the relative
orientation of the dimer itself, and thus must be excluded in the development
of \emph{transferable} potentials.
Thus as a second ansatz (empirically validated by us in \cref{sec:results} and by
others\cite{Millot1992})
we neglect all contributions from \sfunc that depend
on both local coordinate systems. This leaves us with two sets of \sfunc, namely
%
\begin{align}
\bar{S}^{k0}_{l0l} = C_{lk}(\theta_i,\phi_i)
\end{align}
%
and
%
\begin{align}
\bar{S}^{0k}_{0ll} = C_{lk}(\theta_j,\phi_j)
\end{align}
%
which are simply the renormalized spherical harmonics (\cref{eq:sph_harm})
expressed in each of the two local coordinate systems.

Given our truncated expressions for the \sfunc, we now need only extend our
functional form for \fij to incorporate these anisotropic contributions.
We choose, in a manner analogous to literature precedent,
\cite{Stone2007,Mitchell2001,Price2000,Stone1988,Day2003,Torheyden2006,Totton2010,Misquitta2016,Price2010a}
to expand the \Aex{i} and \Aex{j} parameters of \cref{eq:aij} in terms of a truncated
expansion of \sfunc. 
(In principle, we could also account for anisotropy in the \B parameters of
our model for \fij. However, previous literature suggests that in practice this `hardness'
parameter can often be treated as constant, and we also neglect its possible
anisotropy in this Chapter.) 
Consequently, all short-range anisotropies are modeled in this Chapter
by the expressions given in \cref{eq:gij} and \cref{eq:vex}.

%% Finally, and as with exchange-repulsion, molecular anisotropic long-range
%% dispersion coefficients can be determined from an \sfunc expansion of the form
%% %
%% \begin{align}
%% \vdisp = - \sum\limits_{l_A l'_A l_B l'_B}
%%            \sum\limits_{L_A L_B J K_A K_B}
%%             \bar{C}_n(L_A L_B J; K_A K_B)R^{-n} S^{K_A K_B}_{L_A L_B J}
%% \end{align} 
%% %
