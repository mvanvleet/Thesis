Unlike with the exchange energy, the model for electrostatics must account for
both the effects of multipolar interactions (at long-range) and charge
penetration (at short-range):
%
\begin{align}
\velst_{ij} &= -\textcolor{fit}{\Ael{ij}} P(\B, r_{ij}) \exp(-\B r_{ij}) + \vmultipolecolor
\end{align}
%
\Ael{i} parameters are fit in a similar manner to the exchange energy, though
it is not recommended to attempt to re-fit the $B_i$ parameters to the
electrostatic energy. As for the multipole energy, in the simplest case (i.e.
without off-sites) these
parameters can simply be read in using the \verb|settings.py| file,
%
\begin{lstlisting}[language=python]
# Electrostatic Settings: choose which multipole files the program should use
multipoles_suffix          =   '_ISA-GRID_L2.mom'
\end{lstlisting}
%
where the \verb|multipoles_suffix| should point to some file
\verb|<monomer><multipoles_suffix>| in the \verb|input/| subdirectory.
In general, an $L2$ model is a good accuracy benchmark, and it's often useful
to compare the energies obtained from point-charge
models to those achievable with the $L2$ model.

\begin{paragraph}{Off-site models}
As described in \cref{ch:workflow}, practical software limitations and issues
of computational expense may sometimes require us to forego use of
higher-order multipole terms. 
When modeling the electrostatic energy via a point charge model, best accuracy
is often achieved by including off-site charges. Such off-site point charge
models can be treated using \pointer, though 
the following modifications to the standard input scripts are required:
\begin{enumerate}
\item Add the off-site positions to the .sapt file. Scripts for doing this are
discussed in \cref{sec:workflow-dimer_parameters}.
\item Modify the various monomer parameter files (located in the \verb|input/|
subdirectory to reflect the newly-added off-site positions:
    \begin{enumerate}
    \item Add dispersion coefficients (usually all zero) to each
            \verb|<monomer>.disp| file and for each atomtype
    \item Add extra blocks to each \verb|<monomer>.exp| corresponding to the
off-site positions. The .exp file(s) list exponents for each atom in the same
order as the .sapt file, and the .exp and .sapt file orderings must match.
Additionally, the off-site $B_i$ parameters must be set to a non-zero value to
avoid numerical errors.
    \item Add extra drude parameters to each off-site in the
\verb|<monomer>.drude| file. Atom ordering is as in the .sapt file, and
(assuming you do not want the off-sites to be polarizable) the drude charge
parameters should be set to zero.
    \item Assuming you do not wish short-range parameters to be fit to your
off-site atoms, add the names of all off-site atomtypes to \verb|defaults.py|:
\begin{lstlisting}[language=python]
lone_pair_flags                     =  ['Du' , 'lp']
\end{lstlisting}
    \end{enumerate}
\end{enumerate}
%
Energies between the off-site point charge model and the (more standard) $L2$/$L0$ models should
always be compared as a sanity check: if the errors in the offsite model are
larger than the errors from the $L0$ model, this is usually a sign that
something has gone wrong. On the other hand, if the electrostatic energies are
of a similar accuracy to those obtained with the $L2$ model, this indicates
that the point charge model is a fairly optimal description of the system's
multipolar electrostatics, and can be used without further modification in
molecular simulation.
\end{paragraph}

