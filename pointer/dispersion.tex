Dispersion is the last energy component that we must model in order to
completely describe the two-body energy. Asymptotically, the dispersion energy follows a
well-defined expansion in powers of $1/r^{2n}$, and at shorter distances the
energy expression is typically damped by a Tang-Toennies
function\cite{Tang1984,Tang1992} as follows:
%
\begin{align}
\vdisp_{ij} &= -{\Adisp{i}\Adisp{j}} \sum\limits_{n=3}^{6} f_{2n}(x)
\frac{\textcolor{mon}{C_{ij,2n}}}{r_{ij}^{2n}} \\
\Adisp{i}(\theta_i,\phi_i) &=
\textcolor{cfit}{\Adisp{i,\text{iso}}}
\left(1 +
\sum\limits_{l>0,k} \textcolor{fit}{\adisp}  \mathcal{C}_{lk}(\theta_i,\phi_i)
\right)
\end{align}
%
where $f(x)$ is the Tang-Toennies damping function from
\cref{eq:pointer-ff_form}, and the various colors highlight (as in
\cref{fig:pointer-ff}) the different ways in which the dispersion parameters
are calculated/fit.
Dispersion coefficients must always be read into
\pointer as input, and methods for obtaining these coefficients are as
described in  \cref{sec:workflow-dispersion}. 

In obtaining a final model for dispersion, it is important to ensure that any
model is quantitatively correct in the asymptotic regime, as the
least-squares optimization procedure \pointer uses does not ensure this
physically-correct behavior. Unless directly fitting atomtype-specific scale
factors to the dispersion energy (vida infra), a good strategy is to (using
the methods in \cref{sec:workflow-dispersion}) manually fit a universal scale factor to
the dispersion coefficients in order to achieve correct asymptotic behavior.
Once these scaled dispersion coefficients are read into \pointer, additional
anisotropic parameters can then be fit (or set to zero) by appropriate
modification of the \verb|settings.py| file:
\begin{lstlisting}[language=python]
# Dispersion Settings: Choose which parameters to fit to the dispersion
energies. Fit options
# include 'none' (to fit no parameters), 'anisotropic' (to just fit
# anisotropic dispersion parameters, but to leave isotropic dispersion
# coefficients unscaled), and 'all' (to fit both anisotropic and isotropic
# dispersion coefficients)
fit_dispersion             =    'anisotropic'
\end{lstlisting}

In select cases, such as when a \dccsdt correction is added to the
dispersion energy, it can be worthwhile to scale the isotropic
dispersion coefficients in an atomtype-specific manner. (This strategy was
used in \cref{ch:mastiff} to obtain dispersion parameters for \co, \nh, \ho,
and \cl.) This behavior is also allowed in \pointer using the above flags,
however care must be taken to ensure that this optimization does not degrade
accuracy in the asymptotic regime. In general, it is usually best to
\emph{not} fit isotropic scale factors to the dispersion energy in the absence
of significant improvements to the overall force field fit quality.

