\begin{subsection}{Theory}

For \sapt-based force fields with functional forms similar to those in
\cref{ch:isaff,ch:mastiff} -- \mastiff being a prime example\footnotemark{} -- 
we have discussed in \cref{ch:workflow} a number of practical approaches for
beginning intermolecular force field development.
In particular, we have already described strategies for optimally obtaining benchmark
electronic structure theory data and for calculating some of the
monomer-property-based parameters
that will (vida infra) be utilized in the final force field. Nevertheless, 
we have not yet focused on the actual process of force field fitting, nor on
strategies for assessing the accuracy and transferability of the resulting
functional forms and parameters. It is
to these two crucial topics that we now turn.

\footnotetext{
As mentioned above, our focus in this Chapter is primarily on the
\mastiff force field. Nevertheless, most
of the principles and ideas presented below should pertain generally to other
force fields (\saptff, \isaffold, etc.) that are fit on a term-by-term basis
to reproduce a benchmark \eda, and this Chapter should also provide a helpful
set of `best practices' for fitting and analyzing these types of force fields.
}

%%%%%%%%%%%% Fitting Equations %%%%%%%%%%%%%%%%%%%%%%%%%%%%%%%%%
\newcommand{\bij}{\textcolor{black}{B_{ij}}}
\newcommand{\bijr}{\bij r}
%\newcommand{\vmultipolecolor}{\ensuremath{\sum\limits_{\textcolor{mon}{tu}}\textcolor{mon}{Q_t^i}T_{tu}\textcolor{mon}{Q_u^j}}\xspace}

\begin{figure}
\begin{align}
%\intertext{where}
\begin{split}
\label{eq:pointer-ff_form}
\A &= \color{fit} A_iA_j \\
\B &= \sqrt{\textcolor{cfit}{B_iB_j}} \\
\C &= \sqrt{\textcolor{mon}{C_{i,2n}C_{j,2n}}} \\
P(\bij,r_{ij}) &= \frac13 (\bij r_{ij})^2 + \bij r_{ij} + 1 \\
f_{2n}(x) &= 1 - e^{-x} \sum \limits_{k=0}^{2n} \frac{(x)^k}{k!} \\
x &= \bijr_{ij} - \frac{2 \bij^2 r_{ij} + 3 \bij }
{ \bij^2 r_{ij}^2 + 3 \bij r_{ij} + 3} r_{ij} \\[20pt]
%
\vrep_{ij} &= \textcolor{fit}{\Aex{ij}} P(\bij, r_{ij}) \exp(-\bijr_{ij}) \\
\velst_{ij} &= -\textcolor{fit}{\Ael{ij}} P(\bij, r_{ij}) \exp(-\bijr_{ij}) +
\vmultipolecolor
\\
\vind_{ij} &= -\textcolor{fit}{\Aind{ij}} P(\bij, r_{ij}) \exp(-\bijr_{ij}) + \textcolor{mon}{\vdrudeind} \\
\vdhf_{ij} &= -\textcolor{fit}{\Adhf{ij}} P(\bij, r_{ij}) \exp(-\bijr_{ij}) +
\textcolor{mon}{\vdrudescf} \\
\vdisp_{ij} &= -\textcolor{disp}{\Adisp{ij}} \sum\limits_{n=3}^{6} f_{2n}(x)
\frac{\textcolor{mon}{C_{ij,2n}}}{r_{ij}^{2n}} \\
\end{split}
\\[20pt]
\vtot &= \sum\limits_{ij} \vrep_{ij} + \velst_{ij} + \vind_{ij} + \vdhf_{ij} +
\vdisp_{ij}
\end{align}
\captionsetup{singlelinecheck=off}
\caption[Required parameters for \mastiff]
    {An overview of required force field parameters for the \mastiff and/or
\isaffold force fields. All relevant equations are
displayed in black, and the first instance of each parameter is shown in color according to the
following scheme:
\begin{itemize}
\item \textcolor{fit}{Unconstrained} parameters, which must be directly fit by \pointer
%
\item \textcolor{cfit}{Soft-constrained} parameters 
which, depending on user-specified settings, are treated as either soft- or
hard-constraints
%
\item \textcolor{mon}{Hard-Constrained} parameters read in by \pointer which are
always treated as hard constraints
\end{itemize}
See main text for details.
}
\label{fig:pointer-ff}
\end{figure}
%%%%%%%%%%%% Fitting Equations %%%%%%%%%%%%%%%%%%%%%%%%%%%%%%%%%


In regards to the force field fitting process itself, we begin by asking the obvious
question: "What parameters actually need to be fit in order to obtain a final
force field?" To this end, we have highlighted in \cref{fig:pointer-ff} all
of the parameters required to completely specify the \mastiff force field, and
have grouped these parameters according to how these parameters are
calculated/optimized
in practice. Specifically, all force field parameters in \mastiff can be thought
of in terms of one of the following three categories:
%
\begin{itemize}
\item \textcolor{fit}{Unconstrained} parameters: These parameters have not
been specified on the basis of any monomer properties calculation, and so must
be directly fit to the two-body energy itself. 
%
\item \textcolor{cfit}{Soft-Constrained} parameters: These parameters
\emph{can} be fit entirely on the basis of monomer properties, however it is
sometimes advantageous to further refine these parameters with respect to the
benchmark two-body energies. In this case, soft
constraints\cite{Misquitta2016} are often applied to
the fitting process to ensure that the parameters do not deviate strongly from
their original values as calculated from monomer properties.
%
\item \textcolor{mon}{Hard-Constrained} parameters: These parameters are calculated
entirely from monomer properties, and are not futher involved in the force
field fitting process except as hard constraints.
\end{itemize}

To use \mastiff (see \cref{ch:mastiff}) as an example, an overview of the
required parameters, and
the manner in which these parameters are fit, is as follows:
\begin{itemize}
\item \textcolor{fit}{\Aex{i}, \Ael{i}, \Aind{i}, \Adhf{i}}: 
The force field energy depends linearly on a number of short-range prefactors, and
in practice it is fairly straightforward to directly fit each of these
prefactors to the corresponding benchmark \sapt component energy. Note that, for anisotropic atomtypes,
each $A$ coefficient may in fact involve several parameters, all of which must
be directly fit:
\begin{align}
\label{eq:pointer-anisotropy}
\Aex{i}(\theta_i,\phi_i) &=
\textcolor{fit}{\Aex{i,\text{iso}}}\left(1 + 
\sum\limits_{l>0,k} \textcolor{fit}{\aex}  \mathcal{C}_{lk}(\theta_i,\phi_i)
\right)
\end{align}
(Though not entirely standard notation,\cite{stone2013theory}
for clarity in this Chapter we use
$\mathcal{C}$ to denote the set of renormalized spherical harmonics so as to
make a clear distinction between $\mathcal{C}$, the spherical harmonics, and
$C$, the dispersion coefficients from \cref{sec:workflow-dispersion}).
%
\item \textcolor{cfit}{$B_i$}:
The force field energy depends non-linearly on the short-range exponents \B,
making this parameter relatively difficult to optimize without constraints.
Fortunately, the \B parameters can
instead be calculated on the basis of monomer properties (see
\cref{sec:workflow-exponents}), and for obtaining force fields with
\rmse of \kjmol{\textasciitilde 1} it is often sufficient to use the \isa-obtained \B
parameters without further fitting. For obtaining more accurate force fields,
however, and in order to account for small uncertainties in our method of
obtaining \isa-derived \B parameters (see
\cref{sec:workflow-exponent_algorithm}), we have had good success in allowing the
\B parameter to vary slightly from its \isa-derived value. In practice, this entails
optimizing the \B parameters with respect to the benchmark \sapt exchange energy and subject
to a harmonic penalty function.\cite{Misquitta2016}
%
\item \textcolor{cfit}{\Adisp{i}}:
As with the other $A$ pre-factors, a pre-factor can be fit to the benchmark dispersion
energy so as to enhance the force field accuracy with respect to a given
benchmark electronic structure theory. (Vida infra, this benchmark energy can
either by \dftsapt or \ccsdt). Unlike with other pre-factors, however, and because we generally
have good accuracy in obtaining dispersion coefficients $C$ (see
\cref{sec:workflow-dispersion}),
nominally $\Adisp{i} \approx 1$ for most systems. Still, parameters must
sometimes be fit to the dispersion energy due to one or both of the following reasons: 
    \begin{enumerate}
    \item For anisotropic atoms, we must model the orientational dependence of
    the dispersion energy, and this model requires parameters in addition to
    the isotropic dispersion coefficients calculated in
    \cref{sec:workflow-dispersion}).
    \item Uncertainties in the \idma and/or \isa-pol dispersion coefficients can
    sometimes lead to inaccuracies in the isotropic dispersion coefficients,
    and these inaccuracies can sometimes be corrected by rescaling
    the isotropic dispersion coefficients themselves
    \end{enumerate}
In practice, when calculating \Adisp{ij} we often treat the \emph{anisotropic}
dispersion coefficients \adisp as free parameters, and sometimes additionally
optimize an \emph{isotropic} scale factor subject to
soft constraints.\footnotemark{ }
In total, this leads to the following set of parameters and equations for the
dispersion energy pre-factor:
%
\begin{align}
\Adisp{i}(\theta_i,\phi_i) &=
\textcolor{cfit}{\Adisp{i,\text{iso}}}
\left(1 + 
\sum\limits_{l>0,k} \textcolor{fit}{\adisp}  \mathcal{C}_{lk}(\theta_i,\phi_i)
\right)
\end{align}
where the colors serve to indicate that both free and constrained
parameters are contained within the pre-factor.
%
\item \textcolor{mon}{$Q^i_t$}:
Multipole moments $Q^i_t$ can be directly calculated from monomer properties
using the techniques discussed in \cref{sec:workflow-multipoles}. These
\isa-based multipoles are generally quite accurate, however (vida infra) when
using cheaper point charge models some care must be taken to ensure that the
effective model does not lead to a deterioration in force field accuracy.
%TODO: Reference where we discuss electrostatic models
\item \textcolor{mon}{\vdrude}:
As with multipole moments, polarization parameters
(\cref{sec:workflow-polarizabilities}) are treated as hard
constraints during the force field fitting process. Currently, there is not a
consensus on what functional forms and/or damping parameters should be used to 
model the short-range polarization energy, and this topic and its associated
practical issues will be the subject of \cref{sec:pointer-induction}.
\item \textcolor{mon}{$C_{i,2n}$}:
Dispersion coefficients are calculated via the approaches discussed in
\cref{sec:workflow-dispersion}, and are generally treated as hard constraints in the
force field fitting process. In some cases (vida supra), these dispersion
coefficients are scaled to reproduce the \sapt energies, however we discuss in
\cref{sec:pointer-dispersion} some practical concerns involved with such
scaling.
\end{itemize}


\footnotetext{
Currently, two constraints schemes are possible for 
$\Adisp{i,\text{iso}}$. First, we can treat this parameter as a hard
constraint, which sets $\Adisp{i,\text{iso}} = 1$. Second, we can apply
boundary conditions to treat $\Adisp{i,\text{iso}}$ as a free parameter
within the range $ 0.7 \le \Adisp{i,\text{iso}} \le 1.3 $. In future versions
of the \pointer code, we may also include the option of fitting
$\Adisp{i,\text{iso}}$ subject to a harmonic penalty function.
}

\end{subsection}
