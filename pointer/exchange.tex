From \cref{ch:isaff}, and as in \cref{eq:pointer-ff_form},
our exponentially-decaying model for the exchange-repulsion energy 
requires two sets of parameters per atomtype, \Aex{i} and $B_i$: 
%
\newcommand{\bij}{\textcolor{cfit}{B_{ij}}}
\newcommand{\bijr}{\bij r}
\begin{align}
\vrep_{ij} &= \textcolor{fit}{\Aex{ij}} P(\bij, r_{ij}) \exp(-\bijr_{ij})
\end{align}
%
Because the exchange-repulsion energy has no long-range contributions (unlike
with electrostatics, induction, and dispersion), when analyzing the final
force field it is often easiest to use the exchange energy (by itself) to
compare between models with differing numbers of atomtypes or treatments of
anisotropy. Additionally, when fitting the $B_i$ parameters against a harmonic
penalty function, \pointer takes advantage of the relative simplicity of the
exchange-repulsion model and fits this $B_i$ parameter based solely on the
exchange component.

\begin{paragraph}{Exponent Fitting}

To use \pointer to relax the $B_i$ parameters from their initial \isa-based values, the
following flag in \verb|settings.py| can be set to True:

\begin{lstlisting}[language=python]
...
# Exchange Settings: fit_bii selects whether or not to treat the ISA
# short-range exponents are soft- (fit_bii=True) or hard-constraints
# (fit_bii=False)
fit_bii                    =    True
...
\end{lstlisting}
In general, deviations from the input $B_i$ parameters should be no larger
than 5--10\%. Larger deviations may indicate problems with the calculated \bsisa
exponents or with the fitting process itself.

\end{paragraph}
