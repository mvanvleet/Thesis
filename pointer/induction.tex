We turn now to a discussion of the \sapt induction energy, arguably the most
complicated energy component to understand and correctly model. A good
induction model needs to account for all of the following physical phenomenon:
\begin{itemize}
\item Long-range polarization
\item Polarization damping at short-range
\item Charge penetration effects arising from polarization
\item Charge transfer
\end{itemize}
Aside from long-range polarization, for which asymptotically-exact formulas
exist and can be modeled in simulation,\cite{Rick2002,Holt2008,Misquitta2007a,Misquitta2008b} 
there is currently little literature consensus as how best to separate out and
model the various physically-meaningful induction effects. Complicating matters further,
even though the \sapt induction energy is purely a 2-body effect, the
polarization model we develop based on the induction energy also defines
the model for \emph{many}-body polarization, which accounts for a sizeable
fraction of the total many-body energy discussed in \cref{sec:pointer-many_body}.
Consequently, it is easily possible to obtain an induction model which shows good
accuracy for dimer computations, but which leads to substantial inaccuracies
in modeling larger clusters or bulk liquids.

Improved induction models will certainly need to be the subject of future work,
and are discussed further in \cref{ch:future}. In the meantime, below we present a summary of
common induction models that can be used to fit \sapt-based force fields.

\begin{paragraph}{Polarization}
For reasonably isotropic systems, long-range polarization effects can be described
either by a Drude oscillator model or via induced
dipoles.\cite{Rick2002,Lopes2009} In practice, these models tend to be
numerically similar, and we use them interchangeably in parameterization and
simulation.\footnotemark{} \pointer uses a Drude oscillator model for the
purposes of computing the polarization energy, and the Drude charges and
associated spring constants are read in as input in the \verb|<monomer>.drude|
file(s). These charges can be obtained using the methods described in
\cref{sec:workflow-polarizabilities}, however note that these charges are
sensitive to the choice of polarization damping model described below, and may
need to be refit if either the functional form or parameters for the damping
model are changed.

For more anisotropic systems (such as water), higher-order polarizabilities
have been shown to be important, and need to be included for best
accuracy.\cite{Cisneros2016,Misquitta2008b,Misquitta2016,Welch2008}
At present, however, higher-order polarizabilities have not been implemented
in most common software packages, and we are currently exploring the use of
off-site polarizabilities to describe highly anisotropic systems.
\end{paragraph}

\footnotetext{In specific, Drude oscillator models have been used
historically in our group for their simplicity and ease of implementation.
More recently, we have begun running our simulations with the induced dipole
model in order to maintain compatibility with \openmm.}


\begin{paragraph}{Polarization Damping}
At short range, the induced dipole polarizabilities must be damped in order to
avoid the so-called `polarization catastrophe', an effect in which nearby
polarization sites mutually polarize each other to 
infinite values. While there is widespread consensus as to the importance and
necessity of including polarization damping, functional forms and parameters
for the polarization damping vary widely.\cite{Cieplak2009,Lopes2009,Thole1981,Slipchenko2009,Wang2011}
Thole-type damping functions\footnotemark are some of the more commonly used, and some
effort has been put forth to compare between several similar damping functions
and parameterization schemes.\cite{Wang2011,Wang2012,Liu2017}

Historically,\cite{McDaniel2013} several members of our group have used an
exponentially-decaying Thole function with an associated damping parameter of
2.0.\cite{Yu2011} More recently, and due to software limitations in \openmm,
we have taken to using the `Thole-tinker' model with a universal Thole damping
parameter $a=0.33$, which is reasonably similar to the parameter used by the
AMOEBA force field.\cite{VanVleet2017} These models can be changed in \pointer
in the \verb|settings.py| file:
\begin{lstlisting}[language=python]
# Induction Settings: Choose the type and parameters for the polarization
# damping functions. Options for thole_damping_type are 'thole_tinker' and
# 'thole_exponential', and good defaults for thole_param are the 0.33 and 2.0 with
# respect to the two different damping types
# respectively
thole_damping_type         =   'thole_tinker'
thole_param                =    0.33
\end{lstlisting}
%
Note that the choice of Thole damping parameter can be very important, as this
modifies the relative balance between energies ascribed to polarization vs.
charge transfer, in turn modifying the magnitude of the many-body
polarization. In order to achieve a model that achieves the correct balance
between polarization and other inductive effects, future work may need to
involve some of the following advances:
\begin{enumerate}
\item Atomtype-specific Thole damping parameters
\item New functional forms for polarization damping
\item Explicit separation between the \sapt charge-transfer and polarization
energies. Several schemes have already been proposed to achieve this decomposition,
\cite{Bistoni2016,Misquitta2013,Horn2016b,Lao2016}
and the various available schemes should be tested for their utility in force field
development.
\end{enumerate}
\end{paragraph}

\footnotetext{Be advised, most papers in the literature will simply state that a
`Thole damping function' was used, but will not make explicit which of several
different Thole-type damping functions was meant.}

\begin{paragraph}{Charge transfer and inductive charge penetration}

In addition to polarization damping, charge transfer and inductive charge
penetration can become important at shorter intermolecular separations.
Physically-motivated functional forms for these effects are generally lacking,
though \citeauthor{Misquitta2013} has suggested a double exponential decay,
and work in our own group has empirically found a single exponential decay
(with \isa-derived exponents $B_i$) to be reasonably satisfactory.

\end{paragraph}


In summary, our current approach to modeling inductive effects include a sum
over two contributions:
%
\begin{align}
\vind_{ij} &= -\textcolor{fit}{\Aind{ij}} P(\B, r_{ij}) \exp(-\B r_{ij}) + \textcolor{mon}{\vdrudeind} \\
\vdhf_{ij} &= -\textcolor{fit}{\Adhf{ij}} P(\B, r_{ij}) \exp(-\B r_{ij}) + \textcolor{mon}{\vdrudescf}
\end{align}
%
The \sapt benchmark separates the induction energy into 2\super{nd}- and higher-order
(i.e. \dhf) induction, and we fit the terms separately (Note that this expansion is in orders of perturbation theory,
not in orders of the \acrlong{mbe}.)
All parameters for the polarization model \vdrude are currently read in as hard
constraints, and the \Aind{i}\/\Adhf{i} prefactors (which effectively accounts for both
charge transfer and charge penetration) are explicitly fit by \pointer.

