Having identified the required parameters that completely specify \mastiff
and other similar force fields, we now turn to a discussion of the actual
fitting process itself. We begin in this section with an overview of the
software used to optimize each unconstrained/soft-constrained parameter, and
next (in \cref{sec:pointer-fitting}) discussion principles and practices
related to fitting each component of the benchmark \sapt energy.

As the name suggests, the \acrfull{pointer} is a Python package developed to aid in the fitting of
(two-body + $N$-body polarization) intermolecular force fields. \pointer is
open-source and is available for download from
\url{https://git.chem.wisc.edu/schmidt/force_fields}.
%TODO: Update this url and the one below for the pointer code once it's been
%uploaded to github
Documentation and examples for using \pointer are available through the wiki at
\url{https://git.chem.wisc.edu/schmidt/force_fields/wikis/home}, but for
convenience we include here a brief overview of the program input, usage, and
main capabilities:


\begin{subsection}{Input}
Owing to the large number of parameters that serve as hard constraints in
fitting the final \mastiff force field
(see \cref{fig:pointer-ff}), a number of input parameters files are required
in \pointer.
Fortunately, provided the user has already executed the scripts and steps
from
\cref{ch:workflow}, all required input scripts should have all been created
automatically and copied over to the force field fitting subdirectory
(\verb|ff_fitting|) from which the \pointer code is intended to be run.
Thus in practice, \pointer is designed to be run in combination with the
Workflow so as to minimize the amount of required manual input.

In total, the following input files are required by the \pointer program,
where the tag <monomer> indicates that a separate file is required for each
unique monomer being fit. Files highlighted in
\textcolor{codegreen}{teal} or \textcolor{codepurple}{red} indicates that the input file
sometimes or always require
manual modification before running \pointer, whereas files in black are
created automatically from the various scripts used in the Workflow:

\begin{itemize}
\item \textbf{<monomer1>\_<monomer2>.sapt}:
Summarizes the output \sapt energies for each dimer configuration from
\cref{sec:workflow-geometries}, and specifies the atomtype for each atom in
each monomer 
\item \textbf{<monomer>.disp}:
Contains dispersion parameters for each monomer
\item \textbf{<monomer>.drude}:
Contains drude oscillator charges for each monomer
\item \textbf{<monomer>.exp}:
Contains short-range exponents for each monomer
\item \textbf{<monomer>\_<multipole\_suffix>.mom}:
Contains multipole moments for each monomer
\item \textbf{\_\_init\_\_.py}:
Empty file required to keep Python's module structure happy
\item \textcolor{codegreen}{\textbf{<monomer>.constraints}:}
Constraints file, used to include hard-constraints for any \A parameters
for any previously-fit atomtypes. See \cref{sec:pointer-prior_fitting} for
details and \cref{lst:pointer-constraints} for an example input file.
\item \textcolor{codegreen}{\textbf{<monomer>.axes}:}
Axes file, used to specify the local axes and included spherical harmonics for
any anisotropic atomtypes. See \cref{sec:pointer-anisotropy} for details and
\cref{lst:pointer-axes} for an example input file.
\item \textcolor{codegreen}{\textbf{defaults.py}:}
List of default settings for the \pointer program; these defaults rarely need
to be changed for routine force field development. 
See \cref{lst:pointer-defaults} for an example input file.
\item \textcolor{codepurple}{\textbf{settings.py}:}
List of modular settings for the \pointer program; many of these settings can
get changed in the course of routine force field development.  See
\cref{sec:pointer-fitting} for details and \cref{lst:pointer-settings} for
an example input file.
\end{itemize}

\end{subsection}

\begin{subsection}{Usage and Output}

Once the required input files have been created/modified, running the \pointer
program is straightforward:
\begin{lstlisting}
./run_pointer.py
\end{lstlisting}

After a few minutes of runtime, \pointer will generate the following important output
files (file prefixes and suffixes may differ slightly depending on the choice
of input variables \verb|file_prefix| and \verb|file_suffix| from \verb|settings.py|):
\begin{itemize}
\item \textbf{coeffs.out}:
Output file containing fit parameters and error metrics
\item \textbf{exchange.dat}:
\sapt and force field exchange energies given in a two-column format with
ordering identical to the input .sapt file
\item \textbf{electrostatics.dat}:
\sapt and force field electrostatic energies
\item \textbf{multipoles.dat}:
\sapt electrostatic and force field multipolar energies
\item \textbf{induction.dat}:
\sapt and force field second-order induction energies
\item \textbf{dhf.dat}:
\sapt and force field \dhf energies
\item \textbf{edrudes.dat}:
polarization energies, \vdrudeind and \vdrudescf, given in two-column format
\item \textbf{dispersion.dat}:
\sapt and force field dispersion energies
\item \textbf{total\_energy.dat}:
\sapt and force field total energies
\end{itemize}

We now discuss specific details related to the fitting of each energy
component.

\end{subsection}

