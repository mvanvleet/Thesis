\begin{quote}
The challenge for current and future work on force-field development is to
improve the form of the potential energy function, to improve the methodology
used in determining the potential energy parameters, and to use these
advances to generate improved potential energy functions.
\dots That these force fields differ substantially in form and in manner of
derivation serves to emphasize that force field development is still as much a
matter of art as of science. Someday, consensus on the form and manner of
parameterization of molecular force fields may exist, but for now much remains
to be learned.
--- \citeauthor{Halgren1995}
\end{quote}



Thus far, our focus has primarily been on addressing 
the following fundamental (and occasionally practical) issues related to ab-initio intermolecular force field development:
\begin{enumerate}
\item What functional forms should we use to optimally describe two-body
intermolecular interactions? 
(\cref{ch:isaff,ch:mastiff})
\item How can we obtain accurate parameters for the two-body force field?
(\cref{ch:isaff,ch:mastiff,ch:workflow})
\item What benchmark \pes should we use in the fitting process, and how should
it be sampled?
(\cref{ch:lmoeda,ch:workflow})
\end{enumerate}
Building on these theoretical advances, we turn now our discussion to bear on
a much
more practical set of questions: using currently available techniques and softwares,
how can we optimally develop intermolecular force fields for simulating
arbitrary $N$ body systems?
%
This Chapter is designed to address several practical issues related to 
ab initio intermolecular force field development, with a special emphasis on
explaining and documenting the main software (\acrshort{pointer}) that has been
developed expressly for the purposes of optimizing parameters for the various
force field functional forms described in
\cref{ch:isaff,ch:mastiff}. In particular, we attempt in this Chapter to
address the following questions:
\begin{enumerate}
\item What algorithms and procedures are involved in fitting \sapt-based force
fields?
\item What functional forms should be used to describe the two-body force
field for a particular system or application?
\item How can we properly include a treatment of many-body effects 
so as to yield a final model for describing arbitrary $N$-body
systems?
\item Once we've settled on a functional form and set of parameters for the
$N$-body force field, how can we evaluate the
quality of this force field and its utility for molecular simulation? 
\end{enumerate}



=========================================



