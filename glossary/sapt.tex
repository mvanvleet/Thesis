%%% The glossary entry the acronym links to   
\newglossaryentry{saptg}{name={SAPT},
    description={Symmetry-Adapted Perturbation Theory, a perturbative
treatment of intermolecular interactions which is pretty cool}
    }

%%% define the acronym and use the see= option
\newglossaryentry{sapt}{type=\acronymtype, name={SAPT},
description={Symmetry-Adapted Perturbation Theory},
text={SAPT},
short={SAPT},
long={Symmetry-Adapted Perturbation Theory},
first={Symmetry-Adapted Perturbation Theory (SAPT)\glsadd{saptg}}, 
see=[Glossary:]{saptg},
    }


\newglossaryentry{ccsdtf}{name={CCSD(T)-f12},
    description={Explicitly-correlated CCSD(T). Given a sufficiently large
(\avdz or \avtz) basis set, used throughout this work as a `gold-standard'
estimate of the exact \acrlong{pes}}
    }

\newglossaryentry{ccsdt}{name={CCSD(T)},
    description={Coupled Cluster methods including singles, doubles, and
perturbative triples excitations. CCSD(T). Given a sufficiently large
(\avqz or better) basis set, can be used as a `gold-standard'
estimate of the exact \acrlong{pes}}
    }


\newglossaryentry{lmoeda}{name={LMO-EDA},
    description={FILL  }
    }


\newacronym{dft}{DFT}{Density Functional Theory}
\newacronym{dftsapt}{DFT-SAPT}{Density Functional Theory Symmetry-Adapted
Perturbation Theory}

\newacronym{dma}{DMA}{Distributed Multipole Analysis}
