\begin{chapter}{Background}
\label{ch:background}



\begin{section}{Molecular Mechanics and the Theory of Intermolecular Forces}

What is a force field?
What are the important components of a force field, and how do we model them?
%
\begin{subsection}{The Many-Body Expansion}
How do we break apart a force field into manageable pieces? Why does it make
sense to break a force field into 2- and many-body components?
\end{subsection}
\begin{subsection}{Energy Decomposition Schemes}
\begin{subsubsection}{Intramolecular Interactions}
Brief commentary on the non-intermolecular portions of a force field
\end{subsubsection}
\begin{subsubsection}{Electrostatics}
Conceptual description of electrostatics: long-range multipoles and charge
penetration
\end{subsubsection}
\begin{subsubsection}{Exchange}
Quantum-mechanically-based Pauli Exclusion.
Theoretical grounds for exponential behavior
\end{subsubsection}
\begin{subsubsection}{Induction}
Charge transfer.
Polarization.
Polarization Damping.
\end{subsubsection}
\begin{subsubsection}{Dispersion}
Theoretical Formulation.
Damping.
\end{subsubsection}
\end{subsection}
%
\end{section}


\begin{section}{Ab-Initio Force Field Development}
\begin{subsection}{Electronic Structure Benchmarks}
\begin{subsubsection}{SAPT}
General SAPT methodology.
DFT-SAPT.
\end{subsubsection}
\begin{subsubsection}{Coupled-Cluster Methods}
CCSD(T).
CCSD(T)-f12.
\end{subsubsection}
\end{subsection}
\end{section}


\begin{section}{ISA-based methods for force field development}
What is ISA? How can ISA be used to generate parameters for intermolecular
force field development? What progress has been made from this approach?
\end{section}



\end{chapter}
