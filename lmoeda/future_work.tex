\begin{section}{Future Work}
\label{sec:lmoeda-future_work}

Throughout this Chapter, we have attempted to highlight some of the key
limitations of our force field development methodology for \cus-\mofs. 
In summary, the following issues would need to be resolved in order to expand the
scope and utility of the present research:
\begin{enumerate}
%
\item \textbf{Memory Limitations with \lmoeda in GAMESS}: 
%
As evidenced in this work,
relatively large (60+ atom) cluster models are required to correctly
parameterize force fields for the M-MOF-74 system. While these cluster sizes
do not present difficulties for standard \dft calculations with reasonable
basis sets, the corresponding
\lmoeda calculations were, as implemented in the GAMESS software package,
infeasible due to memory requirements. Some time was spent attempting to
address these memory issues, particularly for the memory-intensive
Edmiston-Ruedenberg localization subroutine that is the source of the problem.
However, due to our lack of familiarity with the GAMESS software and the
\lmoeda source code, this pursuit was eventually dropped. 
%
\item \textbf{Fundamental Issues with \lmoeda}: 
%
As discussed in \cref{sec:lmoeda-theory}, the \lmoeda method has several
theoretical limitations. In particular, and especially for functionals with no
defined separation between exchange and correlation functionals, \lmoeda does
not offer a clean separation between the exchange and dispersion energies.
Furthermore, and unlike some recent \eda
methods,\cite{Misquitta2013,Horn2016b} \lmoeda cannot separate induction
into charge transfer and polarization components. 
%
\item \textbf{Transferability of the Force Field Functional Form}: 
%
While our
final force field for 
studying \co interactions in \mgmof is highly accurate (both with respect to
ab initio theory and with respect to experiment), it does not appear that this 
accuracy extends to models for the adsorption of other small molecules, such as
\ch{N2}. This transferability limitation is almost certainly due to the chosen
double-exponential functional form and/or the parameterization process used to
obtain Mg parameters, and improvements to this methodology will be essential
to make our work on the \co--\mgmof system applicable to general force field
development for \cus-\mofs. In particular, future work will require a better force field for describing
short-range interactions, as the functional forms and paramters used in this
work struggled to both accurately and transferably model the \mgmof exchange energies.
%
\end{enumerate}

While several of these issues (particularly practical limitations with the
\lmoeda implementation) have yet to be addressed in a meaningful fashion,
several recent theoretical advances may pave the way for continued work on this
project. Thus for \cus-\mofs and other systems where \dftsapt might be in
error, we offer the following recommendations:
%
\begin{enumerate}
%
\item \textbf{Improved \sapt energies}: 
%
Recently, it has been proposed that the commonly used single-exchange (`$S^2$')
approximation can lead to errors in the description of the induction
energy, particularly for ionic systems.\cite{Jansen2012,Lao2015a} 
While it is difficult to attribute errors in \sapt to a particular energy
component, it may well be that \sapt poorly describes \mgmof due to the $S^2$
treatment of the induction energy. In this case, eliminating the $S^2$
approximation might improve the \dftsapt total interaction energies, thus
enabling \sapt to be used for (at least) closed-shell \cus-\mofs.
%
\item \textbf{New \sapt Correction Schemes}: 
%
As discussed in \cref{ch:mastiff}, deviations between \sapt and \ccsdt can be
rectified by adding a \dccsdt correction to the total \sapt energy.
As discussed in \cref{ch:mastiff}, we have empirically had good success modeling this \dccsdt correction as part of
the dispersion energy. Though this partitioning choice may require adjustment
for treating \mgmof, the results in
\cref{ch:mastiff}
indicate that simply correcting (rather than entirely ignoring) the \dftsapt energies
is a promising strategy for transferable force field development.
%
\item \textbf{New \eda schemes}: 
%
Since this work was completed, a second-generation ALMO-EDA scheme has been
implemented in the Q-Chem software package.\cite{Horn2016b} Crucially, and unlike its
predecessor, this ALMO-EDA scheme now breaks up the interaction energy into
electrostatic, exchange, polarization, charge-transfer, and dispersion
components. While there is no guarantee that such an \eda could serve as the
basis for \cus-\mof force field development (see \cref{sec:lmoeda-theory}),
these and other recently developed \edas may be worth investigation, and could
eventually replace the (practically problematic) \lmoeda method.
%
\item \textbf{Improved Force Field Functional Forms}: 
%
Since 2014, we have made significant progress in developing more accurate and
transferable intermolecular force fields (see \cref{ch:isaff,ch:mastiff}), and
many of these advances particularly improve the description of the short-range
potential itself. There is a good chance that either the \isaffold or \mastiff
methodologies would yield high-quality force fields for \mgmof and other
systems. In this case, continued work on this project might be an exciting
avenue for showcasing the \mastiff methodology in the context of
accurate inorganic/organometallic force field development.
%
\end{enumerate}



\end{section}
