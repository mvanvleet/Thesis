\begin{section}{Computational Methods}
\label{sec:lmoeda-methods}

\begin{subsection}{Partial Charge Determination}

Partial charges for \mgmof were determined in a manner analagous to
\citen{McDaniel2015} using the $Q_{\text{SBU}}$ method. Two cluster models, one
a hydrogen-capped DOBDC ligand environment, and one a capped \ch{MgO5} inorganic chain,
were constructed and analysed using a \acrfull{dma}. The resulting \dma charges were
then used to obtain charge paramters for the ligand and inorganic SBU,
respectively. See \cref{sec:lmoeda-params} for final charge parameters.

\end{subsection}
\begin{subsection}{Force Field Fitting}

Two types of force field functional forms were considered in this work. The
first, a `single-exponential' functional form, exactly matches that used in
\citen{McDaniel2013}, with the exception that \dhf parameters were not fit to
the Mg atomtype. This fitting choice was due to the fact that \lmoeda only provides a
total \induction term (rather than splitting into 2\textsuperscript{nd}- and
higher-order \induction energies, as with \sapt). 

For the `double-exponential' functional form used to fit the \mgmof-Yu cluster
model, the same functional form was used as in the single-exponential case,
with the exception that two sets of short-range interaction parameters
(labeled Mg and Du in \cref{sec:lmoeda-params}) were assigned to the Mg atomic center. This
effectively meant that Mg was described by two separate exponential decays,
thus enabling additional parameterization flexibility for the force fields discussed in
\cref{sec:lmoeda-mgmof}.

In all cases, force fields were fit using the Fortran code described in the
Appendix of \citen{McDaniel2014a}.


\end{subsection}




\end{section}
