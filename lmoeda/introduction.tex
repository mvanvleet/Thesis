\begin{section}{Preface}
The preceding sections have been devoted to a development of various methodologies for ab initio intermolecular force
field development, all generally assuming that \sapt can be used as a benchmark electronic structure
theory. Critically, and especially given the developments discussed in
\cref{ch:mastiff},
we can now usually expect our model force field energies to be within
\kjmol{\textasciitilde1} of the \sapt reference values! In spite of this
success, this high precision between the model
and \sapt energies can only lead to experimentally-accurate molecular
simulation in the event that the \sapt energies themselves are
accurate with respect to the exact underlying \pes. 
Indeed, for systems where \sapt
and \ccsdt (a gold-standard electronic structure theory that closely matches
the exact \pes) disagree by several
\kjmol{}, there is little advantage in developing
\sapt-based force fields with sub-\kjmol{} precision. This limitation raises to two
fundamentally important questions. 
First, for what types of systems might we expect \sapt to be inaccurate? Second,
for the systems where \sapt and the exact \pes are in disagreement, how
might we best modify our typical methodology for ab-initio force field development?

The purpose of this Chapter is to partially address these two questions, all
within the specific context of force field development for \cus \mofs.
Vida infra, \sapt energy calculations have been shown to be in error for select \cus-\mofs,
and we investigate how our \sapt-based force field development methodology
can be extended to allow for force field development based on
generic \edas,
\cite{Pastorczak2017,Horn2016b,Su2014,Su2009,Fedorov2006,CruzHernandez2006,Jeziorski1994}
with a particular emphasis on the Localized Molecular Orbital EDA (\lmoeda).
\cite{Su2009,Chen2010}
The work in this Chapter represents results gathered from 2012 -- 2014, after
which the project was discontinued,  largely due to memory limitations in the
\lmoeda implementation in GAMESS. 
Additionally, some
important advances (namely those presented in \cref{ch:isaff,ch:mastiff}) were
unavailable during this project, leading to additional fundamental challenges
in the accuracy and transferability of the resulting force fields.
Should this project be continued in the
future, it will likely prove necessary to refit \eda-based force
fields to the functional forms and monomer-based parameters
discussed in \cref{ch:mastiff}.
In spite of these challenges and limitations, the results presented in this
Chapter provide a valuable
demonstration for the utility of using alternate \eda schemes as the basis
of ab initio force field development.


\end{section}
\begin{section}{Introduction}
\glsreset{mof}

\Glspl{mof} are an increasingly important class of compounds, and are
defined as porous
materials containing inorganic nodes connected by organic linkers. Within
this general motif, more than 20,000 compounds have been reported and
studied,\cite{Furukawa2013} and this vast diversity of \mof materials shows great promise
for chemical customization and optimization. Over the past two decades, a
huge body of research has been devoted to the design and study of \mofs, and
current applications range from gas separation and storage to catalysis and
biomedical imaging.\cite{Furukawa2013}

Somewhat recently, it has been discovered that so-called \cus \mofs can be
created by activation of solvent-coordinated inorganic nodes to yield exposed
(or 'open') metal sites.\cite{Millward2005b,Dietzel2009,Dzubak2012} These \cus-\mofs have
been shown to exhibit exhibit excellent uptakes and selectivities in a number
of gas separation and storage problems,\cite{Czaja2009,Millward2005b,Dietzel2009}
making this family of compounds an excellent target for future
investigation and materials design. Owing to the vast scope of hypothetical
\cus-\mof materials, however, and the number of chemically-distinct targets
for gas separation/storage, it is unlikely that experiment alone can
be used to screen for new and promising \cus-\mof materials.\cite{Krishna2011} 
Rather, a combination of
experiment and computational modeling will be required to identify (or possibly
even rationally design) optimal \cus-\mofs.\cite{Getman2012,Czaja2009,Krishna2011}

%Todo: Mention other force fields that succeed at describing cusmofs.
Despite the utility of computational studies, it remains challenging to
develop molecular models for
\cus-\mofs.\cite{Dzubak2012,Krishna2011,Getman2012} Because the strong binding
between metal and adsorbate leads to chemical environments substantially
different from typical coordinatively-saturated \mofs, many standard force
fields (such as UFF and DREIDING) that yield good predictions for these
CS-\mofs can frequently (and substantially!) underpredict adsorption in
\cus-\mofs.\cite{Yazaydin2009,Krishna2011,Getman2012} Importantly, these
underpredictions are especially prominent at low pressures, where
metal-adsorbate interactions
dominate.\cite{Yazaydin2009,Krishna2011,Getman2012} While \cus-\mofs can
sometimes be studied using quantum mechanical
means,\cite{Getman2012,Valenzano2010,Poloni2014} clearly new and improved
force fields will be required to perform in-depth simulations and large-scale
screenings of these materials, and such studies are already being
undertaken.\cite{Lin2014,Haldoupis2015,Mercado2016,Becker2017}

The goal of the present chapter is two-fold: first, to present a general 
methodology for developing accurate and transferable force fields for
\cus-\mofs, and second, to showcase how generic \edas can be used as the basis
for force field development.
The current study is limited to a discussion of the MOF-74 series (a
prototypical and well-studied \cus-\mof) and \lmoeda,\cite{Su2009,Chen2010} however it
is expected that the methods presented herein might also be applicable to
other systems and \edas. After outlining the methodologies used in this
Chapter
(\cref{sec:lmoeda-background,sec:lmoeda-theory}), we next show how our force
fields can be applied to accurately predict \co adsorption isotherms in
\mgmof. At the present time, we do not have results for other compounds in the
M-MOF-74 series (M = Co, Cr, Cu, Fe, Mn, Ni, Ti, V, and Zn), largely as a
result of techical challenges in the force field parameterization process.
We discuss these technical limitations in some detail, and conclude with our
perspective on the challenges and opportunities associated with developing
transferable force fields for the M-MOF-74 series and other similar \cus-\mof systems.



\end{section}


%% =======================
%% 
%% \mgmof has good \co capacity: \cite{Krishna2011}
%% Choice of cluster significantly impacts binding energies: \cite{Getman2012}
%% \co-\mof force field for Cu, Co, Mn, Ni-MOF-74: \cite{Haldoupis2015}
%%     Isotherms are okay (better than UFF, certainly, but still overpredicts
%%     adsorption at high loadings), and would only be applicable to \co adsorption.
%% Another \co-\mof force field for M = Co, Cr, Cu, Fe, Mg, Mn, Ni, Ti, V, and
%% Zn: \cite{Becker2017} 
%%     - Uses UFF (framework) and Trappe (CO2) as starting points, but also
%%     includes polarization. Pretty good agreement between adsorption isotherms. 
%%     - Fit Mg-MOF-74/\co potential to experiment, but then used the same
%%     scaling parameters to compute other M-MOF-74 and CH4 adsorption. Some MOFs and
%%     the CH4 isotherms are not reproduced well, but technically their model isn't
%%     site-specific.
%% "Approaching-paths" non-polarizable \co adsorption model: \cite{Lin2014}
%%     Reproduces Mg and Zn energies, and can be parameterized for both \co and
%%     h2o. Non-polarizable, so the parameters probably don't transfer well to new Mg
%%     environments.
%% 
%% \gls{sapt}
%% \gls{saptg}
%% \gls{abintff}


