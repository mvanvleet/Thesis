\begin{section}{Preface}
The preceding sections have been devoted to a development of various methodologies for ab initio intermolecular force
field development, all generally assuming that \sapt can be used as a benchmark electronic structure
theory. Critically, and especially given the developments discussed in
\cref{ch:mastiff},
we can now usually expect our model force field energies to be within
\kjmol{\textasciitilde1} of the \sapt reference values! In spite of this
success, this high precision between the model
and \sapt energies can only lead to experimentally-accurate molecular simulation provided that the \sapt energies themselves are
accurate, either with respect to the exact underlying \pes or (in
practice)
with respect to gold-standard \ccsdt calculations. Indeed, for systems where \sapt
and \ccsdt disagree by several \kjmol{}, there is little point in developing
\sapt-based force fields with sub-\kjmol{} accuracy! This limitation raises to two
fundamentally important questions. 
First, for what types of systems might we expect \sapt to be inaccurate? Second,
for the systems where \sapt and the exact \pes are in disagreement, how
must we modify our typical methodology for ab-initio force field development?

The purpose of this chapter is to partially address these questions, all
within the specific context of 
force field development for \glspl{mof}. Note that the results presented here were
gathered from 2013--2015, so some important advances (namely those
presented in \cref{ch:isaff,ch:mastiff}) haven't been incorporated into
the force fields presented here. This is probably to the determinent of the accuracy and
transferability that might be possible with the \lmoeda-based methodology, and
(should this project be picked up in the future) it may be necessary to
refit these force fields to the functional forms and monomer-based parameters
discussed in \cref{ch:mastiff}.


\end{section}
\begin{section}{Introduction}
\glsreset{mof}

\Glspl{mof} are an increasingly important class of compounds, 
fundamentally defined as porous
materials comprised of inorganic nodes connected by organic linkers. Within
this general motif, more than 20,000 compounds have been reported and
studied,\cite{Furukawa2013} and this vast diversity of \mof materials shows great promise
for chemical customization and optimization. Within the past two decades, a
huge body of research has been devoted to the design and study of \mofs, and
current applications range from gas separation and storage to catalysis and
biomedical imaging.\cite{Furukawa2013}

Somewhat recently, it has been discovered that so-called \cus \mofs can be
created by activation of solvent-coordinated inorganic nodes to yield exposed
(or 'open') metal sites.\cite{Millward2005b,Dietzel2009,Dzubak2012} These \cus-\mofs have
been shown to exhibit exhibit excellent uptakes and selectivities in a number
of gas separation and storage problems,\cite{Czaja2009,Millward2005b,Dietzel2009}
making this family of compounds an excellent target for future
investigation and materials design. Owing to the vast scope of hypothetical
\cus-\mof materials, however, and the number of chemically-distinct targets
for gas separation/storage, it is unlikely that experiment alone can
be used to screen for new and promising \cus-\mof materials.\cite{Krishna2011} 
Rather, a combination of
experiment and computational modeling will be required to find (or possibly
even rationally design) optimal \cus-\mofs.\cite{Getman2012,Czaja2009,Krishna2011}

Despite the utility of computational studies, it remains challenging to
develop molecular models for \cus-\mofs.\cite{Dzubak2012} Because the strong
binding between metal and adsorbate leads to chemical environments
substantially different from typical coordinatively-saturated \mofs, many
standard force fields (such as UFF and DREIDING) which yield good predictions
for these \mofs frequently (and substantially!) underpredict adsorption in
\cus-\mofs, especially at low
pressures.\cite{Yazaydin2009,Krishna2011,Getman2012} While \cus-\mofs can
sometimes be studied using quntum mechanical
means,\cite{Getman2012,Valenzano2010} clearly
new and improved force fields will be required to perform in-depth simulations
and large-scale screenings of these materials.

The goal of the present chapter is to present a general 
methodology for developing accurate and transferable force fields for \cus-\mofs. 
The current study is limited to a discussion of the MOF-74 series (a
prototypical and well-studied \cus-\mof), however it
is expected that the methods presented herein might also be applicable to
other systems. After outlining this methodology
(\cref{sec:lmoeda-background,sec:lmoeda-theory}), we next show how our force
fields can be applied to accurately predict \co adsorption isotherms in
\mgmof. At the present time, we do not have results for other compounds in the
M-MOF-74 series (M = Co, Cr, Cu, Fe, Mn, Ni, Ti, V, and Zn), largely as a
result of techical challenges in the force field parameterization itself.
We discuss these technical challenges in some detail, and conclude with our
perspective on the challenges and opportunities associated with developing
transferable force fields for these and other \cus-\mof systems.



\end{section}


%% =======================
%% 
%% \mgmof has good \co capacity: \cite{Krishna2011}
%% Choice of cluster significantly impacts binding energies: \cite{Getman2012}
%% \co-\mof force field for Cu, Co, Mn, Ni-MOF-74: \cite{Haldoupis2015}
%%     Isotherms are okay (better than UFF, certainly, but still overpredicts
%%     adsorption at high loadings), and would only be applicable to \co adsorption.
%% Another \co-\mof force field for M = Co, Cr, Cu, Fe, Mg, Mn, Ni, Ti, V, and
%% Zn: \cite{Becker2017} 
%%     - Uses UFF (framework) and Trappe (CO2) as starting points, but also
%%     includes polarization. Pretty good agreement between adsorption isotherms. 
%%     - Fit Mg-MOF-74/\co potential to experiment, but then used the same
%%     scaling parameters to compute other M-MOF-74 and CH4 adsorption. Some MOFs and
%%     the CH4 isotherms are not reproduced well, but technically their model isn't
%%     site-specific.
%% "Approaching-paths" non-polarizable \co adsorption model: \cite{Lin2014}
%%     Reproduces Mg and Zn energies, and can be parameterized for both \co and
%%     h2o. Non-polarizable, so the parameters probably don't transfer well to new Mg
%%     environments.
%% 
%% \gls{sapt}
%% \gls{saptg}
%% \gls{abintff}


