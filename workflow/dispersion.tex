\begin{subsubsection}{Theory}

Dispersion coefficients can also be determined from distributed molecular
(that is, \aim)
property calculations using either an \isa- or \dma-based approach. The method
for obtaining distributed dispersion coefficients has been described in detail elsewhere for an assortment of \dma-based
approaches,
\cite{Williams2003,Misquitta2008,McDaniel2012,McDaniel2013,stone2013theory,McDaniel2014a}
and \citen{McDaniel2014a} in particular provides a useful summary of the
different equations and molecular properties that are needed to derive
the types of dispersion models used in \cref{ch:isaff,ch:mastiff}.
In brief, \aim dispersion energies can be obtained by integrating over
distributed 
frequency-dependent polarizabilities for each monomer.
%% \begin{align}
%% E^{ab}_\text{disp} = - \frac{\hbar}{2\pi} T^{ab}_{tu} T^{ab}_{t'u'}
%% \int\limits_{0}^{\infty} 
%% \alpha^a_{tt'} (i\omega) \alpha^b_{uu'} (i\omega) d\omega
%% \end{align}
%% where here 
%% $\alpha$ is the frequency-dependent polarizability for atom $a$ or $b$, $T$
%% is an interaction function (see Appendix F in \citen{stone2013theory} for
%% details), and
%% the subscripts $t,t',u,u'$ represent the rank and order (e.g. t = 11c, to use
%% Stone's notation\cite{stone2013theory}) of the associated spherical harmonic
%% component. 
Under the simplifying assumption that we can treat these polarizabilities as
isotropic,
the dispersion energy expression is given by
\begin{align}
\label{eq:workflow-edisp}
E^{ab}_{\text{disp}} &\approx - \frac{C^{ab}_6}{r_{ab}^6} - \frac{C^{ab}_8}{r_{ab}^8} - \ldots
\end{align}
for each atom pair, where 
\begin{align}
\label{eq:workflow-c6}
C^{ab}_6 &= \frac{3}{\pi} \int\limits_{0}^{\infty} \bar{\alpha}^a_{1}
(i\omega) \bar{\alpha}^b_{1} (i\omega) d\omega , \\
%
C^{ab}_8 &= \frac{15}{2\pi} \int\limits_{0}^{\infty} 
\bar{\alpha}^a_{1} (i\omega) \bar{\alpha}^b_{2} (i\omega) 
+ \bar{\alpha}^a_{2} (i\omega) \bar{\alpha}^b_{1} (i\omega) 
d\omega , 
\label{eq:workflow-c8}
\end{align}
and the higher order terms are defined analagously. $C^{ab}_n$ are the
atom-atom dispersion coefficients, and $\bar{\alpha}^a_l$ are the
rank $l$, isotropic, \aim frequency-dependent polarizabilities. 
The formalisms in \cref{eq:workflow-c6,eq:workflow-c8} can be somewhat
involved,
but for our purposes the important take-away is the understanding that the dispersion
coefficients can be entirely determined by calculating the frequency-dependent
polarizabilities for each atom in its molecular environment.

Although it is straightforward to calculate \emph{molecular} frequency-dependent
polarizabilities, a central difficulty in obtaining transferable dispersion
coefficients is that, 
in order to evaluate \cref{eq:workflow-c6,eq:workflow-c8},
we must have some physically-meaningful method for
calculating \emph{\acrlong{aim}} polarizabilities.
Many
distribution strategies exist in the literature, and here we focus on two such
techniques. First, and as we have used in \cref{ch:isaff,ch:mastiff}, one can
utilize a
\dma-based approach to partition the polarizabilities into \aim
contributions. In this case, and due to deficiencies in the \dma partitioning
scheme,
the resulting atomic polarizabilities are not always 
positive-definite and monotonically-decaying, and this unphysical behavior can lead to a breakdown in
transferable parameterization.\cite{Williams2003}
To correct for this undesirable behavior, \citeauthor{McDaniel2013} have
proposed a constrained
fitting procedure whereby atomic polarizabilities can be optimized in an
iterative fashion, thereby generating transferable atomic polarizabilities at
the expense of requiring a fairly large training set for each unparameterized atomtype
(see \secref{sec:workflow-jesse} for details).

As an alternative to the above iterative polarization partitioning scheme, recently
Misquitta has developed an \isa-based partitioning scheme to extract the
atomic frequency-dependent polarizabilities. While this approach requires
further testing, and is not yet published,
% in a manner analagous to our discussion of multipole moments (see \cref{ch:background} and
% \citen{Misquitta2014}) 
the resulting `\isa-pol' method appears to lead to a more
physically-meaningful partitioning of the molecular polarizabilities. For
practical purposes, this more physical partitioning 
enables us to determine transferable dispersion coefficients without
resorting to large training sets. Formalisms and technical details
related to \isa-pol are the subject of \cref{sec:workflow-alston}, and a
comparison between the two methods for obtaining dispersion coefficients is
given in \cref{sec:workflow-dispersion_comparison}. Finally, each method for
obtaining dispersion coefficients requires a small amount of post-processing,
and this is discussed in \cref{sec:workflow-cn_postprocess}.

\end{subsubsection}

\begin{subsubsection}{Iterative-\dma-pol}
\label{sec:workflow-jesse}

As described in \citen{McDaniel2014a}, the iterative-\dma-pol (\idma) method of
\citeauthor{McDaniel2013} performs a constrained optimization of
atomtype-specific frequency dependent polarizabilities by fitting all
polarizabilities to reproduce the so-called `point-to-point response',
$\alpha_{PQ}$. This point-to-point response is a
molecular quantity that
describes the change in electrostatic potential at point P due to an induced
change in the electron density of a molecule caused by a point charge
perturbation $q_Q$ at point Q. For an isotropic polarizability model,
%
\begin{align}
\label{eq:workflow-apq}
\alpha_{PQ} = -q_Q \sum\limits_{a,lm} T^{Pa}_{0,lm} \bar{\alpha}^a_l T^{aQ}_{lm,0}
\end{align}
%
where the $T$ are the spherical harmonic interaction functions described above and in
\citen{stone2013theory}. Aside from the isotropic polarizabilities
$\bar{\alpha}^a_l$, all quantities in \cref{eq:workflow-apq} are directly
calculated in \camcasp, enabling us to fit
the isotropic polarizabilities based on a \camcasp properties calculation (see
Appendix A of \citen{McDaniel2014a} for details).

Using the \idma method in the Workflow has two software dependencies: 
\begin{enumerate}
\item The \idma fiting program itself, which can be downloaded at
\url{https://github.com/mvanvleet/p2p-fitting}. Three executables
(\verb|main_dispersion|, \verb|main_drude|, and \verb|localize.sh|) need to be added to your bash
\verb|$PATH| for the scripts listed in this section to work properly.
%
\item \camcasp, which can be downloaded from 
\url{http://www-stone.ch.cam.ac.uk/programs/camcasp.html}. \camcasp also
requires several environment variables to be added to your bash \verb|$PATH|, and some of these
environment variables are also used by the \idma fitting program.
%
\end{enumerate}
and requires two additional input files:
\begin{enumerate}
\item \verb|input/<monomer>.atomtypes|: The \idma fitting program performs
a constrained optimization whereby the $\bar{\alpha}^a_l$ are set to be
identical for atoms with the same atomtype. Consequently, the
\verb|<monomer>.atomtypes| input file is required to specify the atomtypes in
each monomer. This .atomtypes file has the same
format as an .xyz file, with the exception that the element names for each atom
are replaced with a user-defined atomtype. See
\cref{lst:workflow-pyridine.atomtypes} for an example with pyridine.
%
\item \verb|templates/dispersion_base_constraints.index|: As described below,
with \idma it is usually advisable to only fit one or two atomtype polarizabilities at a
time, with the remaining atomtype polarizabilities read in as hard
constraints. The \verb|dispersion_base_constraints.index| file lists these
hard constraints in a block format, 

\begin{minipage}{\linewidth}
%[float,floatplacement=H]
\begin{lstlisting}
CT
1 
 7.14483224 7.11095841 6.87452508 6.19718464 4.87589777 
 3.17818610 1.56461102 0.51670933 0.09175313 0.00367230 
2 
 20.26394042 20.00584110 17.66562710 14.33668329 12.03179893 
 11.49156262 7.86254302 3.10936998 0.53746459 0.01774391 
3 
 77.37303638 73.13014787 24.68682297 -13.48390193 0.40172836 
 29.76747226 34.31668916 17.88515654 3.13260459 0.10137127

\end{lstlisting}
\end{minipage}
which lists each constrained atomtype along with 10 frequency-dependent
polarizabilities for each polarizability rank (1-3). (\camcasp uses numerical
integration to solve \cref{eq:workflow-c6}, and the 10 polarizabilities per
rank correspond to the frequencies \camcasp needs to perform the numerical
quadrature. See the
\href{http://www-stone.ch.cam.ac.uk/programs/camcasp.html}{\camcasp user
manual} for details.) Each polarizability block should be separated by a blank
line, and the atomtypes listed in the .index file \emph{must} match those in
the .atomtypes file for any hard constraints to be applied. 
Previously-fit atomtype polarizabilities from \citen{McDaniel2013} are already included in 
\verb|dispersion_base_constraints.index| so as to minimize the number of hard
constraints that the user will need to add manually, and these hard
constraints should be used whenever possible.
\end{enumerate}

Once all required input files have been created, and assuming the IP calculations
from \cref{sec:workflow-sapt} have already been performed,
the \camcasp calculations necessary to run the \idma program
can be performed by executing the command 
%
\begin{lstlisting}
./scripts/make_dmapol_files.py
\end{lstlisting}
%
and running the resulting input files through the \camcasp software (a process
which can take several hours).
Once the \camcasp calculations finish, dispersion coefficients can be obtained by running the following
work-up script:
%
\begin{lstlisting}
./scripts/workup_dispersion_files.sh
\end{lstlisting}
%
The resulting dispersion coefficients will be listed in the
\verb|dispersion/<monomer>.cncoeffs| output file.

When generating dispersion coefficients using \idma, the following sanity-checks should
always be performed:
\begin{enumerate}
\item The \verb|<monomer>_fit_dispersion.out| file lists the number and names of
unconstrained atomtypes. Ensure that the number and type of unconstrained atomtypes match your
expectations, and that the number of fit atomtypes is kept relatively small
(1-2 max). If you need to fit multiple atomtypes simultaneously, or you obtain
unphysical disperion coefficients (see next point), you'll likely need to
utilize the iterative fitting algorithm outlined in 
\citen{McDaniel2013} or obtain dispersion coefficients from an \isa-based
scheme (\cref{sec:workflow-alston}).\footnote{Scripts to perform the iterative
\idma fitting algorithm can be made available upon request.}
%
\item Dispersion coefficients should always be positive. Any negative
dispersion coefficients are likely a sign of unphysical atomic
polarizabilities (see next point).
%
\item Phsyically-speaking, the atomic polarizabilities at each rank should be positive
definite, and monotomically-decreasing.\cite{Williams20013,stone2013theory} Unphysical behavior (especially at
rank 3) is sometimes unavoidable, but often indicates poor fit quality and can
lead to inaccurate and/or non-transferable dispersion coefficients. Always
check the output \verb|.casimir| files for the physicality
(positive-definiteness and monatomic-decrease) of the frequency-dependent
polarizabilities for each atomtype and each rank.
%
\end{enumerate}

Finally, given a set of physical atomic polarizabilities and dispersion
coefficients, dispersion coefficients from the \idma method can be worked-up
using the post-processing scripts described in
\cref{sec:workflow-cn_postprocess}.
\end{subsubsection}







\begin{subsubsection}{\isa-pol}
\label{sec:workflow-alston}
%TODO: Check over these formulas with Alston

\begin{paragraph}{Theory}

Rather than iteratively fitting polarizabilities to reproduce the
point-to-point response, with \isa it is possible to
compute the atomic polarizabilities directly. First, note that the
frequency-dependent, molecular polarizabilities are given by the following formula:
\begin{align}
\label{eq:workflow-fdds}
\alpha_{lm,l'm'}(\omega) = \int \int
\hat{Q}_{lm}(\bm{r})
\alpha(\bm{r},\bm{r'}|\omega)
\hat{Q}_{l'm'}(\bm{r'})d\bm{r}d\bm{r'}
\end{align}
Here $\hat{Q}$ are the regular spherical harmonic operators (defined in
Appendix A of \citen{stone2013theory}) of rank $l$ and order $m$, and 
$\alpha(\bm{r},\bm{r'}|\omega)$ is the \fdds, or charge density
susceptibility, which measures the change in charge density at $\bm{r'}$ that
results from a delta-function change in the electric potential at point
$\bm{r}$. From \cref{eq:background-isa}, we have that
\begin{align}
1 = \sum\limits_a \left ( 
\frac{\bar{w}^a(\bm{r})}{\sum_m \bar{w}^m(\bm{r})}
\right )
= \sum\limits_a \bar{p}_a(\bm{r}),
\end{align}
where the bars indicate that we have normalized the \acrlong{aim} densities
and weight functions. Substituting this
equation into \cref{eq:workflow-fdds}, we arrive at an \isa-based definition
of the \aim polarizabilities:
\begin{align}
\label{eq:isa-pol}
\begin{split}
\alpha_{lm,l'm'}(\omega) &= \int \int
\hat{Q}_{lm}(\bm{r})
\alpha(\bm{r},\bm{r'}|\omega)
\hat{Q}_{l'm'}(\bm{r'})d\bm{r}d\bm{r'} \\
%
&= \sum\limits_a \sum\limits_b \int \int
\hat{Q}_{lm}(\bm{r})
p_a(\bm{r})
\alpha(\bm{r},\bm{r'}|\omega)
p_b(\bm{r'})
\hat{Q}_{l'm'}(\bm{r'})d\bm{r}d\bm{r'} \\
%
&\equiv \sum\limits_a \sum\limits_b 
\alpha_{lm,l'm'}^{ab}(\omega) 
\end{split}
\end{align}

While this formula bears similarity to \dma-based polarization
approaches,\cite{Williams2003,Misquitta2008} the advantage of
\cref{eq:isa-pol} is that the \aim polarizabilities are defined in a
physically-meaningful and transferable manner.
Consequently, with little refinement these \isa-based polarizabilities (\isa-pol) can be used to
directly obtain transferable dispersion coefficients for individual
\acrlong{aim},
all without recourse to the iterative fitting process required in
\cref{sec:workflow-jesse}.


\end{paragraph}
\begin{paragraph}{Practicals}

The \isa-pol method has been completely implemented as of \camcasp-6.0, though
the input scripts are (as of this writing) still in beta. Consult 
\href{http://www-stone.ch.cam.ac.uk/programs/camcasp.html}{the \camcasp user
manual} for up-to-date details and required input files.

\end{paragraph}

\end{subsubsection}
\begin{subsubsection}{Method Comparison}
\label{sec:workflow-dispersion_comparison}

%TODO: Add in Alston vs. Jesse comparison?

Preliminary results for the \isa-pol method, tested on the 91 dimer test set,
appear to be of similar accuracy compared to the \idma method, though both
methods appear to have their own strengths and weaknesses when it comes to
obtaining dispersion coefficients for different atomtypes. A comparison
between the two different methods is given in
\cref{tab:workflow-dispersion_comparison}. Overall, \isa-pol appears to give
more physically-meaningful atomic polarizabilities, whereas an isotropic \idma
description is (for
anisotropic systems) sometimes a better `effectively anisotropic'
model.\footnotemark{ }

\footnotetext{
A main difference between the \idma and \isa-pol coefficients is that \idma
fits more strongly to the p2p file, whereas \isa-pol coefficients are set to
the values calculated as in \cref{sec:workflow-alston}. Consequently, \idma is
able to perform better as an `effectively anisotropic' model. In principle,
changing the defaults in CamCASP to use weight type 4 (which uses
dipole-dipole terms as anchors, but completely fits higher ranking terms and
thus fits the p2p better) or 3 (uses all terms as anchors) and a weight
coefficient of 1e-5 (rather than 1e-3) should yield dispersion coefficients
more similar to \idma, though this idea requires further testing.
}

%%%%%%%%%%%%%%%%%%%%%%%%%%%%%%%%%%%%%%%%%%%%%%%%%%%%%%%%%%%%%%%%%%%%%%%%%%%%%%%%%%%
\begin{table}
\centering
%\begin{tabu}{@{}p{0.2\textwidth}@{}p{0.4\textwidth}p{0.4\textwidth}@{}}
\begin{tabu}{XX}

\rowfont[c]\bfseries
\idma & \isa-pol \\
\toprule
\multicolumn2{c}{\textbf{Ease of Parameterization}} \\
\tabuphantomline
\hline
\begin{itemize}[topsep=-1pt]
\item For systems with 1-2 unparameterized atomtypes, little cost to
parameterize new atomtypes
\item For systems requiring dispersion coefficients for several
unparameterized atomtypes, requires a library of systems containing these
atomtypes, and an iterative procedure to fit the new atomtypes.
\end{itemize}
& 
\begin{itemize}[topsep=0pt]
\item Straightforward for all molecules, regardless of number of
unparameterized atomtypes
\end{itemize}
\\ %
\multicolumn2{c}{\textbf{Physicality of the Distributed Polarizabilities}} \\
\tabuphantomline
\hline
\begin{itemize}[topsep=0pt]
\item Polarizabilities tend to be positive-definite and monotonically-decaying
at low rank, but not always at rank 3
\item Physicality highly-dependent on quality of previously parameterized
atomtypes
\end{itemize}
& 
\begin{itemize}[topsep=0pt]
\item With few exceptions, polarizabilities are positive-definite and monotonically-decaying at all
ranks
\end{itemize}
\\ %
\multicolumn2{c}{\textbf{Accuracy of the Dispersion Coefficients}} \\
\tabuphantomline
\hline
\begin{itemize}[topsep=0pt]
\item Good to excellent accuracy for atomtypes which have been fit to a
reproduce large library of molecular systems
\item Fair accuracy for certain atomtypes (such as chlorine or bromine) not
parameterized to an extensive library
\item For anisotropic systems (such as \co), tends to give a better isotropic
description than \isa-pol -- we hypothesize that this is a result of directly fitting the
point-to-point response, leading to an `effectively-anisotropic' model
\end{itemize}
& 
\begin{itemize}[topsep=0pt]
\item Good to very good accuracy for all tested systems, regardless of what
atomtypes are represented
\item Isotropic dispersion coefficients tend to give worse accuracy for
anisotropic systems compared to \idma, whereas anisotropic dispersion models
(see \cref{ch:mastiff} based on \isa-pol are of similar accuracy to the \idma
method
\end{itemize}
\\ %
\bottomrule
\end{tabu}
\caption{Comparison between the \idma and \isa-pol methods.}
\label{tab:workflow-dispersion_comparison}
\end{table}
%%%%%%%%%%%%%%%%%%%%%%%%%%%%%%%%%%%%%%%%%%%%%%%%%%%%%%%%%%%%%%%%%%%%%%%%%%%%%%%%%%%

\end{subsubsection}
\begin{subsubsection}{Dispersion Coefficient Post-processing}
\label{sec:workflow-cn_postprocess}

Regardless of which distribution method is used, some post-processing is
needed to transform the \isa-pol/\idma coefficients into optimal dispersion
force field parameters. In particular, while the \dftsapt energies from
\molpro and \camcasp should agree, in practice the different software packages
use different kernels (ALDA+LHF and ALDA+CHF, respectively) to calculate the
linear response functions. Consequently, this means that the dispersion
coefficients calculated in \camcasp are intended to reproduce the \camcasp-calculated
\dftsapt dispersion energies, but may only be approximately accurate for
\molpro-calculated \dftsapt dispersion energies.\footnotemark{ } In practice, the
\camcasp-calculated dispersion coefficients slightly underestimate the \molpro
dispersion energies, and the coefficients need to be scaled (usually by a factor of 1.03-1.10,
depending on the atomtype) to reproduce the \molpro energies. This scaling can
be carried out by executing the command
%
\begin{lstlisting}
./scripts/get_scaled_dispersion.py <scale_factor>
\end{lstlisting}
%
where \verb|<scale_factor>| is chosen to reproduce the asymptotic \molpro
\dftsapt energies (see \cref{ch:pointer}). This choice may require some
testing, but 1.10 is usually a good default. The above script outputs files
\verb|dispersion/<monomer>.disp|, which can be used as input to the \pointer
program discussed in \cref{ch:pointer}.

\footnotetext{
Additional reasons for discrepancies between CamCASP and MOLPRO dispersion
coefficients include the following:
\begin{enumerate}
\item For PBE calculations, \camcasp uses ALDA with PW91c correlation, whereas
\molpro uses VWN
\item \camcasp writes kernels completely in the
auxiliary basis set, whereas \molpro writes the kernel in a variety of basis
sets 
\end{enumerate}
}

\end{subsubsection}
