\begin{subsubsection}{Theory}

Dispersion coefficients can also be determined from distributed molecular
(that is, \aim)
property calculations using either an \isa- or \dma-based approach. The method
for obtaining distributed dispersion coefficients has been described in detail elsewhere for an assortment of \dma-based
approaches,
\cite{Williams2003,Misquitta2008,McDaniel2012,McDaniel2013,stone2013theory,McDaniel2014a}
and \citen{McDaniel2014a} in particular provides a useful summary of the
different equations and molecular properties that are needed to derive
the types of dispersion models used in \cref{ch:isaff,ch:mastiff}.
In brief, \aim dispersion energies can be obtained by integrating over
distributed 
frequency-dependent polarizabilities for each monomer.
%% \begin{align}
%% E^{ab}_\text{disp} = - \frac{\hbar}{2\pi} T^{ab}_{tu} T^{ab}_{t'u'}
%% \int\limits_{0}^{\infty} 
%% \alpha^a_{tt'} (i\omega) \alpha^b_{uu'} (i\omega) d\omega
%% \end{align}
%% where here 
%% $\alpha$ is the frequency-dependent polarizability for atom $a$ or $b$, $T$
%% is an interaction function (see Appendix F in \citen{stone2013theory} for
%% details), and
%% the subscripts $t,t',u,u'$ represent the rank and order (e.g. t = 11c, to use
%% Stone's notation\cite{stone2013theory}) of the associated spherical harmonic
%% component. 
Under the simplifying assumption that we can treat these polarizabilities as
isotropic,
the dispersion energy expression is given by
\begin{align}
\label{eq:workflow-edisp}
E^{ab}_{\text{disp}} &\approx - \frac{C^{ab}_6}{r_{ab}^6} - \frac{C^{ab}_8}{r_{ab}^8} - \ldots
\end{align}
for each atom pair, where 
\begin{align}
\label{eq:workflow-c6}
C^{ab}_6 &= \frac{3}{\pi} \int\limits_{0}^{\infty} \bar{\alpha}^a_{1}
(i\omega) \bar{\alpha}^b_{1} (i\omega) d\omega , \\
%
C^{ab}_8 &= \frac{15}{2\pi} \int\limits_{0}^{\infty} 
\bar{\alpha}^a_{1} (i\omega) \bar{\alpha}^b_{2} (i\omega) 
+ \bar{\alpha}^a_{2} (i\omega) \bar{\alpha}^b_{1} (i\omega) 
d\omega , 
\label{eq:workflow-c8}
\end{align}
and the higher order terms are defined analagously. $C^{ab}_n$ are the
atom-atom dispersion coefficients, and $\bar{\alpha}_l$ are the
rank $l$, isotropic, frequency-dependent polarizabilities. 
The formalisms in \cref{eq:workflow-c6,eq:workflow-c8} can be somewhat
involved,
but for our purposes the important take-away is the understanding that the dispersion
coefficients can be entirely determined by calculating the frequency-dependent
polarizabilities for each atom in its molecular environment.

Although it is straightforward to calculate \emph{molecular} frequency-dependent
polarizabilities, a central difficulty in obtaining transferable dispersion
coefficients is that we must have some physically-meaningful method for extracting distributed
\emph{\acrlong{aim}} polarizabilities from a molecular calculation. Many
distribution strategies exist in the literature, and here we mention two such
techniques. First, and as is used throughout this work, one can use a
\dma-based approach to partition the polarizabilities into \aim
contributions. In this case, and due to deficiencies in the \dma partitioning
scheme,
the resulting atomic polarizabilities are not always 
positive-definite, which is unphysical and can lead to a breakdown in
transferable parameterization.\cite{Williams2003}
To correct for this undesirable behavior, \citeauthor{McDaniel2013} have
proposed an constrained
optimization process whereby atomic polarizabilities can be iteratively fit to a library
of molecular polarizabilities.\cite{McDaniel2013} Although this process requires a reasonably
large library of representative atomtypes, the fitting
procedure does result in transferable atomic polarizabilities. Practical
details for this procedure are discussed in \cref{sec:workflow-jesse}.

As an alternative to the \dma-based polarization partitioning scheme, recently
Misquitta has developed an \isa-based partitioning scheme to extract the
atomic frequency-dependent polarizabilities. While this approach requires
further testing, in a manner analagous to our discussion of multipole moments (see \cref{ch:background} and
\citen{Misquitta2014}) the resulting `\isa-pol' method leads to a more
physically-meaningful partitioning of the molecular polarizabilities, which in
turn enables us to determine transferable dispersion coefficients without
resorting to the library-based paramterization process. Practical
issues with \isa-pol are the subject of \cref{sec:workflow-alston}, and a
comparison between the two methods for obtaining dispersion coefficients is
given in \cref{sec:workflow-dispersion_comparison}.



========================


In
particular, dispersion parameters are obtained from frequency dependent
polarizabilities: the linear response of a molecule to an electric field
generated by a point charge.

\cite{Williams2003,Misquitta2008}

calculations of the frequency
dependent polarizabilities associated with a given molecule.

\end{subsubsection}

\begin{subsubsection}{Iterative-\dma-pol}
\label{sec:workflow-jesse}

As described in \citen{McDaniel2014a}, the iterative-\dma-pol (\idma) method of
\citeauthor{McDaniel2013} performs a constrained optimization of
atomtype-specific frequency dependent polarizabilities by fitting all
polarizabilities to reproduce the so-called `point-to-point response',
$\alpha_{PQ}$. This `point-to-point' response is a
molecular quantity that
describes the change in electrostatic potential at point P due to an induced
change in the electron density of a molecule caused by a point charge
perturbation $q_Q$ at point Q. For an isotropic polarizability model,
%
\begin{align}
\label{eq:workflow-apq}
\alpha_{PQ} = -q_Q \sum\limits_{a,lm} T^{Pa}_{0,lm} \bar{\alpha}^a_l T^{aQ}_{lm,0}
\end{align}
%
where the $T$ are the spherical interaction functions described above and in
\citen{stone2013theory}. Aside from the isotropic polarizabilities
$\bar{\alpha}^a_l$, all quantities in \cref{eq:workflow-apq} are directly
calculated in \camcasp, enabling us to fit
the isotropic polarizabilities based on a \camcasp properties calculation.

The \idma method has a few additional dependencies: 
\begin{enumerate}
\item The \idma fiting program itself, which can be downloaded at
\url{https://github.com/mvanvleet/p2p-fitting}. Three executables
(\verb|main_dispersion|, \verb|main_drude|, and \verb|localize.sh|) need to be added to your bash
\verb|$PATH| for the scripts listed in this section to work properly.
%
\item \camcasp, which can be downloaded from 
\url{http://www-stone.ch.cam.ac.uk/programs/camcasp.html}. \camcasp also
requires several environment variables to be added to your bash \verb|$PATH|, and some of these
environment variables are also used by the \idma fitting program.
%
\end{enumerate}
and two additional input files:
\begin{enumerate}
\item \verb|input/<monomer>.atomtypes|: The \idma fitting program performs
a constrained optimization whereby the $\bar{\alpha}^a_l$ are set to be
identical for atoms with the same atomtype. Consequently, the
\verb|<monomer>.atomtypes| input file is required to specify the atomtypes in
each monomer. This .atomtypes file has the same
format as an .xyz file, with the exception that the element names for each atom
are replaced with a user-defined atomtype. See
\cref{lst:workflow-pyridine.atomtypes} for an example.
%
\item \verb|templates/dispersion_base_constraints.index|: As described below,
with \idma it is usually advisable to only fit one or two atomtype polarizabilities at a
time, with the remaining atomtype polarizabilities read in as hard
constraints. The \verb|dispersion_base_constraints.index| file lists these
hard constraints in a block format, 

\begin{minipage}{\linewidth}
%[float,floatplacement=H]
\begin{lstlisting}
CT
1 
 7.14483224 7.11095841 6.87452508 6.19718464 4.87589777 
 3.17818610 1.56461102 0.51670933 0.09175313 0.00367230 
2 
 20.26394042 20.00584110 17.66562710 14.33668329 12.03179893 
 11.49156262 7.86254302 3.10936998 0.53746459 0.01774391 
3 
 77.37303638 73.13014787 24.68682297 -13.48390193 0.40172836 
 29.76747226 34.31668916 17.88515654 3.13260459 0.10137127

\end{lstlisting}
\end{minipage}
which lists each constrained atomtype along with 10 frequency-dependent
polarizabilities for each polarizability rank (1-3). (\camcasp uses numerical
integration to solve \cref{eq:workflow-c6}, and the 10 polarizabilities per
rank correspond to the frequencies \camcasp needs to perform the numerical
quadrature. See the
\href{http://www-stone.ch.cam.ac.uk/programs/camcasp.html}{\camcasp user
manual} for details.) Each polarizability block should be separated by a blank
line, and the atomtypes listed in the .index file \emph{must} match those in
the .atomtypes file for any hard constraints to be applied. 
Previously-fit atomtype polarizabilities from \citen{McDaniel2013} are already included in 
\verb|dispersion_base_constraints.index| so as to minimize the number of hard
constraints that the user will need to add manually.
\end{enumerate}

Once all required input files have been created, and assuming the IP calculations
from \cref{sec:workflow-sapt} have already been performed,
the \camcasp calculations necessary to run the \idma program
can be performed by executing the command 
%
\begin{lstlisting}
./scripts/make_dmapol_files.py
\end{lstlisting}
%
and running the resulting input files through the \camcasp software (a process
which can take several hours).
Once the \camcasp calculations finish, dispersion coefficients can be obtained by running the following
work-up script:
%
\begin{lstlisting}
./scripts/workup_dispersion_files.sh
\end{lstlisting}
%
The resulting dispersion coefficients will be listed in the
\verb|dispersion/<monomer>.cncoeffs| output file.

When generating dispersion coefficients using \idma, the following sanity-checks should
always be performed:
\begin{enumerate}
\item The \verb|<monomer>_fit_dispersion.out| file lists the number and names of
unconstrained atomtypes. Ensure that the fit atomtypes match your
expectations, and that the number of fit atomtypes is kept relatively small
(1-2 max). If you need to fit multiple atomtypes simultaneously, or you obtain
unphysical disperion coefficients (see next point), you'll likely need to
utilize the iterative fitting algorithm outlined in 
\citen{McDaniel2013}.\footnote{Scripts to perform this iterative fitting
algorithm can be made available
upon request.}
%
\item Dispersion coefficients should always be positive. Any negative
dispersion coefficients are likely a sign of unphysical atomic
polarizabilities (see next point).
%
\item Phsyically-speaking, the atomic polarizabilities at each rank should be positive
definite, and monotomically-decreasing.\cite{Williams20013,stone2013theory} Unphysical behavior (especially at
rank 3) is sometimes unavoidable, but often indicates poor fit quality and can
lead to inaccurate and/or non-transferable dispersion coefficients. Always
check the output \verb|.casimir| files for the physicality
(positive-definiteness and monatomic-decrease) of the frequency-dependent
polarizabilities for each atomtype and each rank.
%
\end{enumerate}


Finally, given a set of physical atomic polarizabilities and dispersion
coefficients, dispersion coefficients from the \idma method can be worked-up
using the post-processing scripts described in
\cref{sec:workflow-cn_postprocess}.

\end{subsubsection}
\begin{subsubsection}{Alston's Method}
\label{sec:workflow-alston}

give brief theoretical overview of the ISA-pol method

%% αlm,l0m0(ω) = Z Z Qˆ lm(r)α(r, r 0 |ω)Qˆ l 0m0(r 0)drdr 0
%% = Z Z Qˆ lm(r)1α(r, r 0 |ω)1Qˆ l 0m0(r 0)drdr 0
%% = X a X b Z Z Qˆ lm(r)pa(r)α(r, r 0 |ω)pb(r 0)Qˆ l 0m0(r 0)drdr 0 
%% = X a X b α ab lm,l0m0(ω) 

\end{subsubsection}
\begin{subsubsection}{Dispersion Coefficient Post-processing}
\label{sec:workflow-cn_postprocess}

compare the two methods

why are the disp. coeffs not asymptotically-exact?

\end{subsubsection}
