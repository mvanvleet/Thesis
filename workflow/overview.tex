Due in part to the improvements in \cref{ch:isaff,ch:mastiff}, the development
protocol for \sapt-based, ab initio force fields is now fairly robust with
respect to many parameterization details. Consequently, much of the workflow
is now automated and requires little user input. The following sections are
designed to give future users familiarity this workflow, not only as a
``blackbox'' tool, but also as a starting point for more complex and/or
system-specific force field development. To this end, we first provide an
overview of the workflow itself, and then describe the theoretical and
practical details of each step in subsequent sections.

In order to gain expertise in practical force field development, new force field developers are encouraged to read through (in order)
\cref{ch:background,ch:workflow,ch:pointer} to obtain a conceptual
understanding of the force field development process, after which they should
work on
developng their own force field using
the semi-automated workflow (\cref{ch:workflow}) and the \pointer software
(\cref{ch:pointer}). Developing a water force field
makes for an excellent teaching example, however any interesting (and preferably
small!) molecule will suffice.

\begin{section}{Overview}

As discussed in \cref{ch:background}, our \sapt-based force field 
methodology principally requires us to parameterize the two-body interactions
for a given system of interest. These two-body (i.e. dimer)
interactions are completely defined by the 
positions and relative orientation of the two constituent monomers, and in
practice we parameterize the two-body interactions based on benchmark \sapt
energies for a series of gas-phase dimer 
%
configurations.\footnote{This philosophy of force field development was one of the most counter-intuitive ideas I
had to learn as a grad student. Still, and regardless of whether we are ultimately
interested in studying a homogeneous liquid or a heterogeneous supercritical
phase, the best starting point for ab initio force field development is always
to model all relevant gas-phase dimer interactions.}
%
We are usually interested in obtaining
transferable parameters for a new molecule or atomtype, in which case it is
often easiest to model the interactions between two identical monomers (a
so-called \homo dimer interaction). Still, there are reasons why it can
be advantageous to instead study \hetero dimer 
%
interactions,\footnote{In general, force field development based on \homo
interactions involves the fewest atomtypes, and thus the fewest number of free
parameters. Alternately, \hetero-based force field development can yield the
best accuracy for studying systems where either transferability is difficult
(see \cref{ch:lmoeda} for an example) or computational expense is an issue.
(For instance, running \sapt calculations on a napthalene dimer may be
infeasible, whereas studying napthalene-Ar interactions is easily possible.)}
%
and the workflow
described herein applies equally to studying both \homo and \hetero
%
interactions.

Regardless of the chosen dimer of study, developing force fields for the
two-body \pes involves two major steps.  First, we must
obtain benchmark two-body energies for a series of well-chosen dimer
configurations. Second, we must
calculate and/or fit all force field parameters so as to completely describe a
model for the two-body interaction energies. For the \sapt-based force fields described in
\cref{ch:isaff,ch:mastiff}, 
these two overarching steps lead to the following workflow:
%
\begin{figure}
\begin{enumerate}[I.]
\item Generate benchmark two-body energies
\label{workflow:step1}
    \begin{enumerate}[1.]
    \item Generate a series of well-chosen dimer configurations
        (see \cref{sec:workflow-geometries})
    \item Calculate \dftsapt benchmark energies for all dimer configurations from
the previous step
        (see \cref{sec:workflow-sapt})
    \item Optionally (depending on system size and the accuracy of \dftsapt
for the chosen system), calculate \ccsdt or \ccsdtf benchmark energies in
order to correct the \dftsapt energies above
        (see \cref{sec:workflow-ccsdt})
    \end{enumerate}
\item Parameterize the two-body \pes
\label{workflow:step2}
    \begin{enumerate}[1.]
    \item For each unique monomer, obtain the following monomer-specific parameters:
        \begin{enumerate}
        \item Multipole moments, $Q$
            (see \cref{sec:workflow-multipoles})
        \item \acrshort{isa} Exponents, $B$ 
            (see \cref{sec:workflow-exponents})
        \item Induced Dipole Polarizabilities, $\alpha$
            (see \cref{sec:workflow-polarizabilities})
        \item Dispersion Coefficients, $C_n$
            (see \cref{sec:workflow-dispersion})
        \end{enumerate}
    \item Obtain all remaining force field parameters by fitting a chosen
    force field functional form to the two-body benchmark energies from Step \ref{workflow:step1}
        (see \cref{ch:pointer})
        %\label{step:workflow-dimers}
    \end{enumerate}
\end{enumerate}
\caption{The workflow for \sapt-based force field development.}
\end{figure}

Aside from the last step of this workflow, which will be the subject of
\cref{ch:pointer}, the entire force field development
process has been made reasonably `black-box', and can be carried out via a
handful of input files and easy-to-use run scripts. This semi-automated workflow
for \sapt-based force field development is available for download 
at
\url{https://github.com/mvanvleet/workflow-for-force-fields}, and should be
sufficient for most routine force field development. Installation and
usage instructions are included on the website, and are also reprinted in
\cref{fig:workflow-overview} for conveninece. 


% Completion of the steps shown in
% \cref{fig:workflow-overview} will provide all the necessary electronic structure
% benchmarks and monomer parameters required for developing \sapt-based

\end{section}
