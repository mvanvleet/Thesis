\begin{subsubsection}{Theory}

In addition to frequency-dependent polarizabilities, some of the same techniques
described in \cref{sec:workflow-dispersion} can be applied to obtain the
drude oscillator charges that get used in modeling the
induction energy. Though in principle \isa-based polarizabilities could be
used, this technique has not yet been developed. Instead, an \idma-type
procedure can be used to extract the necessary polarization parameters. The
algorithms used to perform this procedure are described in Appendix A of
\citen{McDaniel2014a}. Due to the reduced number of coefficients that need to
be fit, this optimization is generally more robust, and leads to more
transferable dispersion parameters, than the algorithms described in
\cref{sec:workflow-dispersion}.

\end{subsubsection}
\begin{subsubsection}{Practicals}

The drude oscillator fitting code has the same dependencies and input files as
\idma, with the exception that the \verb|dispersion_base_constraints.index|
file is replaced with the following constraint file:
\begin{enumerate}
\item \verb|drude_base_constraints.index|: As with \idma,
it is usually advisable to only fit a few atomtype static polarizabilities at a
time, with the remaining atomtype polarizabilities read in as hard
constraints. The \verb|drude_base_constraints.index| file lists these
hard constraints in a block format, 

\begin{minipage}{\linewidth}
%[float,floatplacement=H]
\begin{lstlisting}
C
1
0.0

N     
1
-11.7529643 

H     
1
-1.254

\end{lstlisting}
\end{minipage}
which lists each constrained atomtype along with the rank 1 static
polarizabilities, separated by a blank line. Unlike with the generation of
dispersion coefficients, an initial guess must be given for \emph{all}
atomtypes in the \verb|<monomer>.atomtypes| input file.
The format for the  \verb|drude_base_constraints.index| is such that
positive polarizabilities correspond to these initial guesses, whereas zero or
negative entries for the polarizabilities indicate that the atomtype should be
treated as a hard constraint.
Previously-fit atomtype polarizabilities from \citen{McDaniel2013} are already included in 
\verb|drude_base_constraints.index| so as to minimize the number of hard
constraints that the user will need to add manually, and these hard
constraints should be used whenever possible.
\end{enumerate}

Assuming that the \idma calculations have already been run in \camcasp, the
drude oscillator coefficients can be obtained simply by executing
%
\begin{lstlisting}
./scripts/workup_drude_files.sh
\end{lstlisting}
%
As with the dispersion coefficients, care should be taken to ensure that the
resulting drude oscillator charges are physically-meaningful (i.e. negative).

\end{subsubsection}
