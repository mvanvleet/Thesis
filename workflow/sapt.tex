
\begin{section}{\sapt Benchmarks}
\label{sec:workflow-sapt}

After geometry generation, the next step 
in the Workflow
is to run benchmark \dftsapt calculations on all dimer configurations.
For a detailed analysis of \sapt, and \dftsapt in particular, the reader is
referred to 
\citens{Jeziorski1994,Szalewicz2012,McDaniel2014a}.
\dftsapt calculations can be performed in a fairly black-box
manner using the \molpro software, though the following points are worth note:
\begin{enumerate}
\item For best accuracy, and as described in \citen{Yu2011}, monomer \dft
calculations need to be asymptotically-corrected (AC) in order to achieve best accuracy. This
asymptotic correction is computed as the difference between the HOMO and the
vertical ionization potential for each monomer, and can be calculated
automatically by running the command 
\begin{lstlisting}
./scripts/submit_ip_calcs.py
\end{lstlisting}
(The calculation takes only a few minutes for small molecules, but may take
longer for larger systems.) Importantly, the HOMO calculation should be computed
using the same basis set as the \dftsapt calculations themselves.
\item Accurate \sapt dispersion energies generally require use of midbond
functions, as described in \citen{Yu2011}. Locations for the midbond functions
can be specified in the \verb|dimer_info.dat| file. For most small molecules
(such as those described in \cref{ch:isaff}), it is
often sufficient to place a single midbond at the midpoint between each
monomer's center of mass. For larger molecules, additional midbonds
(especially ones near close-contact interaction sites) may be required.
\item The included workflow assumes an \avtzm basis set (where the +m represents
the midbond functions). This is generally of sufficient accuracy for most
systems, though an \avqzm basis set should be used when possible to ensure
convergence of the \dftsapt dispersion energies.
\end{enumerate}

Once the appropriate midbond functions have been added to the
\verb|dimer_info.dat| input file, and the AC calculations have finished, the
\dftsapt input files can be generated by executing the command

\begin{lstlisting}
./scripts/make_sapt_ifiles.py
\end{lstlisting}

The resulting input files can then be run using the Molpro software, either in
serial or in parallel. \emph{Care should be taken to ensure that multiple
calculations do not end up on the same compute note, as this can often result
in i/o caching issues and reduced computational efficiency.}

\end{section}
