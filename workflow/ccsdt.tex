
\begin{section}{\ccsdt Calculations}
\label{sec:workflow-ccsdt}

When affordable, \ccsdt calculations should be run on (at least a subset of) the
dimer configurations, both in order to benchmark the \dftsapt energies and to
provide a \dccsdt correction for later fitting of the \sapt potential.
Recently, an explicitly-correlated \ccsdtf method has been proposed, which for
practical purposes is identical to \ccsdt but with faster basis set
convergence.\cite{Knizia2009}
Usually \ccsdtf{a}/\avtzm is an excellent
approximation of the \ccsdt/CBS limit. The input files for \ccsdtf/\avtzm
calculations can be set up by executing the command
%
\begin{lstlisting}
./scripts/make_ccsdt_ifiles.py
\end{lstlisting}
%
and by running each input file using the \molpro software package.


\end{section}
