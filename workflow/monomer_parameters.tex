\begin{section}{Monomer-Based Parameterization}
\label{sec:workflow-monomer_parameters}

While/after running the \dftsapt calculations, the next step in the Workflow
is to compute various force field parameters which only depend on the
identities of the individual monomers themselves. The following subsections describe the
calculations of multipole moments (\cref{sec:workflow-multipoles}),
short-range exponents (\cref{sec:workflow-exponents}), dispersion coefficients
(\cref{sec:workflow-dispersion}), and induced dipole polarizabilities
(\cref{sec:workflow-polarizabilities}). First, however, we 
outline the scope and useful features of the \camcasp software used to
perform these monomer property calculations.

\begin{subsection}{Distributed Property Calculations using \camcasp}

\camcasp is a collection of scripts and programs useful for (among other
things) the calculation of distributed multipoles and
polarizabilities.\cite{camcasp5.8} Of particular importance is the
choice of distribution method, as this determines how the various molecular properties
of interest should be mapped onto corresponding atom-in-molecule properties. Currently, two
main distribution (or `charge partitioning') schemes are available in \camcasp:
\dma\cite{Stone2005} and
\isa.\cite{Misquitta2014} The theory behind the \isa procedure has already been detailed in
\cref{ch:introduction}, and monomer property calculations using \dma are
described in 
\citen{Stone2005,Misquitta2006,McDaniel2014a}. In general, and where
available, \isa-based properties are to be preferred, and we recommend an
\isa-based parameterization scheme for obtaining multipoles and
atom-in-molecule exponents. A \dma-based method is currently required for
obtaining dispersion coefficients and static polarizabilities, though
\isa-based strategies for these properties are under active development and
(in the case of dispersion) are discussed in \cref{sec:workflow-alston}.
A complete overview of
available property calculations and distribution schemes, along with relevant references,
is given in \cref{tab:workflow-distribution_schemes}.


\begin{table}[ht]
\centering
\begin{tabular}{@{}lcc@{}}
\hline
\toprule
\multirow{2}{*}{Property}
& \multicolumn{2}{c}{Parameterization Scheme} \\
\cmidrule{2-3} 
                         &  ISA                                    & DMA          \\ 
\midrule
\multirow{2}{*}{Multipoles}               & \cref{sec:workflow-multipoles}          &                      --               \\  
                                          & \citen{Misquitta2014}                   & \citen{Stone2005,McDaniel2014a}                 \\  
\addlinespace
\multirow{2}{*}{Exponents}                &  \cref{sec:workflow-exponents}          & \multirow{2}{*}{--}                   \\ 
                                          &  \citen{VanVleet2016}                   &                                       \\ 
\addlinespace
\multirow{2}{*}{Dispersion Coefficients}  &  \cref{sec:workflow-dispersion}         & \cref{sec:workflow-dispersion}        \\ 
                                          &  --                                     & \citen{McDaniel2013}        \\ 
\addlinespace
\multirow{2}{*}{Dipole Polarizabilities}  &  \multirow{2}{*}{--}                    & \cref{sec:workflow-polarizabilities}  \\ 
                                          &                                         & \citen{McDaniel2013}                  \\ 
\addlinespace
\bottomrule
\hline
\end{tabular}
\caption
[Overview of \isa- and \dma-based methods for obtaining distributed monomer
properties]
{Overview of \isa- and \dma-based methods for obtaining distributed monomer
properties. Details for each monomer parameterization are given in the listed
section and/or reference.}
\label{tab:workflow-distribution_schemes}
\end{table}

\end{subsection}



%%%%%%%%%%%%%%%% MULTIPOLES %%%%%%%%%%%%%%%%%%%%%%%%%%%
\begin{subsection}{Multipoles}
\label{sec:workflow-multipoles}

\begin{subsubsection}{Practicals}

\isa-based multipoles are described in detail in \citen{Misquitta2014}, and can
be calculated using the \camcasp software. To set-up the \isa calculations,
execute the command
%
\begin{lstlisting}
./scripts/make_isa_files.py
\end{lstlisting}
%
which creates the necessary \isa files for calculating both distributed
multipoles and exponents (see \cref{sec:workflow-exponents}). After running
these calculations (a process which may require several hours, depending on
the molecule), the multipoles can be extracted simply by running
%
\begin{lstlisting}
./scripts/workup_isa_charges.py
\end{lstlisting}
%
This work-up script produces several output files,
\begin{itemize}[noitemsep,label=]
\item \verb|<monomer>_ISA_L4.mom|
\item \verb|<monomer>_ISA_L2.mom|
\item \verb|<monomer>_ISA_L0.mom|
%\item \verb|<monomer>_ISA_L4_to_L0.mom|
\end{itemize}
which correspond to multipole moments for various long-range electrostatic
models. Using Stone's notation,\cite{stone2013theory} the $Lx$ suffix refers
to the highest order of multipole moments ($L0$ = point charges, $L1$ =
dipoles, $L2$ = quadrupoles, etc.) included in the model. The $L4$ model is
output by the \camcasp software package, and the $L2$ and $L0$ models are
generated by rank-truncation (that is, zeroing out) of the higher-order multipole
moments. 
For most routine
force field development, the $L2$ model is to be preferred for its balance of
accuracy and computational expense. Next, however, we
discuss situations in which different electrostatic models may be desirable.

\end{subsubsection}
\begin{subsubsection}{Advanced Multipole Parameterization Options}

As stated above, for the purposes of obtaining sub-\kjmol{} accuracy
force fields it is often important to model the long-range electrostatics
using \isa-based multipoles truncated to no farther than quadrupolar (i.e. `rank 2' or
$L2$)\cite{stone2013theory} contributions.
Due to computational and/or software limitations, however, there exist
practical cases where it becomes advantageous to exclude all higher-order
multipole moments.\cite{Cardamone2014} In such cases, 
two different types of long-range electrostatic models are useful. First, for
reasonably isotropic molecules a good option is to rank-truncate the \isa
multipoles to the $L0$ point charge contributions, thus yielding a so-called `atom-centered
point charge model'. On the other hand, for more anisotropic functional groups such as those
described in \citen{Kramer2014}, an atom-centered point charge model can be
insufficiently accurate, making it 
necessary to model the long-range electrostatics by including additional 
`off-center/off-site' point charges. Given a well-chosen set of off-site charges, 
an off-center point charge model usually can reasonably reproduce the effects of the
neglected higher-order multipole moments.\cite{Dixon1997}
%% (Using water as an example, off-site charges have been shown to substantially improve
%% force field accuracy, and can either be placed in a
%% tetrahedral arrangement, requiring two additional charge sites, or in the
%% \ch{H-O-H} plane itself, requiring one additional charge site.\cite{Tran2016}) 
In the past, locations for the off-center charges have usually been manually tuned or optimized
in a system-specific manner, though recent work suggests the
possibility of switching to non-empirical methods in order to more easily calculate/optimize 
positions for the extra-atom sites.\cite{Chaudret2013,Unke2017} 

For atom-centered point charge models, the output of the
\verb|workup_isa_charges.py| script automatically provides the required
rank-truncated multipole file (listed as \verb|<monomer>_ISA_L0.mom| in the \verb|isa/|
sub-directory). Note that, because the \verb|<monomer>_ISA_L0.mom| file is given as a
simple rank-truncation of the more complete \verb|<monomer>_ISA_L2.mom|
multipoles, the $L0$ moments (that is, point charges) are identical between the two files.

For developing rank-transformed point charge models, \citeauthor{Ferenczy1997} has
programmed a method for calculating electrostatic potential-fitted charges,
which can be thought of as a `rank transformation' procedure. The author's
MULFIT program can be 
downloaded online at
\url{http://www-stone.ch.cam.ac.uk/pub/gdma/index.php}, and documentation for
the program is found in the \verb|documentation/| sub-directory of the
Workflow. Assuming the \verb|mulfit| executable is in your \verb|$PATH|, a
basic rank transformation can be performed using the following steps:
%
\begin{lstlisting}[language=bash]
cp templates/mulfit.inp isa/<monomer>/OUT/
cd isa/<monomer>/OUT/
mulfit < mulfit.inp
\end{lstlisting}
%
Here the default \verb|mulfit.inp| file is set to take in the $L4$ rank
multipoles and rank-transform them to an $L0$ model. In this case, note that
the $L0$ moments between the rank-transformed and rank-truncated moments will
\emph{not} be identical, and testing is required to ascertain which
moments yield optimal force field parameters.

The MULFIT program can additionally be used to develop off-site point charge models. In
this case, the input multipole file (default \verb|ISA_L4.mom|) should be
edited to include the additional sites, and an example of the required syntax
is given in \verb|documentation/examples/ISA_L4_offsites.mom| for a
4-site water model. Importantly, the MULFIT program does not help
optimize the position(s) of the off-site charge(s), and thus the task of choosing
the number and position(s) of the off-site(s) is left to
the user.

After fitting multipole parameters with the MULFIT program, the program output
gives two indications of fit quality. First, the agreement between the total
reference and fitted multipoles moments is listed, and this should be taken as
a primary indication of multipole quality. Second, the program gives a
`Goodness of fit' parameter, expressed as an energy. While difficult to
interpret in an absolute sense, in comparing different rank-transformed models
we have generally found that models with lower `Goodness of fit' parameters
yield better force field fits.

\end{subsubsection}
\end{subsection}
%%%%%%%%%%%%%%%% MULTIPOLES %%%%%%%%%%%%%%%%%%%%%%%%%%%

%%%%%%%%%%%%%%%% EXPONENTS %%%%%%%%%%%%%%%%%%%%%%%%%%%%
\begin{subsection}{ISA Exponents}
\label{sec:workflow-exponents}

%\begin{subsubsection}{Overview}

As described in \cref{ch:introduction,ch:isaff}, the \isa procedure
produces a set of distributed \aim electron densities. The
orientational average of each of these \aim densities, or
`shape-functions', are spherically-symmetric quantities that describe the
radial decay of the \aim density.\cite{Misquitta2014} As described in
\cref{ch:isaff}, and using the algorithm detailed in 
\cref{sec:workflow-exponent_algorithm},
the shape-functions can be fit to a Slater-type function in
order to yield an isotropic, exponentially-decaying model for the \isa
densities. Importantly, the Slater-exponents in this density model directly
yield the exponents necessary to describe short-range effects (such as
exchange-repulsion and charge penetration) in the two-body force
field (see \cref{ch:isaff} for details).

Assuming the \isa calculations have already been run to obtain multipole
moments (see previous section), the \isa exponents can be obtained very simply
by running the command
%
\begin{lstlisting}
./scripts/workup_isa_exponents.py
\end{lstlisting}
%
The resulting exponents are given in the file \verb|isa/<monomer>.exp|, which
uses a file format recognized by the \pointer pre-prossessing scripts (see
\cref{ch:pointer}).

\end{subsection}
%%%%%%%%%%%%%%%% EXPONENTS %%%%%%%%%%%%%%%%%%%%%%%%%%%%


%%%%%%%%%%%%%%%% DISPERSION %%%%%%%%%%%%%%%%%%%%%%%%%%%
\begin{subsection}{Dispersion Coefficients}
\label{sec:workflow-dispersion}

\begin{subsubsection}{Theory}

Dispersion coefficients can also be determined from distributed molecular
(that is, \aim)
property calculations using either an \isa- or \dma-based approach. The method
for obtaining distributed dispersion coefficients has been described in detail elsewhere for an assortment of \dma-based
approaches,
\cite{Williams2003,Misquitta2008,McDaniel2012,McDaniel2013,stone2013theory,McDaniel2014a}
and \citen{McDaniel2014a} in particular provides a useful summary of the
different equations and molecular properties that are needed to derive
the types of dispersion models used in \cref{ch:isaff,ch:mastiff}.
In brief, \aim dispersion energies can be obtained by integrating over
distributed 
frequency-dependent polarizabilities for each monomer.
%% \begin{align}
%% E^{ab}_\text{disp} = - \frac{\hbar}{2\pi} T^{ab}_{tu} T^{ab}_{t'u'}
%% \int\limits_{0}^{\infty} 
%% \alpha^a_{tt'} (i\omega) \alpha^b_{uu'} (i\omega) d\omega
%% \end{align}
%% where here 
%% $\alpha$ is the frequency-dependent polarizability for atom $a$ or $b$, $T$
%% is an interaction function (see Appendix F in \citen{stone2013theory} for
%% details), and
%% the subscripts $t,t',u,u'$ represent the rank and order (e.g. t = 11c, to use
%% Stone's notation\cite{stone2013theory}) of the associated spherical harmonic
%% component. 
Under the simplifying assumption that we can treat these polarizabilities as
isotropic,
the dispersion energy expression is given by
\begin{align}
\label{eq:workflow-edisp}
E^{ab}_{\text{disp}} &\approx - \frac{C^{ab}_6}{r_{ab}^6} - \frac{C^{ab}_8}{r_{ab}^8} - \ldots
\end{align}
for each atom pair, where 
\begin{align}
\label{eq:workflow-c6}
C^{ab}_6 &= \frac{3}{\pi} \int\limits_{0}^{\infty} \bar{\alpha}^a_{1}
(i\omega) \bar{\alpha}^b_{1} (i\omega) d\omega , \\
%
C^{ab}_8 &= \frac{15}{2\pi} \int\limits_{0}^{\infty} 
\bar{\alpha}^a_{1} (i\omega) \bar{\alpha}^b_{2} (i\omega) 
+ \bar{\alpha}^a_{2} (i\omega) \bar{\alpha}^b_{1} (i\omega) 
d\omega , 
\label{eq:workflow-c8}
\end{align}
and the higher order terms are defined analagously. $C^{ab}_n$ are the
atom-atom dispersion coefficients, and $\bar{\alpha}_l$ are the
rank $l$, isotropic, frequency-dependent polarizabilities. 
The formalisms in \cref{eq:workflow-c6,eq:workflow-c8} can be somewhat
involved,
but for our purposes the important take-away is the understanding that the dispersion
coefficients can be entirely determined by calculating the frequency-dependent
polarizabilities for each atom in its molecular environment.

Although it is straightforward to calculate \emph{molecular} frequency-dependent
polarizabilities, a central difficulty in obtaining transferable dispersion
coefficients is that we must have some physically-meaningful method for extracting distributed
\emph{\acrlong{aim}} polarizabilities from a molecular calculation. Many
distribution strategies exist in the literature, and here we mention two such
techniques. First, and as is used throughout this work, one can use a
\dma-based approach to partition the polarizabilities into \aim
contributions. In this case, and due to deficiencies in the \dma partitioning
scheme,
the resulting atomic polarizabilities are not always 
positive-definite, which is unphysical and can lead to a breakdown in
transferable parameterization.\cite{Williams2003}
To correct for this undesirable behavior, \citeauthor{McDaniel2013} have
proposed an constrained
optimization process whereby atomic polarizabilities can be iteratively fit to a library
of molecular polarizabilities.\cite{McDaniel2013} Although this process requires a reasonably
large library of representative atomtypes, the fitting
procedure does result in transferable atomic polarizabilities. Practical
details for this procedure are discussed in \cref{sec:workflow-jesse}.

As an alternative to the \dma-based polarization partitioning scheme, recently
Misquitta has developed an \isa-based partitioning scheme to extract the
atomic frequency-dependent polarizabilities. While this approach requires
further testing, in a manner analagous to our discussion of multipole moments (see \cref{ch:background} and
\citen{Misquitta2014}) the resulting `\isa-pol' method leads to a more
physically-meaningful partitioning of the molecular polarizabilities, which in
turn enables us to determine transferable dispersion coefficients without
resorting to the library-based paramterization process. Practical
issues with \isa-pol are the subject of \cref{sec:workflow-alston}, and a
comparison between the two methods for obtaining dispersion coefficients is
given in \cref{sec:workflow-dispersion_comparison}.



========================


In
particular, dispersion parameters are obtained from frequency dependent
polarizabilities: the linear response of a molecule to an electric field
generated by a point charge.

\cite{Williams2003,Misquitta2008}

calculations of the frequency
dependent polarizabilities associated with a given molecule.

\end{subsubsection}

\begin{subsubsection}{Iterative-\dma-pol}
\label{sec:workflow-jesse}

As described in \citen{McDaniel2014a}, the iterative-\dma-pol (\idma) method of
\citeauthor{McDaniel2013} performs a constrained optimization of
atomtype-specific frequency dependent polarizabilities by fitting all
polarizabilities to reproduce the so-called `point-to-point response',
$\alpha_{PQ}$. This `point-to-point' response is a
molecular quantity that
describes the change in electrostatic potential at point P due to an induced
change in the electron density of a molecule caused by a point charge
perturbation $q_Q$ at point Q. For an isotropic polarizability model,
%
\begin{align}
\label{eq:workflow-apq}
\alpha_{PQ} = -q_Q \sum\limits_{a,lm} T^{Pa}_{0,lm} \bar{\alpha}^a_l T^{aQ}_{lm,0}
\end{align}
%
where the $T$ are the spherical interaction functions described above and in
\citen{stone2013theory}. Aside from the isotropic polarizabilities
$\bar{\alpha}^a_l$, all quantities in \cref{eq:workflow-apq} are directly
calculated in \camcasp, enabling us to fit
the isotropic polarizabilities based on a \camcasp properties calculation.

The \idma method has a few additional dependencies: 
\begin{enumerate}
\item The \idma fiting program itself, which can be downloaded at
\url{https://github.com/mvanvleet/p2p-fitting}. Three executables
(\verb|main_dispersion|, \verb|main_drude|, and \verb|localize.sh|) need to be added to your bash
\verb|$PATH| for the scripts listed in this section to work properly.
%
\item \camcasp, which can be downloaded from 
\url{http://www-stone.ch.cam.ac.uk/programs/camcasp.html}. \camcasp also
requires several environment variables to be added to your bash \verb|$PATH|, and some of these
environment variables are also used by the \idma fitting program.
%
\end{enumerate}
and two additional input files:
\begin{enumerate}
\item \verb|input/<monomer>.atomtypes|: The \idma fitting program performs
a constrained optimization whereby the $\bar{\alpha}^a_l$ are set to be
identical for atoms with the same atomtype. Consequently, the
\verb|<monomer>.atomtypes| input file is required to specify the atomtypes in
each monomer. This .atomtypes file has the same
format as an .xyz file, with the exception that the element names for each atom
are replaced with a user-defined atomtype. See
\cref{lst:workflow-pyridine.atomtypes} for an example.
%
\item \verb|templates/dispersion_base_constraints.index|: As described below,
with \idma it is usually advisable to only fit one or two atomtype polarizabilities at a
time, with the remaining atomtype polarizabilities read in as hard
constraints. The \verb|dispersion_base_constraints.index| file lists these
hard constraints in a block format, 

\begin{minipage}{\linewidth}
%[float,floatplacement=H]
\begin{lstlisting}
CT
1 
 7.14483224 7.11095841 6.87452508 6.19718464 4.87589777 
 3.17818610 1.56461102 0.51670933 0.09175313 0.00367230 
2 
 20.26394042 20.00584110 17.66562710 14.33668329 12.03179893 
 11.49156262 7.86254302 3.10936998 0.53746459 0.01774391 
3 
 77.37303638 73.13014787 24.68682297 -13.48390193 0.40172836 
 29.76747226 34.31668916 17.88515654 3.13260459 0.10137127

\end{lstlisting}
\end{minipage}
which lists each constrained atomtype along with 10 frequency-dependent
polarizabilities for each polarizability rank (1-3). (\camcasp uses numerical
integration to solve \cref{eq:workflow-c6}, and the 10 polarizabilities per
rank correspond to the frequencies \camcasp needs to perform the numerical
quadrature. See the
\href{http://www-stone.ch.cam.ac.uk/programs/camcasp.html}{\camcasp user
manual} for details.) Each polarizability block should be separated by a blank
line, and the atomtypes listed in the .index file \emph{must} match those in
the .atomtypes file for any hard constraints to be applied. 
Previously-fit atomtype polarizabilities from \citen{McDaniel2013} are already included in 
\verb|dispersion_base_constraints.index| so as to minimize the number of hard
constraints that the user will need to add manually.
\end{enumerate}

Once all required input files have been created, and assuming the IP calculations
from \cref{sec:workflow-sapt} have already been performed,
the \camcasp calculations necessary to run the \idma program
can be performed by executing the command 
%
\begin{lstlisting}
./scripts/make_dmapol_files.py
\end{lstlisting}
%
and running the resulting input files through the \camcasp software (a process
which can take several hours).
Once the \camcasp calculations finish, dispersion coefficients can be obtained by running the following
work-up script:
%
\begin{lstlisting}
./scripts/workup_dispersion_files.sh
\end{lstlisting}
%
The resulting dispersion coefficients will be listed in the
\verb|dispersion/<monomer>.cncoeffs| output file.

When generating dispersion coefficients using \idma, the following sanity-checks should
always be performed:
\begin{enumerate}
\item The \verb|<monomer>_fit_dispersion.out| file lists the number and names of
unconstrained atomtypes. Ensure that the fit atomtypes match your
expectations, and that the number of fit atomtypes is kept relatively small
(1-2 max). If you need to fit multiple atomtypes simultaneously, or you obtain
unphysical disperion coefficients (see next point), you'll likely need to
utilize the iterative fitting algorithm outlined in 
\citen{McDaniel2013}.\footnote{Scripts to perform this iterative fitting
algorithm can be made available
upon request.}
%
\item Dispersion coefficients should always be positive. Any negative
dispersion coefficients are likely a sign of unphysical atomic
polarizabilities (see next point).
%
\item Phsyically-speaking, the atomic polarizabilities at each rank should be positive
definite, and monotomically-decreasing.\cite{Williams20013,stone2013theory} Unphysical behavior (especially at
rank 3) is sometimes unavoidable, but often indicates poor fit quality and can
lead to inaccurate and/or non-transferable dispersion coefficients. Always
check the output \verb|.casimir| files for the physicality
(positive-definiteness and monatomic-decrease) of the frequency-dependent
polarizabilities for each atomtype and each rank.
%
\end{enumerate}


Finally, given a set of physical atomic polarizabilities and dispersion
coefficients, dispersion coefficients from the \idma method can be worked-up
using the post-processing scripts described in
\cref{sec:workflow-cn_postprocess}.

\end{subsubsection}
\begin{subsubsection}{Alston's Method}
\label{sec:workflow-alston}

give brief theoretical overview of the ISA-pol method

%% αlm,l0m0(ω) = Z Z Qˆ lm(r)α(r, r 0 |ω)Qˆ l 0m0(r 0)drdr 0
%% = Z Z Qˆ lm(r)1α(r, r 0 |ω)1Qˆ l 0m0(r 0)drdr 0
%% = X a X b Z Z Qˆ lm(r)pa(r)α(r, r 0 |ω)pb(r 0)Qˆ l 0m0(r 0)drdr 0 
%% = X a X b α ab lm,l0m0(ω) 

\end{subsubsection}
\begin{subsubsection}{Dispersion Coefficient Post-processing}
\label{sec:workflow-cn_postprocess}

compare the two methods

why are the disp. coeffs not asymptotically-exact?

\end{subsubsection}


\end{subsection}
%%%%%%%%%%%%%%%% DISPERSION %%%%%%%%%%%%%%%%%%%%%%%%%%%

%%%%%%%%%%%%%%%% POLARIZATION %%%%%%%%%%%%%%%%%%%%%%%%%
\begin{subsection}{Polarization Charges}
\label{sec:workflow-polarizabilities}

\begin{subsubsection}{Theory}

In addition to frequency-dependent polarizabilities, some of the same techniques
described in \cref{sec:workflow-dispersion} can be applied to obtain the
drude oscillator charges that get used in modeling the
induction energy. Though in principle \isa-based polarizabilities could be
used, this technique has not yet been developed. Instead, an \idma-type
procedure can be used to extract the necessary polarization parameters. The
algorithms used to perform this procedure are described in Appendix A of
\citen{McDaniel2014a}. Due to the reduced number of coefficients that need to
be fit, this optimization is generally more robust, and leads to more
transferable dispersion parameters, than the algorithms described in
\cref{sec:workflow-dispersion}.

\end{subsubsection}
\begin{subsubsection}{Practicals}

The drude oscillator fitting code has the same dependencies and input files as
\idma, with the exception that the \verb|dispersion_base_constraints.index|
file is replaced with the following constraint file:
\begin{enumerate}
\item \verb|drude_base_constraints.index|: As with \idma,
it is usually advisable to only fit a few atomtype static polarizabilities at a
time, with the remaining atomtype polarizabilities read in as hard
constraints. The \verb|drude_base_constraints.index| file lists these
hard constraints in a block format, 

\begin{minipage}{\linewidth}
%[float,floatplacement=H]
\begin{lstlisting}
C
1
0.0

N     
1
-11.7529643 

H     
1
-1.254

\end{lstlisting}
\end{minipage}
which lists each constrained atomtype along with the rank 1 static
polarizabilities, separated by a blank line. Unlike with the generation of
dispersion coefficients, an initial guess must be given for \emph{all}
atomtypes in the \verb|<monomer>.atomtypes| input file.
The format for the  \verb|drude_base_constraints.index| is such that
positive polarizabilities correspond to these initial guesses, whereas zero or
negative entries for the polarizabilities indicate that the atomtype should be
treated as a hard constraint.
Previously-fit atomtype polarizabilities from \citen{McDaniel2013} are already included in 
\verb|drude_base_constraints.index| so as to minimize the number of hard
constraints that the user will need to add manually, and these hard
constraints should be used whenever possible.
\end{enumerate}

Assuming that the \idma calculations have already been run in \camcasp, the
drude oscillator coefficients can be obtained simply by executing
%
\begin{lstlisting}
./scripts/workup_drude_files.sh
\end{lstlisting}
%
As with the dispersion coefficients, care should be taken to ensure that the
resulting drude oscillator charges are physically-meaningful (i.e. negative).

\end{subsubsection}


\end{subsection}
%%%%%%%%%%%%%%%% POLARIZATION %%%%%%%%%%%%%%%%%%%%%%%%%

\end{section}
