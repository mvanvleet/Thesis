\begin{section}{Dimer-Based Parameterization}
\label{sec:workflow-dimer_parameters}

After obtaining monomer parameters for a given system of interest, the final
task is to fit the remaining force field parameters to reproduce the
\dftsapt calculations performed earlier in the Workflow. Dimer-based
parameterization is carried out by the \pointer program, which will be the
subject of the next chapter. The Workflow is useful for preparing input
files for this dimer-based parameterization, as follows:
%
\begin{lstlisting}
./scripts/workup_sapt_energies.py
./scripts/gather_pointer_input_files.py
\end{lstlisting}
%
For modeling off-site point charges, the following additional steps are
required (assuming the offites .xyz file has already been added to the input
directory):
\begin{lstlisting}
cd geometries
mkdir xyz
mv * xyz
cd ../
./scripts/get_sapt_file_wi_offsites.py
\end{lstlisting}
%
The output of these scripts will generate a .sapt file (containing results
from the \dftsapt calculations, with atomtype
labels taken from each \verb|input/<monomer>.atomtypes| file) and a new
directory, \verb|ff_fitting|, which automatically sets up all of the input files and monomer
parameters needed to easily run the \pointer fitting code. The theory and
practice of the \pointer fitting code is the subject of the next chapter,
however in practice the software can be run very simply by modifying the
required input files (see \cref{sec:pointer-pointer} for details) and running
\begin{lstlisting}
cd ff_fitting
(modify input scripts)
./run_pointer.py
\end{lstlisting}

\end{section}

