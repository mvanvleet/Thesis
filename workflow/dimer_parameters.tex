
\begin{section}{Dimer-Based Parameterization}

After obtaining monomer parameters for a given system of interest, the final
remaining task is to fit the remaining force field parameters to reproduce the
\dftsapt calculations performed earlier in the Workflow. Dimer-based
parameterization is carried out by the \pointer program, which will be the
subject of the next chapter. The Workflow is useful for preparing input
files for this dimer-based parameterization, as follows:
%
\begin{lstlisting}
./scripts/workup_sapt_energies.py
./scripts/gather_pointer_input_files.py
\end{lstlisting}
%
The output of these scripts will generate a .sapt file (containing results
from the \dftsapt calculations, with atomtype
labels taken from each \verb|input/<monomer>.atomtypes| file) and a new
directory, \verb|ff_fitting|, which automatically sets up all of the input files and monomer
parameters needed to easily run the \pointer fitting code. The theory and
practice of the \pointer fitting code is the subject of the next chapter.

\end{section}
