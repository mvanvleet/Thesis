\begin{section}{Ab initio force field development with \acrshort{sapt} and
\acrshort{isa}}

As implied throughout the preceding discussion, goals for ab
initio force field development are as follows:
\begin{enumerate}
\item \textbf{Accuracy}: 
Ab initio force fields should
ideally be able to reproduce a benchamrk \pes (as calculated from high-quality
\est) to within chemical accuracy or better, with the knowledge that accuracy
compared to the \pes will be well-correlated with accuracy compared to
experiment
\item \textbf{Transferability}: The parameters and
functional forms used in ab inito force fields should be transferable between
chemical and physical environments without loss of accuracy
\item \textbf{Cost-Efficiency}: 
The computational cost of ab initio force fields should ideally be comparable to that
of empirically-derived models
\item \textbf{Physicality}: So as to minimize a reliance on error cancellation
and promote accuracy and transferability, functional forms and parameters for
ab initio force fields should be grounded in accurate and
physically-meaningful first principles theories
\item \textbf{Simplicity}: When possible, and where the accuracy and
physicality of the model is not compromised, the parameterization methodologies
and functional forms used in ab initio force field development should be kept
as simple as possible, particularly so as to avoid overfitting
\end{enumerate}

A number of strategies for ab inito force field development are present in the
literature,\cite{Stone2007,Ballone2014} however here we focus on the general
approach used in our group\cite{Schmidt2015} to generate
optimal ab initio force fields. Additionally,
as the intramolecular portion of a force field is
usually more straightforward to optimize,\cite{} we limit our discussion to
the functional forms and parameters used in developing the intermolecular part
of the potential.
In what follows, we describe three main strategies employed in our group to
guide ab inito force field development: separation of the $N$-body
potential into 2- and many-body contributions via the \mbe (\cref{sec:intro-mbe}), decomposition and
subsequent component-by-component parameterization of the total two-body
interaction energy using \sapt (\cref{sec:intro-sapt}), and characterization
of the atom-in-molecule contributions to each energy component via \isa
(\cref{sec:intro-isa}).

\begin{subsection}{The Many-Body Expansion}
\label{sec:intro-mbe}


For an $N$-body system (here and throughout we use the terms `body' and `atom'
synonymously), the molecular \pes is given as a $3N-6$ 
dimensional function of particle positions,
\cite{Stone2007,Cieplak2009,McDaniel2014,Elrodt1997}
%
\begin{align}
V_N(\vec r_1 ,\vec r_2 ,\dots,\vec r_N ) =
    \sum\limits_{i}^{N} V_1(\vec r_i) +
    \sum\limits_{i < j}^{N} \Delta V_2(\vec r_i, \vec r_j) +
    \sum\limits_{i < j < k}^{N} \Delta V_3(\vec r_i, \vec r_j, \vec r_k) +
\dots
\end{align}
%
Here, and without loss of generality, we have expressed this roughly $3N$-dimensional
surface as a `many-body' expansion of $n$-body cluster interactions.
Thus $V_1$ describes one-body, or intramolecular, contributions to the overall
\pes. $\Delta V_2$ is referred to as the `pair potential', and represents the
difference in interaction energies between a two-body cluster, or `dimer', and the
individual monomers themselves. In a similar fashion,
$\Delta V_3$ corresponds to
the non-additive contributions (energy not accounted for
in $\Delta V_2$) to the interaction energies of trimers, and $\Delta V_4$ and
higher-order terms are defined analagously. 

The utility of the \mbe comes from the fact that, aside from 
many-body polarization, for which the complete $N$-body effects can readily be
calculated,\cite{Stone2007,Rick2002} the \mbe is typically
rapidly convergent, and often only $\Delta V_2$ and
 $\Delta V_3$ terms are required to completely and accurately describe
$V_N$.\cite{Stone2007,stone2013theory} In fact, the combination of $\Delta
V_2$ and $N$-body polarization often account for
upwards of 90--95\%
of the total interaction energy,
\cite{McDaniel2014,stone2013theory}
such that the accuracy of a given ab initio force field depends primarily on
the accuracy of the pair potential itself. 
When
required, explicit terms for 
$\Delta V_3$ can easily be added to an ab initio force field as an additive
correction, and accurate models for  $\Delta V_3$ have been outline in
previous work.
\cite{McDaniel2014}
Nevertheless,
we can usually restrict our focus to the development of accurate models for $\Delta
V_2$, with the knowledge that accuracy in describing $\Delta V_2$ will have a
direct effect on accuracy with respect to $V_N$ and/or experiment. 

%TODO: Describe the practical advantages of the mbe?

\end{subsection}

\begin{subsection}{\sapt}
\label{sec:intro-sapt}

Having limited our attention to modeling the pair potential,  $\Delta V_2$,
a second technique we can employ in the development of ab initio force fields is
to fit our force field parameters on a component-by-component basis to a
physically-meaningful \eda of dimer interaction energies. Force field fitting on a component-by-component
basis enables the following:
\begin{enumerate}
\item By increasing the amount of ab initio data used in the force field fits,
we reduce the possibility of overfitting the potential, which in turn enables
transferability\cite{Schmidt2015}
\item By enforcing a one-to-one correspondence between force field functional
forms and benchmark ab initio energies, we reduce reliance on error
cancellation and ensure that all fitted parameters describe the intended physical
feature
\item By evaluating the resulting fits on a component-by-component basis, we
can directly relate errors in the potential to errors in the individual energy
comopnents, thus providing insight into how a current model might be improved
\end{enumerate}

In most cases (with the work described in \cref{ch:lmoeda} being an exception),
we use \acrfull{sapt} as our \eda of choice. \sapt, and \dftsapt in particular
(a variant of \sapt based on a \dft-based description of monomers, which
scales reasonably as $N^5$ with respect to the number of electrons in the
system), serves as an accurate yet affordable approximation to the
gold-standard \ccsdt calculations discussed earlier. Theories and formalisms
for \sapt are reviewed in \citens{Szalewicz2012,Jeziorski1994,McDaniel2014a},
and a variety of examples of \sapt-based ab initio force field development is given
in \citens{McDaniel2016a,Schmidt2015}.
Overall,
ab inito force fields
fit to \dftsapt energies have been shown to lead to good accuracy in experimental property
predictions,
\cite{McDaniel2016a,McDaniel2013}
thus justifying our approach.
Furthermore, and as is especially important in the development of
\emph{transferable} ab initio force fields, \sapt provides a natural and
physically-meaningful decomposition into energy components of electrostatics,
exchange, induction, and dispersion. By fitting these energy terms on a
component-by-component basis, the \sapt energy decomposition can be fully
taken advantage of to yield (as in \citen{McDaniel2013}) a library of accurate
and transferable force field parameters with broad applicability to a range of
chemical and physical environments.

\end{subsection}

\begin{subsection}{\isa-\acrshort{dma}}
\label{sec:intro-isa}

Especially in the asymptotic regime (i.e. at large
intermolecular separations), the pair potential can be described solely in
terms of properties that depend only on the identities of the individual monomers.
\cite{Stone2007,Metz2016}
These `monomer' properties range in scope from the monomer electron density
itself to the molecular polarizability, and these monomer properties define in
turn the electrostatic,
induction, and dispersion interactions at long-range. (Monomer properties can
also define interactions at short-range, and we
show in \cref{ch:isaff} how these quantities can help define parameters for exchange and
charge penetration effects.) Because we can
describe portions of the pair potential in terms of quantifiable ab initio monomer property
calculations, we should be able to considerably reduce the number of 
parameters required to fit a force field,
and thus increase the accuracy and transferability of the resulting model. 

Nevertheless, and although it is straightforward to calculate \emph{molecular} monomer
properties, it is usually advantageous\cite{Stone2007} to describe the pair
potential in terms of \emph{\aim} quantities, thus necessitating that we
\emph{partition} the results of each monomer property calculation so as to describe the
contribution of each atom in its molecular environment. \aim properties are
not experimental observables, and so it can be quite complicated to find an
ideal and physically-meaningful partitioning method for the purposes of force
field development.\cite{camcasp5.8,Misquitta2014,Stone2007,Pastorczak2017} 
Historically in our group, we have had reasonable success with using a 
\acrfull{dma} partitioning scheme, and methods for
obtaining distributed multipolar electrostatic,\cite{Stone1981,Stone2005}
polarization,\cite{Misquitta2006}
and
dispersion\cite{Williams2003,Misquitta2008,McDaniel2013}
parameters
from \dma are well-documented.\cite{McDaniel2014a,Stone2007}

More recently, \citet{Misquitta2014} has, in conjunction with important
contributions from
\citeauthor{Lillestolen2008},\cite{Lillestolen2008,Lillestolen2009} 
built upon the existing class of \citet{Hirshfeld1977} atom-in-molecule
charge partitioning schemes to develop an improved distribution scheme based on
\acrfull{isa}, 
termed \isa-\dma. 
In brief, \isa-\dma operates by partitioning a monomer
electron density into atomic contributions,
\begin{align}
\label{eq:intro-isa}
\rho_i(\mathbf{r}) = \rho_I(\mathbf{r})
\frac{ w_i(\mathbf{r}) }{ \sum \limits_{a \in I} w_a(\mathbf{r}) }, 
\end{align}
where $\rho(\mathbf{r})$ is an electron density, $w_a(\mathbf{r})$ is a
spherically-symmetric weight function which is iteratively determined in the
course of the \isa analysis, and lower- and upper-case subscripts
represent, respectively, \aim or molecular quantities.
\cite{Misquitta2014}
Using these atom-in-molecule charge densities, recent work has shown how
new and/or improved parameters for multipolar electrostatics 
(\cref{ch:workflow} and \citen{Misquitta2014}),
exchange-repulsion (\cref{ch:isaff} and \citen{VanVleet2016}),
and dispersion (\cref{ch:workflow})
are now possible.
Notably, ours is not the only group to take advantage of the \isa-\dma or related
methods,
\cite{Verstraelen2016}
and a number of ab intio force fields have been developed with the aid of such
distribution schemes.
\cite{Vandenbrande2016,Verstraelen2014,Misquitta2016,Metz2016}
% TODO: cite more force fields here?
Regardless, the \isa-\dma parameters have shown good promise for the
development of accurate and transferable force fields, and the manner in which
we can include these parameters in force field development will be a main
focus of this dissertation.

\end{subsection}


\end{section}
