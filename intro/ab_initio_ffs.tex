\begin{section}{Ab initio force field development with \acrshort{sapt} and
\acrshort{isa}}

As implied throughout the preceding discussion, our overall goals for ab
initio force field development can be summarized as follows:
\begin{enumerate}
\item \textbf{Accuracy}: 
Ab initio force fields should
ideally be able to reproduce a benchamrk \pes (as calculated from high-quality
\est) to within chemical accuracy or better, with the knowledge that accuracy
compared to the \pes will be well-correlated with accuracy compared to
experiment
\item \textbf{Transferability}: The parameters and
functional forms used ab inito force fields should be transferable between
chemical and physical environments without loss of accuracy
\item \textbf{Cost-Efficiency}: 
The computational cost of ab initio force fields should ideally be comparable to that
of empirically-derived models
\item \textbf{Physicality}: So as to minimize a reliance on error cancellation
and promote accuracy and transferability, functional forms and parameters for
ab initio force fields should be grounded in accurate and
physically-meaningful first principles theories
\item \textbf{Simplicity}: When possible, and where the accuracy and
physicality of the model is not compromised, the parameterization methodologies
and functional forms used in ab initio force field development should be kept
as simple as possible, particularly so as to avoid overfitting
\end{enumerate}

A number of strategies for ab inito force field development are present in the
literature,\cite{Stone2007,Ballone2014} however here we focus on the general
approach used in our group\cite{Schmidt2015} to meet the above goals for
developing optimal ab initio force fields. Additionally,
as the intramolecular portion of a force field is
usually more straightforward to optimize,\cite{} we limit our discussion to
the functional forms and parameters used in developing the intermolecular part
of the potential.
In what follows, we describe three main strategies employed in our group to
guide ab inito force field development: separation of the $N$-body
potential into 2- and many-body contributions via the \mbe (\cref{sec:intro-mbe}), decomposition and
subsequent component-by-component parameterization of the total two-body
interaction energy using \sapt (\cref{sec:intro-sapt}), and characterization
of the atomic contributions to each energy component via \isa
(\cref{sec:intro-isa}).

\begin{subsection}{The Many-Body Expansion}
\label{sec:intro-mbe}


For an $N$-body system (here and throughout we use the terms `body' and `atom'
synonymously), the molecular \pes is given as a $3N-6$ 
dimensional function of particle positions,
\cite{Stone2007,Cieplak2009,McDaniel2014,Elrodt1997}
%
\begin{align}
V_N(\vec r_1 ,\vec r_2 ,\dots,\vec r_N ) =
    \sum\limits_{i}^{N} V_1(\vec r_i) +
    \sum\limits_{i < j}^{N} \Delta V_2(\vec r_i, \vec r_j) +
    \sum\limits_{i < j < k}^{N} \Delta V_3(\vec r_i, \vec r_j, \vec r_k) +
\dots
\end{align}
%
Here, and without loss of generality, we have expressed this roughly $3N$-dimensional
surface as a `many-body' expansion in terms of $n$-body interacting clusters.
Thus $V_1$ describes one-body, or intramolecular, contributions to the overall
\pes, and $\Delta V_2$ is referred to as the `pair potential', which represents the
difference in interaction energies between a two-body cluster (i.e. dimer), and the
individual monomers themselves. In a similar fashion,
$\Delta V_3$ corresponds to
the non-additive contributions (that is, energy not accounted for
in $V_2$) to the interaction energies of 3-body clusters, and $\Delta V_4$ and
higher-order terms are defined analagously. 

The utility of the \mbe comes from the fact that, aside from 
many-body polarization, for which the complete $N$-body effects can readily be
calculated,\cite{Stone2007,Rick2002} the \mbe is typically
rapidly convergent, and often only $\Delta V_2$ and
 $\Delta V_3$ terms are required to completely and accurately describe
$V_N$.\cite{Stone2007,stone2013theory} In fact, the combination of $\Delta
V_2$ and $N$-body polarization often account for
upwards of 90--95\%
of the total interaction energy,
\cite{McDaniel2014,stone2013theory}
such that the accuracy of a given ab initio force field depends primarily on
the accuracy of the pair potential itself. 
When
required, explicit terms for 
$\Delta V_3$ can easily be added to an ab initio force field as an additive
correction, and accurate models for  $\Delta V_3$ have been outline in
previous work.
\cite{McDaniel2014}
Nevertheless,
we can usually restrict our focus to the development of accurate models for $\Delta
V_2$, with the knowledge that accuracy in describing $\Delta V_2$ will have a
direct effect on accuracy with respect to $V_N$ and/or experiment. 

%TODO: Describe the practical advantages of the mbe?

\end{subsection}

\begin{subsection}{\sapt}
\label{sec:intro-sapt}

Having limited our attention to modeling the pair potential,  $\Delta V_2$,
a second technique we can employ in the development of ab initio force fields is
to fit our force field parameters on a component-by-component basis to a
physically-meaningful \eda. Force field fitting on a component-by-component
basis enables the following:
\begin{enumerate}
\item By increasing the amount of ab initio data used in the force field fits,
we reduce the possibility of overfitting the potential, which in turn aids in
generating transferable force field parameters\cite{Schmidt2015}
\item By enforcing a one-to-one correspondence between force field functional
forms and benchmark ab initio energies, we reduce reliance on error
cancellation and ensure that all fitted parameters describe the intended physical
feature, further increasing parameter transferability
\item By evaluating the resulting fits on a component-by-component basis, we
can directly relate errors in the potential to errors in the individual energy
comopnents, thus systematically guiding further improvements to the ab initio
force field
\end{enumerate}

In most cases (with the work described in \cref{ch:lmoeda} being an exception,
we use \acrfull{sapt} as our \eda of choice. \sapt, and \dftsapt in particular
(a variant of \sapt based on a \dft-based description of monomers, which
scales reasonably as $N^5$ with respect to the number of electrons in the
system), serves as an accurate yet affordable approximation to the
gold-standard \ccsdt calculations discussed earlier, and ab inito force fields
fit to \dftsapt energies have been shown to lead to good experimental property
predictions.
\cite{McDaniel2016a,McDaniel2013}
Furthermore, and as is especially important in the development of
\emph{transferable} ab initio force fields, \sapt provides a natural and
physically-meaningful decomposition into energy components of electrostatics,
exchange, polarization, and dispersion.

\end{subsection}

\begin{subsection}{\isa}
\label{sec:intro-isa}

how is isa defined
how do distribution schemes work
what property calculations are possible using isa

\end{subsection}


\end{section}
