\begin{section}{Force Fields}

Despite the advantages and opportunities afforded by their computational efficiency, molecular
simulation can also be \emph{limited} by force fields in the sense that, in
the absence of fortuitous error cancellation, the predictive accuracy of
molecular simulation is inextricably tied to the accuracy of the force field
used to run the simulation. For this reason, one of the central challenges
facing molecular simulation today is the development of new and more accurate
force fields.\cite{Freddolino2010,Ballone2014}

To understand why the development of accurate force fields remains so
challenging, it's worthwhile to briefly discuss the development process
itself, both in terms of the functional forms that get used in force fields as
well as the manner in which these functional forms are parameterized.
Traditionally, force fields have been crafted by an `empirical' development
proccess,\cite{Stone2007}
in which the force field functional form is parameterized so as to reproduce
select experimental properties of interest. The obvious advantage of such a
strategy is that, so long as one parameterizes and investigates a limited
scope of chemical, physical, and/or structural conditions, 
there is a good chance that empirical force fields will be of good accuracy in
providing a microscopic picture of the macroscopic experimental properties.
Thus, for instance, empirical force fields can have 
remarkable success in simulating the behavior of folded proteins in
biologically-relevant environments.
\cite{Caleman2011,Piana2014,Lopes2015} 

Despite these successes, empirical force field development also faces
significant challenges. The first challenge is one of `transferability':
outside of the parameterization scope discussed above,
there is little
guarantee that empirical force fields will retain 
the good accuracy that can be expected within the original parameterization
conditions.
\cite{Hornak2006,Freddolino2010}
To continue with our protein example, it has recently been shown how many
empirical force fields, all of which generally provide similar predictions
regarding the properties of folded proteins, differ widely when it comes to
predicting a structurally-distinct class of partially unfolded,
`intrinsically-disorded' proteins.
\cite{Rauscher2015,Piana2014}
Similarly, and despite much effort, it's still difficult to find an
empirically-developed force field capable of correctly describing
water across a wide range of physical and chemical environments.
\cite{Cisneros2016a,Jorgensen2005}
More distressingly than the transferability problem, 
force field accuracy can sometimes be an
issue 
even within the limited range of experimental conditions over which the force
field was originally parameterized,
and time-consuming re-parameterization methods must often be employed
in order to correct for deficiencies in the original force field parameters.
\cite{Hornak2006,Freddolino2010}

Arguably, the underlying reason why empirical force field development is
limited (both in terms of accuracy and transferability) is that the
force fields themselves are typical based on rather limited, or `effective',
physics.
%TODO: Check (and maybe find more) references here
\cite{Parker2015,Sherrill2009,Zgarbova2010}
Explicit many-body polarization, for instance, is not often accounted for in
empirical force fields, despite the fact that it is known to be an important
factor in many important chemical phenomenon.
\cite{Cisneros2016a,Freddolino2010,Cieplak2009}
Similarly, accurate multipolar expansions of electrostatic energies are often
reduced to more approximate point charge models,\cite{Cardamone2014} 
charge penetration effects are usually neglected,
\cite{Parker2015,Sherrill2009}
and exchange effects are described by an overly-repulsive (but computationally
convenient) $1/r_{ij}^{12}$ functional form.
\cite{Abrahamson1963,Mackerell2004,Parker2015,Sherrill2009,Zgarbova2010}
In some cases, these modeling choices are justified by increased gains in
computational efficiency; indeed, it is only within the past decade or so that
explicit polarization and higher-order multipolar electrostatic
treatments have become computationally-affordable.
\cite{Albaugh2016,Cardamone2014,Cieplak2009,Simmonett2015,Simmonett2016,Demerdash2014}
In other cases, however, empirical force field development is limited by the
significant complications involved in parameterization.
With empirical force field development, each additional parameter must be
optimized on the basis of costly molecular simulations.
Moreover, and because experiment typically probes 
only the average total energy of a given system, parameters in empirical
force fields must be fit simultaneously. Models with many parameters are
usually too time-consuming to optimize, and too prone to issues of
overfitting,\cite{Hawkins2004} to warrant the effort. For these reasons, it is
likely that empirically-fit force fields will remain restricted to
parametrically simple, physically-approximate, models.

To circumvent the practical limitations of empirical force field development,
an alternate strategy is to fit force fields, not directly against
experimental properties, but rather to benchmark calculations of the
underlying \pes itself.\cite{Stone2007} 
The drawback of such a first-principles, or 'ab initio', methodology, is
obvious: by not fitting to experimental quantities, 
the resulting force fields will not closely match experiment unless we
accurately and systematically account for all the relevant physics for a given
system.
For this reason, comparisons between an ab initio force fields and experiment (\cref{ch:mastiff})
are often complicated by factors such as
the accuracy of the underlying
benchmark \pes or the treatment of many-body and/or quantum effects.
\cite{Johnson2009,Taylor2016,Chalasinski2000}

Nevertheless, ab initio force field development has several clear advantages
over its empirical counterpart. 
First, and especially for systems where experimental data is lacking, ab inito
force fields can be fit to calcuated data in order to make novel experimental
predictions. Furthermore (and as discussed in \cref{sec:intro-sapt}), ab
initio force fields can be fit, not merely to the total energy of a system,
but also on a component-by-component basis to individually reproduce each
physically-meaningful contribution to the \pes. This, along with the
simplicitly afforded by directly parameterizing the \pes,
means that ab initio force fields can be fit to more complicated and more
physically-motivated functional
forms, thus enabling the possibility of increased accuracy in molecular
simulation. Furthermore, we show in \cref{sec:intro-sapt} how advanced
ab initio parameterization methods can lead to decreased reliance on error
cancellation and minimize overfitting, thus augmenting both the accuracy and
transferability of the resulting force fields. Finally, with ab initio force
fields we can easily assess the fit quality compared to an underlying
benchmark \pes; 
as will be a theme of this dissertation, such an ability to
directly compare between putative model \glspl{pes} enables us
quickly evaluate new and improved functional forms and parameterization
methods for ab initio force field development. 

\end{section}


%% step of the parameter optimization process, the required comparisons to experiment 
%% depend on the results from molecular simulation , 
%% and so fitting parameters to an empirical force field is usually an incredibly
%% involved process. As a result, functional forms for empirical force fields
%% can depend on only a select few parameters, and very simple models for the
%% physics of molecular interactions are typical. 
%% With empirical force fields, the molecular \pes is typically
%% decomposed into `bonding' and 'non-bonding' contributions,
%% \begin{align}
%% U = U_b + U_{nb}.
%% \end{align}
%% The bonded energy term is designed to model intramolecular covalent
%% interactions, and is
%% generally represented in terms of two-, three-, and four-body contributions
%% between connected atoms,
%% \begin{align}
%% \begin{split}
%% U_b &= \frac12 \sum\limits_{(ij)} K^s_{ij}(r_{ij} - \bar r_{ij})^2
%% + \frac12 \sum\limits_{(ijk)} K^b_{ijk}(\theta_{ijk} - \bar \theta_{ijk})^2 \\
%% &\qquad + \frac12 \sum\limits_{(ijkl)} K^{\tau}_{ijkl} \left(1 + \cos(n\phi_{ijkl} - \bar
%% \phi_{ijkl}) \right)
%% \end{split}
%% \end{align}
%% where the various terms represent, respectively, the effects of bond
%% stretching, angle bending, and torsions.
%% \cite{Ballone2014}
%% Non-bonded interactions are approximately modeled via a simple
%% functional form intended to account for both
%% Coulombic and `van
%% der Waals' (repulsion plus dispersion) interactions,
%% \begin{align}
%% U_{nb} = \frac{1}{4\pi\epsilon_0} \sum\limits_{i\ne j} \frac{q_iq_j}{r_{ij}} +
%% \sum\limits_{i\ne j} 4\epsilon_{ij} \left [
%% \left(\frac{\sigma_{ij}}{r_{ij}}\right)^{12}
%% -\left(\frac{\sigma_{ij}}{r_{ij}}\right)^{6}
%% \right ]
%% \end{align}
%% where the first term is sometimes referred to as a `point charge' model, and the second
%% term is the popular Lennard-Jones 12-6 potential. Notably, while some of the
%% above non-bonded functional forms are grounded in approximate first-principles
%% theories, others (such as the
%% $\sfrac{1}{r_{ij}^{12}}$ repulsive term) are merely a computationally
%% convenient choice.
