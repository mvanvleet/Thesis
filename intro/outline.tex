\begin{section}{Outline}

Having described the utility of
molecular simulation and 
the important goals of accuracy and transferabilty in force field development,
the purpose of this dissertation is to describe new and better methods for
obtaining functional forms and parameters that will lead to improved ab initio force
field development. To this end,
\cref{ch:isaff} describes an approach whereby the \isa partitioning scheme can
be used to develop new models 
for the \sapt exchange-repulsion energy and related
short-range energy contributions. The methods in \cref{ch:isaff} neglect
important effects due to the orientation dependence, or `atomic-level
anisotropy', of the \isa charge densities, and \cref{ch:mastiff} extends the
original \isa-based method to account for this atomic-level anisotropy in an
accurate cost-efficient
manner amenable to large-scale molecular simulation.
As a third investigation of methods development for ab inito
force fields, \cref{ch:lmoeda} explores the ways in which additional \eda
methods (aside from \sapt) can be used to benchmark and parameterize ab initio
force fields in cases where \sapt itself is in error. Finally, in the course
of our research we have developed many automated tools and best practices for ab initio force 
field development (\sapt-based or otherwise), and these practical
considerations are the subject of \cref{ch:workflow,ch:pointer}. Overall
conclusions and avenues for future research are the subject of
\cref{ch:conclusions}. 


\end{section}
