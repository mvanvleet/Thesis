\begin{chapter}{Introduction}
\label{ch:intro}


What are the functions of proteins in the body? How can we identify new and better drugs for improved
disease treatment, or optimal materials for designing efficient solar cells?
What are the microscopic mechanisms by which chemicals interact, undergo phase
transitions, or react to form entirely new species? Increasingly, these and other 
essential chemical questions can be addressed with the aid of computer
simulation, 
\cite{VanGunsteren1990,Hospital2015,Chen2015,Jiang2011,Bereau2016,Karplus2002,Maurin2016}
enabling us to (as but a small subset of examples!) peer into the detailed mechanisms of enzyme catalysis,
\cite{Warshel2003}
watch proteins fold,
\cite{Levitt1975,Lane2013,Piana2014,Perez2016}
virtually screen for novel drug candidates, 
\cite{DeVivo2016}
and directly simulate hard-to-understand nucleation processes at an atomistic level.
\cite{Kalikmanov2013}
The question, of course, is: how?

Beginning as early as the 1950s,\cite{VanGunsteren1990} 


%% \begin{quote}
%% Although the laws governing the motions of atoms are quantum mechanical, the
%% key realization that made possible the simulation of the dynamics of complex
%% systems, including biomolecules, was that a classical mechanical description
%% of the atomic motions is adequate in most cases. \hfill Karplus
%% \end{quote}


-- quantum, not classical, mechanics
-- statistical mechanics: mapping between potential energy surface and
properties of interest
-- need two things: pes and equations of motion






========================================================



The aim of computer simulations of molecular systems is to compute macroscopic
behavior from microscopic interactions.
The main contributions a microscopic consideration can offer are (1) the
understanding and (2) interpretation of experimental results, (3)
semiquantitative estimates of experimental re- sults, and (4) the capability
to interpolate or extrapolate experimental data into regions that are only
difficultly accessible in the laboratory.
\cite{VanGunsteren1990}

The design of materials guided by computation is expected to
lead to the discovery of new materials, reduction of materials development
time and cost, and the rapid evolution of new materials into products.1
\cite{Chen2015}

Simulations can provide the ultimate detail concerning individ- ual particle
motions as a function of time. Thus, they can be used to address specific
questions about the properties of a model sys- tem, often more easily than
experiments on the actual system. For many aspects of biomolecular function,
it is these details that are of interest (for example, by what pathways does
oxygen enter into and exit from the heme pocket in myoglobin?). Of course,
experiments play an essential role in validating the simulation methodology:
comparisons of simulation and experimental data serve to test the accuracy of
the calculated results and to provide criteria for improving the methodology.
This is particularly important because theoretical estimates of systematic
errors inherent in simulations have not been possible — that is, the errors
introduced by the use of empirical potentials are difficult to quantify
...
There are three types of applications of simulation methods in the
macromolecular area, as well as in other areas involving mesoscopic systems.
The first uses simulation simply as a means of sampling configuration space.
This is involved in the utiliza- tion of molecular dynamics, often with
simulated annealing pro- tocols, to determine or refine structures with data
obtained from
experiments, as mentioned above. The second uses simulations to obtain a
description of the system at equilibrium, including structural and motional
properties (for example, atomic mean- square fluctuation amplitudes) and the
values of thermodynamic parameters. For such applications, it is necessary
that the simula- tions adequately sample configuration space, as in the first
appli- cation, with the additional condition that each point be weighted by
the appropriate Boltzmann factor. The third area uses simula- tions to examine
the actual dynamics. Here not only is adequate sampling of configuration space
with appropriate Boltzmann weighting required, but it must be done so as to
correctly repre- sent the development of the system over time. For the first
two areas, Monte Carlo simulations can be used, as well as molecular dynamics.
By contrast, in the third area where the motions and their development with
time are of primary interest, only molec- ular dynamics can provide the
necessary information. The three sets of applications make increasing demands
on simulation methods as to their required accuracy and precision.
\cite{Karplus2002}

\begin{section}{The Importance of Molecular Simulation}
This ref\cite{stone2013theory} is super cool!

What is molecular simulation?
What types of problems can it solve?
How does molecular simulation work? (Be sure to include solving Newton's EQs
of motion and relevant details on the partition function and interaction
energies!)

\end{section}





\end{chapter}
